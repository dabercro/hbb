\documentclass{beamer}

\author[D. Abercrombie]{
  Daniel Abercrombie
}

\title{\bf \sffamily $Z \rightarrow \nu\nu$ Update}
\date{\today}

\usecolortheme{dove}

\usepackage{listings}

\usepackage[absolute,overlay]{textpos}
\usefonttheme{serif}
\usepackage{appendixnumberbeamer}
\usepackage{isotope}
\usepackage{hyperref}
\usepackage[english]{babel}
\usepackage{amsmath}
\setbeamerfont{frametitle}{size=\Large,series=\bf\sffamily}
\setbeamertemplate{frametitle}[default][center]
\usepackage{siunitx}
\usepackage{tabularx}
\usepackage{makecell}

\setbeamertemplate{navigation symbols}{}
\usepackage{graphicx}
\usepackage{color}
\setbeamertemplate{footline}[text line]{\parbox{1.083\linewidth}{\footnotesize \hfill \insertshortauthor \hfill \insertpagenumber /\inserttotalframenumber}}
\setbeamertemplate{headline}[text line]{\parbox{1.083\linewidth}{\footnotesize \hspace{-0.083\linewidth} \textcolor{blue}{\sffamily \insertsection \hfill \insertsubsection}}}

\IfFileExists{/Users/dabercro/GradSchool/Presentations/MIT-logo.pdf}
             {\logo{\includegraphics[height=0.5cm]{/Users/dabercro/GradSchool/Presentations/MIT-logo.pdf}}}
             {\logo{\includegraphics[height=0.5cm]{/home/dabercro/MIT-logo.pdf}}}

\usepackage{changepage}

\newcommand{\beginbackup}{
  \newcounter{framenumbervorappendix}
  \setcounter{framenumbervorappendix}{\value{framenumber}}
}
\newcommand{\backupend}{
  \addtocounter{framenumbervorappendix}{-\value{framenumber}}
  \addtocounter{framenumber}{\value{framenumbervorappendix}}
}

\graphicspath{{figs/}}

\newcommand{\link}[2]{\href{#2}{\textcolor{blue}{\underline{#1}}}}
\newcommand{\clink}[2]{\link{#1}{http://t3serv001.mit.edu/~dabercro/redir/?k=#2}}}

\newcommand{\twofigs}[4]{
  \begin{columns}
    \begin{column}{0.5\linewidth}
      \centering
      \textcolor{blue}{#1} \\
      \includegraphics[width=\linewidth]{#2}
    \end{column}
    \begin{column}{0.5\linewidth}
      \centering
      \textcolor{blue}{#3} \\
      \includegraphics[width=\linewidth]{#4}
    \end{column}
  \end{columns}
}

\newcommand{\fourfigs}[8]{
  \begin{columns}
    \begin{column}{0.3\linewidth}
      \centering
      \textcolor{blue}{#1} \\
      \includegraphics[width=\linewidth]{#2} \\
      \textcolor{blue}{#3} \\
      \includegraphics[width=\linewidth]{#4}
    \end{column}
    \begin{column}{0.3\linewidth}
      \centering
      \textcolor{blue}{#5} \\
      \includegraphics[width=\linewidth]{#6} \\
      \textcolor{blue}{#7} \\
      \includegraphics[width=\linewidth]{#8}
    \end{column}
  \end{columns}
}

\newcommand{\ttbar}{\ensuremath{t\bar{t}}}

\begin{document}

\begin{frame}[nonumbering]
  \titlepage
\end{frame}

\section{$Z \nu\nu Hbb$}

\begin{frame}
  \frametitle{Outline}

  \begin{itemize}
  \item Regression for b-jets
  \item Event Classifier
  \item Disconcerting Plots
  \item Expected Limits
  \end{itemize}

\end{frame}

\subsection{Regression for b-jets}

\begin{frame}
  \frametitle{Event Selection}
  In order to train on events that do not overlap with
  my\footnote{May need to revisit cuts to select events orthogonal to all analyses}
  portion of the analysis,
  I select for fully leptonic \ttbar.
  
  \begin{itemize}
  \item $U > \SI{150}{GeV}$
  \item Exactly two loose leptons, at least one tight
  \item Jet with $p_T > \SI{30}{GeV}$
  \item This jet is matched with a b-flavored gen jet
    \begin{itemize}
    \item I don't care if the regression is wrong for other flavor jets
    \end{itemize}
  \end{itemize}

\end{frame}

\begin{frame}
  \frametitle{Adding $\nu$ back in}

  I do not recluster with all gen particles again.

  \begin{enumerate}
  \item Waste of time
  \item Will change direction and matching of jets
  \end{enumerate}

  Method of adding back in is simple

  \begin{itemize}
  \item Loop through gen particles and find neutrinos
  \item Find closest jet with anti-$k_T$ metric\footnote{
    Basically $\Delta R$ when $p_{T,jet} > p_{T,\nu}$; beam distance with $R = 0.4$}
  \item If a neutrino coming from a massive $W$\footnote{$m_W > \SI{50}{GeV}$},
    skip neutrino\footnote{Remeber, we're looking at fully leptonic \ttbar},
    and save flag for fun
  \item Add neutrino to closest jet
  \end{itemize}

  Note, that $\nu$ from massive $W$s overlap with selected b-jet $\approx 1\%$ of the time.

\end{frame}

\begin{frame}
  \frametitle{Training variables}

  Train on the jet with the highest CMVA value, with the following variables

  \begin{columns}
    \begin{column}{0.33\linewidth}
      \begin{itemize}
      \item NPV
      \item \# central jets
      \item \# total jets
      \item $p_T$
      \item $\eta$
      \item $\phi$
      \item mass
      \end{itemize}
    \end{column}
    \begin{column}{0.33\linewidth}
      \begin{itemize}
      \item CSV
      \item CMVA
      \item QGL
      \item SV \# tracks
      \item SV $p_T$
      \item SV mass
      \item SV distance
      \item SV uncertainty
      \end{itemize}
    \end{column}
    \begin{column}{0.33\linewidth}
      \begin{itemize}
      \item Max track $p_T$
      \item \# leptons
      \item lead lepton $p_T$
      \item LL $p_{T, \perp}$
      \item LL $\Delta R$
      \item EM fraction
      \item NH fraction
      \item CH fraction
      \end{itemize}
    \end{column}
  \end{columns}

\end{frame}

\begin{frame}
  \frametitle{Regression Performance (continues)}

  These plots are of the $Z\nu\nu Hbb$ signal sample with the signal selection.

  \begin{center}
    \includegraphics[width=0.6\linewidth]{180209_v1/regression_perform.pdf}
  \end{center}

  The new regression seems to recover a little bit more of the $p_T$ than the old regression.

\end{frame}

\begin{frame}
  \frametitle{Regression Performance (continued)}

  More importantly, the Higg's mass peak is improved

  \twofigs{Masses with Gen Mass}
          {180209_v1/regression_withgen.pdf}
          {Without Gen (Zoomed)}
          {180209_v1/regression.pdf}

  Even not accounting for potentially different gen jet definitions, the peak is higher

\end{frame}

\begin{frame}
  \frametitle{Stat Dump}
  \lstinputlisting{stat.txt}

\end{frame}

\subsection{Event Classifier}

\begin{frame}
  \frametitle{Event Classification}

  Next, we want to train a BDT to classify events for the ``limits.''

  I do a single signal v.~background classification
  \begin{itemize}
  \item At a glance at the note I only saw a single classifier
  \item I wanted to get limits done fast after vaction
  \item I have no experience with multiclass yet
  \end{itemize}

  Selection I made is kind of weird

  \begin{itemize}
  \item I did not want to train in signal.
    otherwise we're fitting in phase spaces that the BDT knew nothing about
  \item I made the selection back when I thought I still needed QCD control region
  \item Wanted to cut out a lot, but not all, of QCD and \ttbar
  \item Some cuts are looser than preselection\footnote{\emph{e.g.} jet $p_T$ and CMVA}
    to avoid shoulders
  \item Also before I saved PUID bools
  \end{itemize}

\end{frame}

\begin{frame}
  \frametitle{Classifier Cuts}
  \begin{itemize}
  \item $\mathrm{MET} > \SI{150}{GeV}$
  \item leading jet $\mathrm{CHF} > 0.15; \mathrm{EMF} < 0.8$
  \item leading jet $p_T > \SI{40}{GeV}$
  \item dijet $p_T > \SI{100}{GeV}$
  \item loose  $\mathrm{CMVA} > -0.6$\footnote{Loose working point is $-0.5884$}
  \item $n_{jets} < 5$
  \item $\Delta\phi_{min}(j, \mathrm{MET}) > 0.5$
  \end{itemize}
\end{frame}

\begin{frame}
  \frametitle{Classifier Variables}
  \begin{columns}
    \begin{column}{0.5\linewidth}
      \begin{itemize}
      \item Dijet raw mass
      \item Dijet regressed mass
      \item Dijet $p_T$
      \item Dijet $p_T$ over recoil
      \item Daughter $\Delta R$
      \item Daughter $\Delta \phi$
      \item MET (I realize recoil would be better)
      \end{itemize}
    \end{column}
    \begin{column}{0.5\linewidth}
      \begin{itemize}
      \item Highest CMVA
      \item Second CMVA
      \item Third CMVA (I don't like this either)
      \item Highest CMVA jet's $p_T$
      \item Second CMVA jet's $p_T$
      \item Third CMVA jet's $p_T$
      \item $\Delta \phi (jj, U)$
      \end{itemize}
    \end{column}
  \end{columns}
\end{frame}

\begin{frame}
  \frametitle{Selection Results}
  We have some \ttbar~and QCD, but not an overwhelming amount.

  My selection is not validated by any optimization study\footnote{
    I am not worried about discrimination, mostly data-MC agreement}.

  \centering
  \includegraphics[width=0.6\linewidth]{180209_v1/ZvvHbb_classify_cmva_hbb_m.pdf}

\end{frame}

\begin{frame}
  \frametitle{Classifier Output for Different Control Regions}

  \fourfigs{signal}
           {180209_v1/quick_signal_event_class.pdf}
           {\ttbar}
           {180209_v1/quick_tt_event_class.pdf}
           {Z + heavy jets}
           {180209_v1/quick_heavyz_event_class.pdf}
           {Z + light jets}
           {180209_v1/quick_lightz_event_class.pdf}

\end{frame}

\begin{frame}
  \frametitle{Signal Region Shapes}

  The classifier definitely makes a ``nicer'' peak than the mass
  for the eye (though in a region with low yields)

  \twofigs{Regressed Mass}
          {180209_v1/ZvvHbb_signal_cmva_hbb_m_reg.pdf}
          {BDT Output}
          {180209_v1/ZvvHbb_signal_event_class.pdf}

  Hopefully the data matches.

\end{frame}

\subsection{Disconcerting Plots}

\begin{frame}
  \frametitle{CMVA in Data and MC}

  These are in the $Z$ + heavy flavor region

  \twofigs{Highest CMVA jet}
          {180209_v1/ZvvHbb_heavyz_cmva_jet1_cmva.pdf}
          {Second CMVA jet}
          {180209_v1/ZvvHbb_heavyz_cmva_jet2_cmva.pdf}

  Tight working point is especially bad for me

\end{frame}

\begin{frame}
  \frametitle{Trying to Match}

  Changed some of the cuts in heavy flavor and signal,
  and \ttbar crept in.

  \twofigs{Dijet mass}
          {180209_v1/ZvvHbb_heavyz_cmva_hbb_m.pdf}
          {Event Classifier}
          {180209_v1/ZvvHbb_heavyz_event_class.pdf}

  They are also cutting on classification BDT,
  so I'll have to scan to match efficiency.
  May also have to adjust classification philosophy.

\end{frame}

\begin{frame}
  \frametitle{Light jet $\eta$}
  For some reason, jet $\eta$ looks strange in the $Z$ + light jets region.

  \begin{center}
    \includegraphics[width=0.5\linewidth]{180209_v1/ZvvHbb_lightz_cmva_jet1_eta.pdf}
  \end{center}

  I hope it's a cleaning problem.
  Same problem does not exist in other control regions. (See backup slides)
\end{frame}

\begin{frame}
  \frametitle{Classifier Trends}
  Classifier region with signal veto

  \begin{center}
    \includegraphics[width=0.5\linewidth]{180209_v1/ZvvHbb_classifyHveto_event_class.pdf}
  \end{center}

  There seems to be a definite trend in Data vs MC, which I don't like
\end{frame}

\subsection{Expected Limits}

\begin{frame}
  \frametitle{Uncertainties}
  Let's make limits anyway.

  \vspace{24pt}

  I include the following uncertainties.

  \begin{itemize}
  \item Luminosity 2.5\%
  \item Pileup 5\% (Might be garbage I have sitting around)
  \item PDF 30\%
  \item $W$ + $Z$ $k$-factor \tttext{ren} + \tttext{fact} uncertainties \\
    (Four independent shape uncertainties)
  \item $VH$ EWK uncertainties
  \item b-tag calibration
  \end{itemize}

  Don't have the diboson and single top uncertainties yet,
  but will make that addition while rewriting datacard making tool

\end{frame}

\begin{frame}
  \frametitle{Expected Limits}
  \lstinputlisting{lim.txt}

  Changing which dijet mass is cut on (regressed vs. raw)
  does not change the expected limits.
\end{frame}

\section{$Z \nu\nu Hbb$}

\begin{frame}
  \frametitle{Conclusions}
  \begin{itemize}
  \item b-jet regression looks promising with little effort
    \begin{itemize}
    \item I hope to create many different regressions to compare
    \end{itemize}
  \item Classifiers may be needed for cross purposes
    \begin{itemize}
    \item One for cleaning signal and heavy region
    \item One for good Data--MC agreement
    \end{itemize}
  \item Need to check cleaning and calibration until jet $\eta$ is fixed
  \item Limits exist, but not synced with anything yet
  \end{itemize}
\end{frame}

\beginbackup

\begin{frame}
  \frametitle{Backup Slides}
\end{frame}

\begin{frame}
   \frametitle{\small 190611/plot\_time\_60000\_wide}
   \centering
   \includegraphics[width=0.6\linewidth]{190611/plot_time_60000_wide.pdf}
\end{frame}

\begin{frame}
   \frametitle{\small 190611/plot\_time\_wide}
   \centering
   \includegraphics[width=0.6\linewidth]{190611/plot_time_wide.pdf}
\end{frame}

\begin{frame}
   \frametitle{\small 190611/plot\_time\_120000\_compare}
   \centering
   \includegraphics[width=0.6\linewidth]{190611/plot_time_120000_compare.pdf}
\end{frame}

\begin{frame}
   \frametitle{\small 190611/plot\_time\_60000\_compare}
   \centering
   \includegraphics[width=0.6\linewidth]{190611/plot_time_60000_compare.pdf}
\end{frame}

\begin{frame}
   \frametitle{\small 190611/plot\_time\_80000\_compare}
   \centering
   \includegraphics[width=0.6\linewidth]{190611/plot_time_80000_compare.pdf}
\end{frame}

\begin{frame}
   \frametitle{\small 190611/plot\_time\_120000\_narrow}
   \centering
   \includegraphics[width=0.6\linewidth]{190611/plot_time_120000_narrow.pdf}
\end{frame}

\begin{frame}
   \frametitle{\small 190611/plot\_time\_40000\_narrow}
   \centering
   \includegraphics[width=0.6\linewidth]{190611/plot_time_40000_narrow.pdf}
\end{frame}

\begin{frame}
   \frametitle{\small 190611/plot\_time\_160000\_narrow}
   \centering
   \includegraphics[width=0.6\linewidth]{190611/plot_time_160000_narrow.pdf}
\end{frame}

\begin{frame}
   \frametitle{\small 190611/plot\_time\_60000\_narrow}
   \centering
   \includegraphics[width=0.6\linewidth]{190611/plot_time_60000_narrow.pdf}
\end{frame}

\begin{frame}
   \frametitle{\small 190611/plot\_time\_80000\_narrow}
   \centering
   \includegraphics[width=0.6\linewidth]{190611/plot_time_80000_narrow.pdf}
\end{frame}

\begin{frame}
   \frametitle{\small 190611/plot\_time\_narrow}
   \centering
   \includegraphics[width=0.6\linewidth]{190611/plot_time_narrow.pdf}
\end{frame}



\backupend

\end{document}
