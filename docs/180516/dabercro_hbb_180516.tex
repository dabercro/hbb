\documentclass{beamer}

\author[D. Abercrombie]{
  \emph{Daniel Abercrombie}, Brandon Allen, Zeynep Demiragli,
  Guillelmo Gomez-Ceballos, Dylan George Hsu, \\
  Philip Harris, Yutaro Iiyama, Benedikt Maier, \\
  Siddharth Narayanan, Christoph Paus
}

\title{\bf \sffamily Z$\nu\nu$Hbb Modifications and \\ 2017 Data Outlook}
\date{\today}

\usecolortheme{dove}

\usepackage[absolute,overlay]{textpos}
\usefonttheme{serif}
\usepackage{appendixnumberbeamer}
\usepackage{isotope}
\usepackage{hyperref}
\usepackage[english]{babel}
\usepackage{amsmath}
\setbeamerfont{frametitle}{size=\Large,series=\bf\sffamily}
\setbeamertemplate{frametitle}[default][center]
\usepackage{siunitx}
\usepackage{tabularx}
\usepackage{makecell}

\setbeamertemplate{navigation symbols}{}
\usepackage{graphicx}
\usepackage{color}
\setbeamertemplate{footline}[text line]{\parbox{1.083\linewidth}{\footnotesize \hfill \insertshortauthor \hfill \insertpagenumber /\inserttotalframenumber}}
\setbeamertemplate{headline}[text line]{\parbox{1.083\linewidth}{\footnotesize \hspace{-0.083\linewidth} \textcolor{blue}{\sffamily \insertsection \hfill \insertsubsection}}}

\IfFileExists{/Users/dabercro/GradSchool/Presentations/MIT-logo.pdf}
             {\logo{\includegraphics[height=0.5cm]{/Users/dabercro/GradSchool/Presentations/MIT-logo.pdf}}}
             {\logo{\includegraphics[height=0.5cm]{/home/dabercro/MIT-logo.pdf}}}

\usepackage{changepage}

\newcommand{\beginbackup}{
  \newcounter{framenumbervorappendix}
  \setcounter{framenumbervorappendix}{\value{framenumber}}
}
\newcommand{\backupend}{
  \addtocounter{framenumbervorappendix}{-\value{framenumber}}
  \addtocounter{framenumber}{\value{framenumbervorappendix}}
}

\graphicspath{{figs/}}

\newcommand{\link}[2]{\href{#2}{\textcolor{blue}{\underline{#1}}}}
\newcommand{\clink}[2]{\link{#1}{http://t3serv001.mit.edu/~dabercro/redir/?k=#2}}}

\newcommand{\twofigs}[4]{
  \begin{columns}
    \begin{column}{0.5\linewidth}
      \centering
      \textcolor{blue}{#1} \\
      \includegraphics[width=\linewidth]{#2}
    \end{column}
    \begin{column}{0.5\linewidth}
      \centering
      \textcolor{blue}{#3} \\
      \includegraphics[width=\linewidth]{#4}
    \end{column}
  \end{columns}
}

\newcommand{\fourfigs}[8]{
  \begin{columns}
    \begin{column}{0.3\linewidth}
      \centering
      \textcolor{blue}{#1} \\
      \includegraphics[width=\linewidth]{#2} \\
      \textcolor{blue}{#3} \\
      \includegraphics[width=\linewidth]{#4}
    \end{column}
    \begin{column}{0.3\linewidth}
      \centering
      \textcolor{blue}{#5} \\
      \includegraphics[width=\linewidth]{#6} \\
      \textcolor{blue}{#7} \\
      \includegraphics[width=\linewidth]{#8}
    \end{column}
  \end{columns}
}

\newcommand{\ttbar}{\ensuremath{t\bar{t}}}

\begin{document}

\begin{frame}[nonumbering]
  \titlepage
\end{frame}

\begin{frame}
  \frametitle{Outline}
  \tableofcontents
\end{frame}

\section{Additional Cuts for Closer Yields}

\begin{frame}
  \frametitle{Some Final Sync Attempts}

  Two more inconsistencies between our understanding of the AN
  and the VHbb's Heppy framework (as seen on GitHub) were found.

  \begin{enumerate}
  \item Track MET (used in a cut on $\Delta\phi$(trkMET, MET)): \\
    Calculated only using the tracks with $\Delta z < 0.5$ from the Primary Vertex,
    while MINIAOD seems to store Track MET with all tracks
  \item Relative Isolation: \\
    We followed the explicit formula in \S4.4 of the AN for lepton isolation,
    but code on GitHub seems to take the minimum value between this definition
    and ``miniRelIso'', which has a varying cone size.
  \end{enumerate}

  We have changed both things in a final attempt to sync.

  \vspace{12pt}

  Yields still do not match the paper, but brings limit down by \textcolor{red}{[SOME NUMBER]?}

\end{frame}

\section{Z$\nu\nu$ Limit Improvements}

\begin{frame}
  \frametitle{Investigating Cuts}

  There is an asymmetric $p_T$ cut on the di-jet system in the cutstring
  supplied by the 2016 analyzers and various branches on GitHub
  (I will call this ``old cut''):

  \begin{gather*}
    p_{T,1} > \SI{60}{GeV} \\
    p_{T,2} > \SI{35}{GeV}
  \end{gather*}

  where $p_{T,1}$ refers to the $p_T$ of the jet with a higher cMVA score.

  \vspace{12pt}

  The harder jet does not always have a higher cMVA score,
  So we investigated the cut (christened ``new cut''):

  \begin{gather*}
%    \mathrm{max}(p_{T,1}, p_{T,2}) > \SI{60}{GeV} \\
%    \mathrm{min}(p_{T,1}, p_{T,2}) > \SI{35}{GeV}
    p_{T,1} > \SI{60}{GeV} \quad \mathrm{or} \quad p_{T,2} > \SI{60}{GeV} \\
    p_{T,1} > \SI{35}{GeV} \quad \mathrm{and} \quad p_{T,2} > \SI{35}{GeV}
  \end{gather*}

\end{frame}

\begin{frame}
  \frametitle{Yields and Statistical Significances}

  \begin{center}
    \begin{tabular}{|l|r|r|r|r|}
      \hline
      Cut & Signal Yield & Data Yield & Purity & $S/\sqrt{D}$ \\
      \hline
      old cut & 57.11 & 4032 & 1.4\% & 0.90 \\
      new cut & 69.15 & 5326 & 1.3\% & 0.95 \\
      \hline
    \end{tabular}
  \end{center}

  Purity decreases, however there is a 5\% sensitivity increase
  to a stat-uncertainty-only cut-and-count experiment.

  \vspace{12pt}

  Let's quickly run through the shape analysis (with systematics)

\end{frame}

\begin{frame}
  \frametitle{Post-fit Plots (1/2)}

  \twofigs{signal}
          {180514_v1/inclusive_signal_maier_event_class.pdf}
          {\ttbar}
          {180514_v1/inclusive_tt_cmva_jet2_cmva.pdf}
           
  \vspace{12pt}
  Agreement still looks good, CR pre-fit plots are in the backup

\end{frame}

\begin{frame}
  \frametitle{Post-fit Plots (2/2)}

  \twofigs{Z + Heavy Flavor}
          {180514_v1/inclusive_heavyz_cmva_jet2_cmva.pdf}
          {Z + Light Flavor}
          {180514_v1/inclusive_lightz_cmva_jet2_cmva.pdf}

  \vspace{12pt}
  Agreement still looks good, CR pre-fit plots are in the backup

\end{frame}

\begin{frame}
  \frametitle{Result}
  Running an expected limit over just the Z$\nu\nu$ category with the new cut gives a 5\% improvement over the old cut
\end{frame}

\begin{frame}
  \frametitle{Check on Jets (post-fit plots)}
  Other CRs are in backup (they look fine to me)

  \twofigs{Signal}
          {180514_old/inclusive_signal_cmva_jet1_pt.pdf}
          {Z + Heavy Flavor}
          {180514_old/inclusive_heavyz_cmva_jet1_pt.pdf}

  \begin{itemize}
  \item Plots only have stat uncertainties
  \item $W + b$ process is driven up dramatically by fit,
    so WH combination may change the spectrum
  \end{itemize}

\end{frame}

\section{Combination with WH}

\begin{frame}
  \frametitle{Combination with WH}

  Benedikt ran the combination between WH and Z$\nu\nu$.

  \vspace{12pt}

  Limits were \textcolor{red}{[SOME NUMBER]}

  \vspace{12pt}

  Z$\ell\ell$ will be ready to combine soon.

\end{frame}

\section{A Difference in 2017}

\begin{frame}
  \frametitle{New Triggers in 2017}

  2016 Analysis used:

  \vspace{12pt}

  {\setlength{\parindent}{0cm} \ttfamily
    HLT\_PFMETNoMu110\_PFMHTNoMu110\_IDTight
    HLT\_PFMETNoMu120\_PFMHTNoMu120\_IDTight
    HLT\_PFMET170\_NoiseCleaned
    HLT\_PFMET170\_HBHECleaned
    HLT\_PFMET170\_HBHE\_BeamHaloCleaned
  }

  \vspace{24pt}

  The PFMET170 triggers are gone in 2017, and we first try:

  \vspace{12pt}

  {\setlength{\parindent}{0cm} \ttfamily
    HLT\_PFMETNoMu120\_PFMHTNoMu120\_IDTight\_PFHT60
    HLT\_PFMETNoMu120\_PFMHTNoMu120\_IDTight
    HLT\_PFMETNoMu130\_PFMHTNoMu130\_IDTight
    HLT\_PFMETNoMu140\_PFMHTNoMu140\_IDTight
  }
\end{frame}

\begin{frame}
  \frametitle{NPV reweighting}

  Using an independent rough measurement for NPV reweighting on an inclusive
  signal, Z+HF, and Z+LF selection

  \vspace{12pt}

  \twofigs{Before reweighting}
          {180514_quick/npv.pdf}
          {After reweighting}
          {180509_v1/npv.pdf}

\end{frame}

\begin{frame}
  \frametitle{Comparison Plots}

  Jet kinematics look good, but MET turn-on appears slower in the newer data

  \vspace{12pt}

  \twofigs{Number of central jets}
          {180509_v1/n_centerjet.pdf}
          {PF MET}
          {180509_v1/pfmet.pdf}

  \vspace{12pt}

  Other plots are in the backup

\end{frame}

\begin{frame}
  \frametitle{Same trigger}

  \textcolor{red}{Here, I want to place a met plot with the one trigger path that exists in 2016 and 2017: HLT\_PFMETNoMu120\_PFMHTNoMu120\_IDTight}

\end{frame}

\section*{}

\begin{frame}
  \frametitle{Conclusion}
  \begin{itemize}
  \item There is wisdom on GitHub that did not make it to the AN
  \item It is worth some time to revisit 2016 cuts
  \item Combinations are under way for the various VH processes,
    which is prerequisite for accurate assessment of improvements brought by boosted analysis
  \item MET triggers in 2017 data will lead to harder Z$\nu\nu$ events
  \end{itemize}

  We have now processed enough MC with 2017 run conditions to start plotting
\end{frame}

\beginbackup

\begin{frame}
  \frametitle{Backup Slides}

  \centering
  \textcolor{red}{All in random (auto-generated) orders and names. \\ Should fix before Wednesday.}

\end{frame}

\begin{frame}
   \frametitle{\small 190611/plot\_time\_60000\_wide}
   \centering
   \includegraphics[width=0.6\linewidth]{190611/plot_time_60000_wide.pdf}
\end{frame}

\begin{frame}
   \frametitle{\small 190611/plot\_time\_wide}
   \centering
   \includegraphics[width=0.6\linewidth]{190611/plot_time_wide.pdf}
\end{frame}

\begin{frame}
   \frametitle{\small 190611/plot\_time\_120000\_compare}
   \centering
   \includegraphics[width=0.6\linewidth]{190611/plot_time_120000_compare.pdf}
\end{frame}

\begin{frame}
   \frametitle{\small 190611/plot\_time\_60000\_compare}
   \centering
   \includegraphics[width=0.6\linewidth]{190611/plot_time_60000_compare.pdf}
\end{frame}

\begin{frame}
   \frametitle{\small 190611/plot\_time\_80000\_compare}
   \centering
   \includegraphics[width=0.6\linewidth]{190611/plot_time_80000_compare.pdf}
\end{frame}

\begin{frame}
   \frametitle{\small 190611/plot\_time\_120000\_narrow}
   \centering
   \includegraphics[width=0.6\linewidth]{190611/plot_time_120000_narrow.pdf}
\end{frame}

\begin{frame}
   \frametitle{\small 190611/plot\_time\_40000\_narrow}
   \centering
   \includegraphics[width=0.6\linewidth]{190611/plot_time_40000_narrow.pdf}
\end{frame}

\begin{frame}
   \frametitle{\small 190611/plot\_time\_160000\_narrow}
   \centering
   \includegraphics[width=0.6\linewidth]{190611/plot_time_160000_narrow.pdf}
\end{frame}

\begin{frame}
   \frametitle{\small 190611/plot\_time\_60000\_narrow}
   \centering
   \includegraphics[width=0.6\linewidth]{190611/plot_time_60000_narrow.pdf}
\end{frame}

\begin{frame}
   \frametitle{\small 190611/plot\_time\_80000\_narrow}
   \centering
   \includegraphics[width=0.6\linewidth]{190611/plot_time_80000_narrow.pdf}
\end{frame}

\begin{frame}
   \frametitle{\small 190611/plot\_time\_narrow}
   \centering
   \includegraphics[width=0.6\linewidth]{190611/plot_time_narrow.pdf}
\end{frame}



\backupend

\end{document}
