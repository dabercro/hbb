\documentclass{beamer}

\author[D. Abercrombie]{
  \emph{Daniel Abercrombie}, Brandon Allen, Zeynep Demiragli,
  Guillelmo Gomez-Ceballos, Dylan George Hsu, \\
  Philip Harris, Yutaro Iiyama, Benedikt Maier, \\
  Siddharth Narayanan, Christoph Paus
}

\title{\bf \sffamily Z$\nu\nu$Hbb Modifications and \\ 2017 Data Outlook}
\date{May 16, 2018}

\usecolortheme{dove}

\usepackage[absolute,overlay]{textpos}
\usefonttheme{serif}
\usepackage{appendixnumberbeamer}
\usepackage{isotope}
\usepackage{hyperref}
\usepackage[english]{babel}
\usepackage{amsmath}
\setbeamerfont{frametitle}{size=\Large,series=\bf\sffamily}
\setbeamertemplate{frametitle}[default][center]
\usepackage{siunitx}
\usepackage{tabularx}
\usepackage{makecell}

\definecolor{mitred}{RGB}{163,31,52}

\setbeamertemplate{navigation symbols}{}
\usepackage{graphicx}
\usepackage{color}
\setbeamertemplate{footline}[text line]{\parbox{1.083\linewidth}{\footnotesize \hfill \insertshortauthor \hfill \insertpagenumber /\inserttotalframenumber}}
\setbeamertemplate{headline}[text line]{\parbox{1.083\linewidth}{\footnotesize \hspace{-0.083\linewidth} \textcolor{mitred}{\sffamily \insertsection \hfill \insertsubsection}}}

\IfFileExists{/Users/dabercro/GradSchool/Presentations/MIT-logo.pdf}
             {\logo{\includegraphics[height=0.5cm]{/Users/dabercro/GradSchool/Presentations/MIT-logo.pdf}}}
             {\logo{\includegraphics[height=0.5cm]{/home/dabercro/MIT-logo.pdf}}}

\usepackage{changepage}

\newcommand{\beginbackup}{
  \newcounter{framenumbervorappendix}
  \setcounter{framenumbervorappendix}{\value{framenumber}}
}
\newcommand{\backupend}{
  \addtocounter{framenumbervorappendix}{-\value{framenumber}}
  \addtocounter{framenumber}{\value{framenumbervorappendix}}
}

\graphicspath{{figs/}}

\newcommand{\link}[2]{\href{#2}{\textcolor{blue}{\underline{#1}}}}
\newcommand{\clink}[2]{\link{#1}{http://t3serv001.mit.edu/~dabercro/redir/?k=#2}}}

\newcommand{\twofigs}[4]{
  \begin{columns}
    \begin{column}{0.5\linewidth}
      \centering
      \textcolor{blue}{#1} \\
      \includegraphics[width=\linewidth]{#2}
    \end{column}
    \begin{column}{0.5\linewidth}
      \centering
      \textcolor{blue}{#3} \\
      \includegraphics[width=\linewidth]{#4}
    \end{column}
  \end{columns}
}

\newcommand{\fourfigs}[8]{
  \begin{columns}
    \begin{column}{0.3\linewidth}
      \centering
      \textcolor{blue}{#1} \\
      \includegraphics[width=\linewidth]{#2} \\
      \textcolor{blue}{#3} \\
      \includegraphics[width=\linewidth]{#4}
    \end{column}
    \begin{column}{0.3\linewidth}
      \centering
      \textcolor{blue}{#5} \\
      \includegraphics[width=\linewidth]{#6} \\
      \textcolor{blue}{#7} \\
      \includegraphics[width=\linewidth]{#8}
    \end{column}
  \end{columns}
}

\newcommand{\ttbar}{\ensuremath{t\bar{t}}~}
\newcommand{\Znn}{Z$\nu\nu$~}

\begin{document}

\begin{frame}[nonumbering]
  \titlepage
\end{frame}

\begin{frame}
  \frametitle{Outline}
  \tableofcontents
\end{frame}

\section{Sync Improvements}

\begin{frame}
  \frametitle{Improving Sync Between Analyses}

  Identified two more inconsistencies between our understanding of the
  \textcolor{orange}{public analysis}
  (paper and AN) and the \textcolor{green}{VHbb Heppy framework} (as seen on GitHub)

  \begin{enumerate}
  \item Track MET (used in a cut on $\Delta\phi$(trkMET, MET)): \\
    \textcolor{green}
              {Calculated only using the tracks with $\Delta z < 0.5$ from the primary vertex},
    while MINIAOD stores \textcolor{orange}{Track MET with all tracks}
  \item Relative Isolation: \\
    Code on GitHub takes \textcolor{green}{minimum between}
    \textcolor{orange}{relative isolation defined in \S4.4 of the AN}
    \textcolor{green}{and ``miniRelIso''}
  \end{enumerate}

\end{frame}

\begin{frame}
  \frametitle{Sync Status}

  \begin{itemize}
  \item With selection changes, our data yields for \Znn signal and
    control regions agree within 1.5\% with 2016 analysis setup
  \item Currently applying changes to MC samples
  \end{itemize}

\end{frame}

\section{Optimization of 2016 \Znn Analysis}

\begin{frame}
  \frametitle{Investigating Cuts}

  {\centering \bf
    \Large{\textcolor{blue}{Old cuts}}
    \begin{gather*}
      p_{T,1} > \SI{60}{GeV} \\
      p_{T,2} > \SI{35}{GeV}
    \end{gather*}
  }

  where $p_{T,1}$ refers to the $p_T$ of the jet with a higher cMVA score.
  The harder jet does not always have a higher cMVA score

  {\centering \bf
    \Large{\textcolor{blue}{New cuts}}
    \begin{gather*}
      \mathrm{max}(p_{T,1}, p_{T,2}) > \SI{60}{GeV} \\
      \mathrm{min}(p_{T,1}, p_{T,2}) > \SI{35}{GeV}
    \end{gather*}
  }

\end{frame}

\begin{frame}
  \frametitle{Yields and Statistical Significance}

  \centering

  \includegraphics[width=0.7\linewidth]{2dpt.png}

  \begin{center}
    \begin{tabular}{|l|r|r|r|r|}
      \hline
      Cut & Signal Yield & Data Yield & Purity & $S/\sqrt{D}$ \\
      \hline
      old cut & 57.11 & 4032 & 1.4\% & 0.90 \\
      new cut & 69.15 & 5326 & 1.3\% & 0.95 \\
      \hline
    \end{tabular}
  \end{center}

  Purity decreases, but with a 5\% sensitivity increase

\end{frame}

\begin{frame}
  \frametitle{Check on Jet $p_T$ (post-fit plots)}

  \twofigs{Signal}
          {180514_old/inclusive_signal_cmva_jet1_pt.pdf}
          {Z + Heavy Flavor}
          {180514_old/inclusive_heavyz_cmva_jet1_pt.pdf}

  \begin{itemize}
  \item Other CRs are in backup (they look good)
  \item $W + b$ process has a large scale factor ($\approx 5$) and uncertainty
  \item WH combination will constrain better
  \end{itemize}

\end{frame}

\begin{frame}
  \frametitle{Post-fit Plots (1/2)}

  \twofigs{Signal}
          {180515_class_post/inclusive_signal_maier_event_class.pdf}
          {\ttbar}
          {180514_v1/inclusive_tt_cmva_jet2_cmva.pdf}
           
  \vspace{12pt}

  \begin{itemize}
  \item Post-fit agreement looks excellent
  \item CR pre-fit plots are in the backup
  \end{itemize}

\end{frame}

\begin{frame}
  \frametitle{Post-fit Plots (2/2)}

  \twofigs{Z + Heavy Flavor}
          {180514_v1/inclusive_heavyz_cmva_jet2_cmva.pdf}
          {Z + Light Flavor}
          {180514_v1/inclusive_lightz_cmva_jet2_cmva.pdf}

  \vspace{12pt}

  \begin{itemize}
  \item Post-fit agreement looks excellent
  \item CR pre-fit plots are in the backup
  \end{itemize}

\end{frame}

\begin{frame}
  \frametitle{Result}

  \begin{itemize}
  \item Ran an expected limit over just the \Znn resolved category
  \item Fit backgrounds with data, and then toy signal for limit calculation
  \item Gives a 5\% improvement over the old cut from $1.35 \rightarrow 1.30$
  \end{itemize}

\end{frame}

\section{Difference in 2017 MET Triggers}

\begin{frame}
  \frametitle{New Triggers in 2017}

  2016 Analysis used:

  \vspace{12pt}

  {\setlength{\parindent}{0cm} \ttfamily \textcolor{blue}{
    HLT\_PFMETNoMu110\_PFMHTNoMu110\_IDTight \\
    HLT\_PFMETNoMu120\_PFMHTNoMu120\_IDTight \\
    HLT\_PFMET170\_NoiseCleaned \\
    HLT\_PFMET170\_HBHECleaned \\
    HLT\_PFMET170\_HBHE\_BeamHaloCleaned
    }
  }

  \vspace{12pt}

  The PFMET170 triggers are gone in 2017, and we first try:

  \vspace{12pt}

  {\setlength{\parindent}{0cm} \ttfamily \textcolor{blue}{
    HLT\_PFMETNoMu120\_PFMHTNoMu120\_IDTight\_PFHT60 \\
    HLT\_PFMETNoMu120\_PFMHTNoMu120\_IDTight \\
    HLT\_PFMETNoMu130\_PFMHTNoMu130\_IDTight \\
    HLT\_PFMETNoMu140\_PFMHTNoMu140\_IDTight
    }
  }

\end{frame}

\begin{frame}
  \frametitle{NPV re-weighting}

  Using an independent rough measurement for NPV re-weighting on an inclusive
  signal, Z+HF, and Z+LF selection

  \vspace{12pt}

  \twofigs{Before re-weighting}
          {180515_npv/npv.pdf}
          {After re-weighting}
          {180515_lumi/npv.pdf}

\end{frame}

\begin{frame}
  \frametitle{Comparison Plots}

  \vspace{12pt}

  \twofigs{PF MET}
          {180515_fin/pfmet.pdf}
          {Jet Multiplicity (normalized)}
          {180515_fin/n_centerjet.pdf}

  \vspace{12pt}

  \begin{itemize}
  \item MET turn-on is slower in the newer data, \\ leading to boosted topology
  \item Jet multiplicity looks good
  \item Other kinematics and tagging plots are in the backup
  \end{itemize}

\end{frame}

\begin{frame}
  \frametitle{Same trigger}

  \textcolor{red}{Here, I want to place a met plot with the one trigger path that exists in 2016 and 2017: HLT\_PFMETNoMu120\_PFMHTNoMu120\_IDTight}

\end{frame}

\section*{}

\begin{frame}
  \frametitle{Conclusion}
  \begin{itemize}
  \item We found more cuts on GitHub that do not appear in the documentation
  \item It is worth revisiting 2016 cuts for 2017 analysis
  \item MET triggers in 2017 data will lead to harder \Znn events
  \item Combination is under way for the various VH processes,
    which is prerequisite for accurate assessment of improvements brought by boosted analysis
  \end{itemize}

\end{frame}

\beginbackup

\begin{frame}
  \frametitle{Backup Slides}
\end{frame}

\begin{frame}
   \frametitle{\small 190611/plot\_time\_60000\_wide}
   \centering
   \includegraphics[width=0.6\linewidth]{190611/plot_time_60000_wide.pdf}
\end{frame}

\begin{frame}
   \frametitle{\small 190611/plot\_time\_wide}
   \centering
   \includegraphics[width=0.6\linewidth]{190611/plot_time_wide.pdf}
\end{frame}

\begin{frame}
   \frametitle{\small 190611/plot\_time\_120000\_compare}
   \centering
   \includegraphics[width=0.6\linewidth]{190611/plot_time_120000_compare.pdf}
\end{frame}

\begin{frame}
   \frametitle{\small 190611/plot\_time\_60000\_compare}
   \centering
   \includegraphics[width=0.6\linewidth]{190611/plot_time_60000_compare.pdf}
\end{frame}

\begin{frame}
   \frametitle{\small 190611/plot\_time\_80000\_compare}
   \centering
   \includegraphics[width=0.6\linewidth]{190611/plot_time_80000_compare.pdf}
\end{frame}

\begin{frame}
   \frametitle{\small 190611/plot\_time\_120000\_narrow}
   \centering
   \includegraphics[width=0.6\linewidth]{190611/plot_time_120000_narrow.pdf}
\end{frame}

\begin{frame}
   \frametitle{\small 190611/plot\_time\_40000\_narrow}
   \centering
   \includegraphics[width=0.6\linewidth]{190611/plot_time_40000_narrow.pdf}
\end{frame}

\begin{frame}
   \frametitle{\small 190611/plot\_time\_160000\_narrow}
   \centering
   \includegraphics[width=0.6\linewidth]{190611/plot_time_160000_narrow.pdf}
\end{frame}

\begin{frame}
   \frametitle{\small 190611/plot\_time\_60000\_narrow}
   \centering
   \includegraphics[width=0.6\linewidth]{190611/plot_time_60000_narrow.pdf}
\end{frame}

\begin{frame}
   \frametitle{\small 190611/plot\_time\_80000\_narrow}
   \centering
   \includegraphics[width=0.6\linewidth]{190611/plot_time_80000_narrow.pdf}
\end{frame}

\begin{frame}
   \frametitle{\small 190611/plot\_time\_narrow}
   \centering
   \includegraphics[width=0.6\linewidth]{190611/plot_time_narrow.pdf}
\end{frame}



\backupend

\end{document}
