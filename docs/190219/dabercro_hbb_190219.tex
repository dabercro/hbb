\documentclass{beamer}

\author[D. Abercrombie]{
  Daniel Abercrombie
}

\title{\bf \sffamily Planning for 2018 Data and MC}
\date{\today}

\usecolortheme{dove}

\usepackage[absolute,overlay]{textpos}
\usefonttheme{serif}
\usepackage{appendixnumberbeamer}
\usepackage{isotope}
\usepackage{hyperref}
\usepackage[english]{babel}
\usepackage{amsmath}
\setbeamerfont{frametitle}{size=\Large,series=\bf\sffamily}
\setbeamertemplate{frametitle}[default][center]
\usepackage{siunitx}
\usepackage{tabularx}
\usepackage{makecell}
\usepackage{comment}

\setbeamertemplate{navigation symbols}{}
\usepackage{graphicx}
\usepackage{color}
\setbeamertemplate{footline}[text line]{
  \parbox{1.083\linewidth}{\footnotesize \hfill \insertshortauthor \hfill \insertpagenumber /\inserttotalframenumber}
}
\setbeamertemplate{headline}[text line]{\parbox{1.083\linewidth}{\footnotesize \hspace{-0.083\linewidth} \textcolor{blue}{\sffamily \insertsection \hfill \insertsubsection}}}

\IfFileExists{/Users/dabercro/GradSchool/Presentations/MIT-logo.pdf}
             {\logo{\includegraphics[height=0.5cm]{/Users/dabercro/GradSchool/Presentations/MIT-logo.pdf}}}
             {\logo{\includegraphics[height=0.5cm]{/home/dabercro/MIT-logo.pdf}}}

\usepackage{changepage}

\newcommand{\beginbackup}{
  \newcounter{framenumbervorappendix}
  \setcounter{framenumbervorappendix}{\value{framenumber}}
}
\newcommand{\backupend}{
  \addtocounter{framenumbervorappendix}{-\value{framenumber}}
  \addtocounter{framenumber}{\value{framenumbervorappendix}}
}

\graphicspath{{figs/}}

\newcommand{\link}[2]{\href{#2}{\textcolor{blue}{\underline{#1}}}}
\newcommand{\clink}[2]{\link{#1}{http://t3serv001.mit.edu/~dabercro/redir/?k=#2}}}

\newcommand{\twofigs}[4]{
  \begin{columns}
    \begin{column}{0.5\linewidth}
      \centering
      \textcolor{blue}{#1} \\
      \includegraphics[width=\linewidth]{#2}
    \end{column}
    \begin{column}{0.5\linewidth}
      \centering
      \textcolor{blue}{#3} \\
      \includegraphics[width=\linewidth]{#4}
    \end{column}
  \end{columns}
}

\newcommand{\fourfigs}[8]{
  \begin{columns}
    \begin{column}{0.3\linewidth}
      \centering
      \textcolor{blue}{#1} \\
      \includegraphics[width=\linewidth]{#2} \\
      \textcolor{blue}{#3} \\
      \includegraphics[width=\linewidth]{#4}
    \end{column}
    \begin{column}{0.3\linewidth}
      \centering
      \textcolor{blue}{#5} \\
      \includegraphics[width=\linewidth]{#6} \\
      \textcolor{blue}{#7} \\
      \includegraphics[width=\linewidth]{#8}
    \end{column}
  \end{columns}
}

\newcommand{\ttbar}{\ensuremath{t\bar{t}}}
\newcommand{\bbbar}{\ensuremath{b\bar{b}}}

\begin{document}

\begin{frame}
  \titlepage
\end{frame}

\begin{frame}
  \frametitle{Introduction}

  \begin{itemize}
  \item We would like to perform a differential cross section measurement of
    $H\rightarrow \bbbar$
  \item First look at 2018 Data and the Autumn18 MC campaign
  \item {\bf Problems:} We started with the following from old data:
    \begin{itemize}
    \item b-tagging scale factors (this talk)
    \item trigger efficiencies (Benedikt last week)
    \item pileup reweighting (Guillelmo using 2017 MC?)
    \end{itemize}
  \end{itemize}

\end{frame}

\begin{frame}
  \frametitle{NPV Distributions}

  \twofigs{NPV from 2017 MC}
          {190219_v1/lightz_npv.pdf}
          {Not Correcting NPV}
          {190219_nopu/lightz_npv.pdf}

  \begin{itemize}
  \item With and without are not quite right
  \item I used the scale factors throughout
  \item I'll be skipping MET trigger for now (in backup)
  \end{itemize}
\end{frame}

\begin{frame}
  \frametitle{Old b-tag scale factors}

  \twofigs{Z + Heavy Flavor}
          {190131_2018mc/heavyz_jet2_deepCSVb.pdf}
          {\ttbar}
          {190131_2018mc/tt_jet2_deepCSVb.pdf}

  The old b-tag scale factors don't look so great

\end{frame}

\begin{frame}
  \frametitle{b-tag Strategy}

  \begin{itemize}
  \item Select di-lepton events with exactly two jets
  \item At least one tight lepton and exactly two loose leptons
  \item SingleMuon dataset, no triggers
  \item Separate into $\ttbar$ and $Z$ + jets using
    \begin{itemize}
    \item MET cut, di-lepton (opposite sign, same flavor) mass veto, and
      probe jet with deep CSV greater than 0.6
    \item di-muon mass between 60 and 120 GeV, and
      probe jet with deep CSV less than 0.4
    \end{itemize}
  \item Alternated, based on event number, between higher or lower Deep CSV jet as tag jet
    \begin{itemize}
    \item Not sure of the best way, but left over from $Hbb$ n-tuple format
    \end{itemize}
  \item Still need to make sure we are orthogonal to $ZH \rightarrow \ell\ell bb$
  \end{itemize}

\end{frame}

\begin{frame}
  \frametitle{Binning}

  Distribution doesn't change much over $\eta$, but does over $p_T$

  \fourfigs{\mbox{$p_T < 150$, $|\eta| < 1.4$}}
           {190214_tnp_bjets/light_pt_0_150_eta_0_1p4_probe_deepCSVb.pdf}
           {\mbox{$p_T < 150$, $1.4 < \eta < 2.4$}}
           {190214_tnp_bjets/light_pt_0_150_eta_1p4_2p5_probe_deepCSVb.pdf}
           {\mbox{$150 < p_T < 250$, $|\eta| < 1.4$}}
           {190214_tnp_bjets/light_pt_150_250_eta_0_1p4_probe_deepCSVb.pdf}
           {\mbox{$150 < p_T < 250$, $1.4 < \eta < 2.4$}}
           {190214_tnp_bjets/light_pt_150_250_eta_1p4_2p5_probe_deepCSVb.pdf}
           
\end{frame}

\begin{frame}
  \frametitle{Method of Reweighting}

  \begin{itemize}
  \item Using $p_T$ bins $[0, 150), [150, 250), [250, \infty)$
  \item Tried both with and without background subtraction
    \begin{itemize}
    \item Without looks much better due to significant contamination
      at low Deep CSV values in $\ttbar$ (see backup slides)
    \end{itemize}
  \item Normalized distributions beforehand,
    hopefully accounting for lack of scaling of the probe jet
  \item Simple reweight (ratio of data/MC)
  \end{itemize}

\end{frame}

\begin{frame}
  \frametitle{Closure Test}

  Scale factor is applied to every single jet in the event

  \twofigs{\ttbar}
          {190215_closure/tt_jet2_deepCSVb.pdf}
          {$Z$ + jets}
          {190215_closure/z_jet2_deepCSVb.pdf}

  Closure looks horrible, especially at low values. \\ But wait there's more!

\end{frame}

\begin{frame}
  \frametitle{$Z\nu\nu$ Analysis Selection}

  Definite improvement

  \fourfigs{Old \ttbar}
           {190219_olddeep/tt_jet2_deepCSVb.pdf}
           {Old $Z$ + jets}
           {190219_olddeep/heavyz_jet2_deepCSVb.pdf}
           {New \ttbar}
           {190219_newdeep/tt_jet2_deepCSVb.pdf}
           {New $Z$ + jets}
           {190219_newdeep/heavyz_jet2_deepCSVb.pdf}

\end{frame}

\begin{frame}
  \frametitle{Conclusions}

  \begin{itemize}
  \item Need to make a selection that is orthogonal to $Z\ell\ell$ analysis
  \item We have a b-tag scale factor that looks better, but not perfect
  \item Don't understand closure plot
  \item Have framework for measuring and plotting with scale factors within minutes
    \begin{itemize}
    \item Can be used in iterative background subtraction
    \end{itemize}
  \end{itemize}

\end{frame}

\beginbackup

\begin{frame}
  \frametitle{Backup Slides}
\end{frame}

\begin{frame}
   \frametitle{\small 190611/plot\_time\_60000\_wide}
   \centering
   \includegraphics[width=0.6\linewidth]{190611/plot_time_60000_wide.pdf}
\end{frame}

\begin{frame}
   \frametitle{\small 190611/plot\_time\_wide}
   \centering
   \includegraphics[width=0.6\linewidth]{190611/plot_time_wide.pdf}
\end{frame}

\begin{frame}
   \frametitle{\small 190611/plot\_time\_120000\_compare}
   \centering
   \includegraphics[width=0.6\linewidth]{190611/plot_time_120000_compare.pdf}
\end{frame}

\begin{frame}
   \frametitle{\small 190611/plot\_time\_60000\_compare}
   \centering
   \includegraphics[width=0.6\linewidth]{190611/plot_time_60000_compare.pdf}
\end{frame}

\begin{frame}
   \frametitle{\small 190611/plot\_time\_80000\_compare}
   \centering
   \includegraphics[width=0.6\linewidth]{190611/plot_time_80000_compare.pdf}
\end{frame}

\begin{frame}
   \frametitle{\small 190611/plot\_time\_120000\_narrow}
   \centering
   \includegraphics[width=0.6\linewidth]{190611/plot_time_120000_narrow.pdf}
\end{frame}

\begin{frame}
   \frametitle{\small 190611/plot\_time\_40000\_narrow}
   \centering
   \includegraphics[width=0.6\linewidth]{190611/plot_time_40000_narrow.pdf}
\end{frame}

\begin{frame}
   \frametitle{\small 190611/plot\_time\_160000\_narrow}
   \centering
   \includegraphics[width=0.6\linewidth]{190611/plot_time_160000_narrow.pdf}
\end{frame}

\begin{frame}
   \frametitle{\small 190611/plot\_time\_60000\_narrow}
   \centering
   \includegraphics[width=0.6\linewidth]{190611/plot_time_60000_narrow.pdf}
\end{frame}

\begin{frame}
   \frametitle{\small 190611/plot\_time\_80000\_narrow}
   \centering
   \includegraphics[width=0.6\linewidth]{190611/plot_time_80000_narrow.pdf}
\end{frame}

\begin{frame}
   \frametitle{\small 190611/plot\_time\_narrow}
   \centering
   \includegraphics[width=0.6\linewidth]{190611/plot_time_narrow.pdf}
\end{frame}



\backupend

\end{document}
