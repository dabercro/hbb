\documentclass{beamer}

\author[D. Abercrombie]{
  Daniel Abercrombie
}

\title{\bf \sffamily Some Plots of $ZH \rightarrow \nu\bar{\nu} b\bar{b}$ Control Regions with First Pass of b-Jet $p_T$ Regression}
\date{\today}

\usecolortheme{dove}

\usepackage[absolute,overlay]{textpos}
\usefonttheme{serif}
\usepackage{appendixnumberbeamer}
\usepackage{isotope}
\usepackage{hyperref}
\usepackage[english]{babel}
\usepackage{amsmath}
\setbeamerfont{frametitle}{size=\Large,series=\bf\sffamily}
\setbeamertemplate{frametitle}[default][center]
\usepackage{siunitx}
\usepackage{tabularx}
\usepackage{feynmp-auto}
\usepackage{makecell}

\setbeamertemplate{navigation symbols}{}
\usepackage{graphicx}
\usepackage{color}
\setbeamertemplate{footline}[text line]{\parbox{1.083\linewidth}{\footnotesize \hfill \insertshortauthor \hfill \insertpagenumber /\inserttotalframenumber}}
\setbeamertemplate{headline}[text line]{\parbox{1.083\linewidth}{\footnotesize \hspace{-0.083\linewidth} \textcolor{blue}{\sffamily \insertsection \hfill \insertsubsection}}}

\IfFileExists{/Users/dabercro/GradSchool/Presentations/MIT-logo.pdf}
             {\logo{\includegraphics[height=0.5cm]{/Users/dabercro/GradSchool/Presentations/MIT-logo.pdf}}}
             {\logo{\includegraphics[height=0.5cm]{/home/dabercro/MIT-logo.pdf}}}

\usepackage{changepage}

\newcommand{\beginbackup}{
  \newcounter{framenumbervorappendix}
  \setcounter{framenumbervorappendix}{\value{framenumber}}
}
\newcommand{\backupend}{
  \addtocounter{framenumbervorappendix}{-\value{framenumber}}
  \addtocounter{framenumber}{\value{framenumbervorappendix}}
}

\graphicspath{{figs/}}

\newcommand{\link}[2]{\href{#2}{\textcolor{blue}{\underline{#1}}}}
\newcommand{\clink}[2]{\link{#1}{http://t3serv001.mit.edu/~dabercro/redir/?k=#2}}}

\newcommand{\twofigs}[4]{
  \begin{columns}
    \begin{column}{0.5\linewidth}
      \centering
      \textcolor{blue}{#1} \\
      \includegraphics[width=\linewidth]{#2}
    \end{column}
    \begin{column}{0.5\linewidth}
      \centering
      \textcolor{blue}{#3} \\
      \includegraphics[width=\linewidth]{#4}
    \end{column}
  \end{columns}
}

\newcommand{\fourfigs}[8]{
  \begin{columns}
    \begin{column}{0.5\linewidth}
      \centering
      \textcolor{blue}{#1} \\
      \includegraphics[width=0.6\linewidth]{#2} \\
      \textcolor{blue}{#3} \\
      \includegraphics[width=0.6\linewidth]{#4}
    \end{column}
    \begin{column}{0.5\linewidth}
      \centering
      \textcolor{blue}{#5} \\
      \includegraphics[width=0.6\linewidth]{#6} \\
      \textcolor{blue}{#7} \\
      \includegraphics[width=0.6\linewidth]{#8}
    \end{column}
  \end{columns}
}

\newcommand{\ttbar}{\ensuremath{t\bar{t}} \hspace{2pt}}
\newcommand{\MET}{\ensuremath{E_{T}^{\mathrm{miss}}}}

\begin{document}

\begin{frame}[nonumbering]
  \titlepage
\end{frame}

\begin{frame}
  \frametitle{Introduction}

  \centering
  \begin{fmffile}{ZHvvbb}
    \begin{fmfgraph*}(120,80)
      \fmfleft{i1,i0}
      \fmfright{o1,o0,o3,o2}
      \fmf{fermion}{i0,v0,i1}
      \fmf{photon,label=$Z$,label.side=right}{v0,v1}
      \fmf{dashes,label=$H$,lable.side=left}{v1,v2}
      \fmf{fermion}{o3,v2,o2}
      \fmf{photon}{v1,v3}
      \fmf{fermion}{o1,v3,o0}
      \fmflabel{$\bar{b}$}{o3}
      \fmflabel{$b$}{o2}
      \fmflabel{$\bar{\nu}$}{o1}
      \fmflabel{$\nu$}{o0}
    \end{fmfgraph*}
  \end{fmffile}

\end{frame}

\begin{frame}
  \frametitle{Control Regions}
  \begin{itemize}
  \item $Z + \mathrm{light}$ jets
  \item $Z + b$ jets
  \item Multijet (QCD)
  \item \ttbar
  \end{itemize}
\end{frame}

\begin{frame}
  \frametitle{Physics Definitions}
\end{frame}

\begin{frame}
  \frametitle{Samples}
  \begin{itemize}
  \item $\{W,Z\} + \mathrm{jets}$: Madgraph, \textcolor{red}{HT-binned samples}
  \item QCD: Madgraph, \textcolor{red}{HT-binned samples}
  \item \ttbar: amc@NLO for single lep, POWHEG for double lep
  \item Single Top: inclusive decays (POWHEG)
  \item Tiny bit diboson: (Pythia)
  \item Various others that don't show up
  \end{itemize}

\end{frame}

\begin{frame}
  \frametitle{Scale Factors}

  \begin{itemize}
  \item Pileup reweighting (by true PU)
  \item NLO k-factors applied to HT-binned samples
  \item Additional higher order corrections from EWK vertices
  \item MET trigger efficiencies
  \item $p_T$ reweighting for \ttbar samples
  \end{itemize}

\end{frame}

\begin{frame}
  \frametitle{Pre-Selection}
  All events have the following cuts:

  \begin{itemize}
  \item $\MET > \SI{170}{GeV}$
  \item $p_{T, jj} > \SI{120}{GeV}$
  \item $p_{T, j1} > \SI{60}{GeV}$ and $p_{T, j2} > \SI{35}{GeV}$
  \item Both jets to reconstruct Higgs pass loose working point
    ($\mathrm{CSV} > 0.5426$ to be changed to CMVA working point)
  \end{itemize}
\end{frame}

\begin{frame}
  \frametitle{Pre-Selection Plots (Higgs Kinematics)}
  \twofigs{Higgs mass}
          {171201/ZvvHbb_common_hbbm.pdf}
          {Higgs $p_T$}
          {171201/ZvvHbb_common_hbbpt.pdf}

  We will be operating under the assumption that this plot includes some Higgs
\end{frame}

\begin{frame}
  \frametitle{Pre-Selection Plots (Jets and \MET)}
  \twofigs{$\Delta\phi(\MET, \mathrm{Higgs})$}
          {171201/ZvvHbb_common_deltaPhi__pfmetphi__hbbphi__.pdf}
          {$\Delta\phi_{\mathrm{min}}(\MET, \mathrm{jet})$}
          {171201/ZvvHbb_common_dphipfmet.pdf}

  QCD jets generate \MET and top events are just messy
\end{frame}

\begin{frame}
  \frametitle{Farther Selections}

  \centering
  \begin{tabular}{| l | c | c | c | c |}
    \hline
    Variable & \ttbar & Multijet & $Z + \mathrm{light}$ & $Z + b$ \\
    \hline
    \Delta\phi(V, H) & $> 2.0$ & $< 2.0$ & $> 2.0$ & $> 2.0$ \\
    $N_{\mathrm{lep}}$ & $\ge 1$ & 0 & 0 & 0 \\
    $N_{\mathrm{jets}}^{\mathrm{central}}$ & $\ge 4$ & -- & $\le 3$ & $\le 2$ \\
    Max $b$ tag & Medium & Loose & Not Med. & Tight \\
    $\Delta\phi_{\mathrm{min}}(\MET, j)$ & $< \pi/2$ & $< 0.4$ & $> 0.5$ & $>0.5$ \\
    \hline
  \end{tabular}

  \begin{itemize}
  \item $Z + b$ also has mass veto around the Higgs
  \item Track \MET $\phi$ cuts missing
  \item Adapted from Table 16 of AN, which does stupid things like split rows.
    Maybe I should find out why... \\
    In backup slides.
  \end{itemize}
\end{frame}

\begin{frame}
  \frametitle{Scaling \ttbar}
  AN has a scale factor of 0.78 for \ttbar selection in $Z(\nu\nu)$

  \twofigs{Before}
          {171212/ZvvHbb_tt_hbbm.pdf}
          {After}
          {171212/ZvvHbb_scaledtt_hbbm.pdf}

\end{frame}

\begin{frame}
  \frametitle{Plot Dump}
  \centering
  I have nothing interesting left to say, \\
  so now I will just show plots
\end{frame}

\begin{frame}
  \frametitle{Plot Dump: Higgs Mass}
  \fourfigs{Multijet}
           {171212/ZvvHbb_multijet_hbbm.pdf}
           {\ttbar}
           {171212/ZvvHbb_scaledtt_hbbm.pdf}
           {$Z + \mathrm{light}$}
           {171212/ZvvHbb_lightz_hbbm.pdf}
           {$Z + b$}
           {171212/ZvvHbb_heavyz_hbbm.pdf}
\end{frame}

\begin{frame}
  \frametitle{Plot Dump: Higgs $p_T$}
  \fourfigs{Multijet}
           {171212/ZvvHbb_multijet_hbbpt.pdf}
           {\ttbar}
           {171212/ZvvHbb_scaledtt_hbbpt.pdf}
           {$Z + \mathrm{light}$}
           {171212/ZvvHbb_lightz_hbbpt.pdf}
           {$Z + b$}
           {171212/ZvvHbb_heavyz_hbbpt.pdf}
\end{frame}

\begin{frame}
  \frametitle{Plot Dump: Higgs $p_T$}
  \fourfigs{Multijet}
           {171212/ZvvHbb_multijet_hbbpt.pdf}
           {\ttbar}
           {171212/ZvvHbb_scaledtt_hbbpt.pdf}
           {$Z + \mathrm{light}$}
           {171212/ZvvHbb_lightz_hbbpt.pdf}
           {$Z + b$}
           {171212/ZvvHbb_heavyz_hbbpt.pdf}
\end{frame}

\begin{frame}
  \frametitle{Plot Dump: \MET}
  \fourfigs{Multijet}
           {171212/ZvvHbb_multijet_pfmet.pdf}
           {\ttbar}
           {171212/ZvvHbb_scaledtt_pfmet.pdf}
           {$Z + \mathrm{light}$}
           {171212/ZvvHbb_lightz_pfmet.pdf}
           {$Z + b$}
           {171212/ZvvHbb_heavyz_pfmet.pdf}
\end{frame}

\begin{frame}
  \frametitle{Plot Dump: Leading Jet $p_T$}
  \fourfigs{Multijet}
           {171212/ZvvHbb_multijet_jet1Pt.pdf}
           {\ttbar}
           {171212/ZvvHbb_scaledtt_jet1Pt.pdf}
           {$Z + \mathrm{light}$}
           {171212/ZvvHbb_lightz_jet1Pt.pdf}
           {$Z + b$}
           {171212/ZvvHbb_heavyz_jet1Pt.pdf}
\end{frame}

\begin{frame}
  \frametitle{Plot Dump: Sub-Leading Jet $p_T$}
  \fourfigs{Multijet}
           {171212/ZvvHbb_multijet_jet2Pt.pdf}
           {\ttbar}
           {171212/ZvvHbb_scaledtt_jet2Pt.pdf}
           {$Z + \mathrm{light}$}
           {171212/ZvvHbb_lightz_jet2Pt.pdf}
           {$Z + b$}
           {171212/ZvvHbb_heavyz_jet2Pt.pdf}
\end{frame}

\begin{frame}
  \frametitle{Plot Dump: Number (Central?) Jets}
  \fourfigs{Multijet}
           {171212/ZvvHbb_multijet_nJet.pdf}
           {\ttbar}
           {171212/ZvvHbb_scaledtt_nJet.pdf}
           {$Z + \mathrm{light}$}
           {171212/ZvvHbb_lightz_nJet.pdf}
           {$Z + b$}
           {171212/ZvvHbb_heavyz_nJet.pdf}
\end{frame}

\begin{frame}
  \frametitle{In Summary}
  \begin{itemize}
  \item I made some plots
  \item $Z + b$ jets seems to be the worst control reason
    \begin{itemize}
    \item Slightly alarming because it's basically the signal region
    \item Going to use different taggers and scale factors anyway
    \end{itemize}
  \item Should start separating backgrounds based on flavor
  \end{itemize}
\end{frame}

\beginbackup

\begin{frame}
  \frametitle{Backup Slides}
\end{frame}

\begin{frame}
  \frametitle{AN Table}
  \includegraphics[width=\linewidth]{table.png}
\end{frame}

%\begin{frame}
   \frametitle{\small 190813\_testreg\_013/Jet\_puppi\_charged\_ptfrac}
   \centering
   \includegraphics[width=0.6\linewidth]{190813_testreg_013/Jet_puppi_charged_ptfrac.pdf}
\end{frame}

\begin{frame}
   \frametitle{\small 190813\_testreg\_013/Jet\_puppi\_neutral\_ptfrac}
   \centering
   \includegraphics[width=0.6\linewidth]{190813_testreg_013/Jet_puppi_neutral_ptfrac.pdf}
\end{frame}

\begin{frame}
   \frametitle{\small 190813\_testreg\_013/Jet\_puppi\_charged\_pu\_ptfrac}
   \centering
   \includegraphics[width=0.6\linewidth]{190813_testreg_013/Jet_puppi_charged_pu_ptfrac.pdf}
\end{frame}

\begin{frame}
   \frametitle{\small 190813\_testreg\_013/Jet\_puppi\_neutral\_pu\_ptfrac}
   \centering
   \includegraphics[width=0.6\linewidth]{190813_testreg_013/Jet_puppi_neutral_pu_ptfrac.pdf}
\end{frame}

\begin{frame}
   \frametitle{\small 190813\_bukin/signal\_hbb\_m\_190725\_lstm\_pf}
   \centering
   \includegraphics[width=0.6\linewidth]{190813_bukin/signal_hbb_m_190725_lstm_pf.pdf}
\end{frame}

\begin{frame}
   \frametitle{\small 190813\_bukin/signal\_hbb\_m}
   \centering
   \includegraphics[width=0.6\linewidth]{190813_bukin/signal_hbb_m.pdf}
\end{frame}

\begin{frame}
   \frametitle{\small 190813\_bukin/signal\_hbb\_m\_190723\_origin}
   \centering
   \includegraphics[width=0.6\linewidth]{190813_bukin/signal_hbb_m_190723_origin.pdf}
\end{frame}

\begin{frame}
   \frametitle{\small 190813\_bukin/signal\_hbb\_m\_190724\_puppi\_direction}
   \centering
   \includegraphics[width=0.6\linewidth]{190813_bukin/signal_hbb_m_190724_puppi_direction.pdf}
\end{frame}

\begin{frame}
   \frametitle{\small 190813\_ratio/signal\_hbb\_m\_190724\_puppi\_direction}
   \centering
   \includegraphics[width=0.6\linewidth]{190813_ratio/signal_hbb_m_190724_puppi_direction.pdf}
\end{frame}

\begin{frame}
   \frametitle{\small 190813\_bukin/signal\_hbb\_m\_190724\_origin\_direction}
   \centering
   \includegraphics[width=0.6\linewidth]{190813_bukin/signal_hbb_m_190724_origin_direction.pdf}
\end{frame}

\begin{frame}
   \frametitle{\small 190813\_ratio/signal\_hbb\_m\_190724\_origin\_direction}
   \centering
   \includegraphics[width=0.6\linewidth]{190813_ratio/signal_hbb_m_190724_origin_direction.pdf}
\end{frame}

\begin{frame}
   \frametitle{\small 190813\_testreg\_013/Jet\_pf\_0\_transformed\_px}
   \centering
   \includegraphics[width=0.6\linewidth]{190813_testreg_013/Jet_pf_0_transformed_px.pdf}
\end{frame}

\begin{frame}
   \frametitle{\small 190813\_testreg\_013/Jet\_pf\_1\_transformed\_px}
   \centering
   \includegraphics[width=0.6\linewidth]{190813_testreg_013/Jet_pf_1_transformed_px.pdf}
\end{frame}

\begin{frame}
   \frametitle{\small 190813\_testreg\_013/Jet\_pf\_2\_transformed\_px}
   \centering
   \includegraphics[width=0.6\linewidth]{190813_testreg_013/Jet_pf_2_transformed_px.pdf}
\end{frame}

\begin{frame}
   \frametitle{\small 190813\_testreg\_013/Jet\_pf\_3\_transformed\_px}
   \centering
   \includegraphics[width=0.6\linewidth]{190813_testreg_013/Jet_pf_3_transformed_px.pdf}
\end{frame}

\begin{frame}
   \frametitle{\small 190813\_testreg\_013/Jet\_pf\_0\_transformed\_py}
   \centering
   \includegraphics[width=0.6\linewidth]{190813_testreg_013/Jet_pf_0_transformed_py.pdf}
\end{frame}

\begin{frame}
   \frametitle{\small 190813\_testreg\_013/Jet\_pf\_1\_transformed\_py}
   \centering
   \includegraphics[width=0.6\linewidth]{190813_testreg_013/Jet_pf_1_transformed_py.pdf}
\end{frame}

\begin{frame}
   \frametitle{\small 190813\_testreg\_013/Jet\_pf\_2\_transformed\_py}
   \centering
   \includegraphics[width=0.6\linewidth]{190813_testreg_013/Jet_pf_2_transformed_py.pdf}
\end{frame}

\begin{frame}
   \frametitle{\small 190813\_testreg\_013/Jet\_pf\_3\_transformed\_py}
   \centering
   \includegraphics[width=0.6\linewidth]{190813_testreg_013/Jet_pf_3_transformed_py.pdf}
\end{frame}

\begin{frame}
   \frametitle{\small 190813\_testreg\_013/Jet\_pf\_0\_transformed\_pz}
   \centering
   \includegraphics[width=0.6\linewidth]{190813_testreg_013/Jet_pf_0_transformed_pz.pdf}
\end{frame}

\begin{frame}
   \frametitle{\small 190813\_testreg\_013/Jet\_pf\_1\_transformed\_pz}
   \centering
   \includegraphics[width=0.6\linewidth]{190813_testreg_013/Jet_pf_1_transformed_pz.pdf}
\end{frame}

\begin{frame}
   \frametitle{\small 190813\_testreg\_013/Jet\_pf\_2\_transformed\_pz}
   \centering
   \includegraphics[width=0.6\linewidth]{190813_testreg_013/Jet_pf_2_transformed_pz.pdf}
\end{frame}

\begin{frame}
   \frametitle{\small 190813\_testreg\_013/Jet\_pf\_3\_transformed\_pz}
   \centering
   \includegraphics[width=0.6\linewidth]{190813_testreg_013/Jet_pf_3_transformed_pz.pdf}
\end{frame}



\backupend

\end{document}
