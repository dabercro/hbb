\documentclass{beamer}

\author[D. Abercrombie]{
  \emph{Daniel Abercrombie}, Brandon Allen, Zeynep Demiragli,
  Guillelmo Gomez-Ceballos, Dylan George Hsu, \\
  Philip Harris, Yutaro Iiyama, Benedikt Maier, \\
  Siddharth Narayanan, Christoph Paus
}

\title{\bf \sffamily VHbb Introduction \\ with focus on Z($\nu\nu$)Hbb}
\date{May 29, 2018}

\usecolortheme{dove}

\usepackage[absolute,overlay]{textpos}
\usefonttheme{serif}
\usepackage{appendixnumberbeamer}
\usepackage{isotope}
\usepackage{hyperref}
\usepackage[english]{babel}
\usepackage{amsmath}
\setbeamerfont{frametitle}{size=\Large,series=\bf\sffamily}
\setbeamertemplate{frametitle}[default][center]
\usepackage{siunitx}
\usepackage{tabularx}
\usepackage{makecell}

\setbeamertemplate{navigation symbols}{}
\usepackage{graphicx}
\usepackage{color}
\setbeamertemplate{footline}[text line]{\parbox{1.083\linewidth}{\footnotesize \hfill \insertshortauthor \hfill \insertpagenumber /\inserttotalframenumber}}
\setbeamertemplate{headline}[text line]{\parbox{1.083\linewidth}{\footnotesize \hspace{-0.083\linewidth} \textcolor{blue}{\sffamily \insertsection \hfill \insertsubsection}}}

\IfFileExists{/Users/dabercro/GradSchool/Presentations/MIT-logo.pdf}
             {\logo{\includegraphics[height=0.5cm]{/Users/dabercro/GradSchool/Presentations/MIT-logo.pdf}}}
             {\logo{\includegraphics[height=0.5cm]{/home/dabercro/MIT-logo.pdf}}}

\usepackage{changepage}

\newcommand{\beginbackup}{
  \newcounter{framenumbervorappendix}
  \setcounter{framenumbervorappendix}{\value{framenumber}}
}
\newcommand{\backupend}{
  \addtocounter{framenumbervorappendix}{-\value{framenumber}}
  \addtocounter{framenumber}{\value{framenumbervorappendix}}
}

\graphicspath{{figs/}}

\newcommand{\link}[2]{\href{#2}{\textcolor{blue}{\underline{#1}}}}
\newcommand{\clink}[2]{\link{#1}{http://t3serv001.mit.edu/~dabercro/redir/?k=#2}}}

\newcommand{\twofigs}[4]{
  \begin{columns}
    \begin{column}{0.5\linewidth}
      \centering
      \textcolor{blue}{#1} \\
      \includegraphics[width=\linewidth]{#2}
    \end{column}
    \begin{column}{0.5\linewidth}
      \centering
      \textcolor{blue}{#3} \\
      \includegraphics[width=\linewidth]{#4}
    \end{column}
  \end{columns}
}

\newcommand{\fourfigs}[8]{
  \begin{columns}
    \begin{column}{0.3\linewidth}
      \centering
      \textcolor{blue}{#1} \\
      \includegraphics[width=\linewidth]{#2} \\
      \textcolor{blue}{#3} \\
      \includegraphics[width=\linewidth]{#4}
    \end{column}
    \begin{column}{0.3\linewidth}
      \centering
      \textcolor{blue}{#5} \\
      \includegraphics[width=\linewidth]{#6} \\
      \textcolor{blue}{#7} \\
      \includegraphics[width=\linewidth]{#8}
    \end{column}
  \end{columns}
}

\newcommand{\ttbar}{\ensuremath{t\bar{t}}}

\begin{document}

\begin{frame}[nonumbering]
  \titlepage
\end{frame}

\begin{frame}
  \frametitle{Introduction}
  \begin{itemize}
  \item VHbb is one analysis that is being pursued to observe the $H \rightarrow bb$ process
    \begin{itemize}
    \item Other analyses: gluon-fusion and vector boson fusion
    \end{itemize}
  \item The VH analysis is split into three categories
    \begin{itemize}
    \item \textcolor{blue}{$Z\nu\nu$}
    \item $Z\ell\ell$
    \item $W\ell\nu$
    \end{itemize}
  \item The approach for the three VH channels are similar
  \end{itemize}
\end{frame}

\begin{frame}
  \frametitle{Signal Selection for Z($\nu\nu$)}
  Signal should look like two b-tagged jets and large MET
  \begin{itemize}
  \item Jets with $p_T > \SI{20}{GeV}$ sorted by cMVA (replacing CSV)
    \begin{itemize}
    \item Require one tight cMVA jet and other loose
    \item Final cut requires one $\SI{60}{GeV}$ jet and other $\SI{35}{GeV}$
    \item Higgs reconstructed using top two jets
    \end{itemize}
  \item MET greater than \SI{170}{GeV}
  \item Di-jet $p_T > \SI{120}{GeV}$ and $\Delta\phi_{\mathrm{MET}} > 2$
  \item No jets with $p_T > \SI{30}{GeV}$ within $\Delta\phi = 0.5$ of MET (QCD)
  \item No veto electrons
  \item At most one additional jet in the central detector
  \end{itemize}
\end{frame}

\begin{frame}
  \frametitle{Plot to Motivate Control Regions}

  Regions that need control are:
  $t\bar{t}$;
  Z + HF;
  W + HF

  \vspace{12pt}

  \centering
  \includegraphics[width=0.6\linewidth]{180529_v1/inclusive_signal_pfmet.pdf}
\end{frame}

\begin{frame}
  \frametitle{Control Region Selection}
  \begin{center}
    {\scriptsize
      \begin{tabular}{r|cccc}
        \hline \hline
        Variable                        & Signal     & Z + HF     & Z + LF     & ${t\bar{t}}$ \\
        \hline
        max($p_T$) [GeV] & $>60$      & $>60$      & $>60$      & $>60$ \\
        min($p_T$) [GeV] & $>35$      & $>35$      & $>35$      & $>35$ \\
        cMVA$_{\mathrm{min}}$           & $>-0.5884$ & $>-0.5884$ & $>-0.5884$ & $>-0.5884$ \\
        $p_T(jj)$ [GeV]                 & $>120$     & $>120$     & $>120$     & $>120$ \\
        MET [GeV]                      & $>170$     & $>170$     & $>170$     & $>170$ \\
        $\Delta\phi(U, jj)$             & $>2.0$     & $>2.0$     & $>2.0$     & $>2.0$ \\
        \hline
        cMVA$_{\mathrm{max}}$           & $>0.9432$  & $>0.9432$  & $<0.4432$  & $>0.4432$ \\
        $m(jj)$ [GeV]                   & $60<m<160$ & Sig. veto and $<500$ & $<500$ & $<500$ \\
        $n_{j}$                         & $\le 3$    & 2          & $\le 3$    & $\ge 4$ \\
        $n_{\ell}$                      & 0          & 0          & 0          & 1 tight \\
        $\Delta\phi(\MET, trk\MET)$     & $<0.5$     & $<0.5$     & $<0.5$     & -- \\
        $\Delta\phi_{\mathrm{min}}$(MET, $j$) & $>0.5$   & $>0.5$     & $>0.5$     & -- \\
        $\Delta\phi_{\mathrm{min}}(U, b)$         & --         & --         & --         & $<\frac{\pi}{2}$ \\
        \hline\hline
      \end{tabular}
    }
  \end{center}
\end{frame}

\begin{frame}
  \frametitle{Distributions for Fitting}
  \begin{itemize}
  \item Second cMVA score used
    \begin{itemize}
    \item Separates Z + bb and Z + b, for example
    \end{itemize}
  \item Note: in 2017, deepCSV will be used over cMVA
  \item Signal region is fit to a signal classifier BDT
  \end{itemize}
\end{frame}

\begin{frame}
  \frametitle{Signal Event Classifier}
  We train a BDT for signal identification in the signal region, minus the $N_{jet}$ cut,
  using the following variables
  \begin{columns}
    \begin{column}{0.5\linewidth}
      \begin{itemize}
      \item MET
      \item regressed di-jet mass
      \item regressed di-jet pt
      \item regressed di-jet pt over MET
      \item cMVA of top three cMVA jets
      \item $p_T$ of top three cMVA jets
      \end{itemize}
    \end{column}
    \begin{column}{0.5\linewidth}
      \begin{itemize}
      \item min($\Delta \phi$(MET, jet))
      \item di-jet $\Delta R$
      \item di-jet $\Delta \eta$
      \item di-jet $\Delta \phi$
      \item $\Delta \phi$(MET, di-jet)
      \item \textcolor{red}{Soft activity ($> \SI{5}{GeV}$)}
      \item Number of jets
      \end{itemize}
    \end{column}
  \end{columns}
\end{frame}

\begin{frame}
  \frametitle{Soft Activity}
  \begin{itemize}
  \item Form an ellipse
    \begin{itemize}
    \item The two b-jets as the foci
    \item Major axis length is $\Delta R(b,b) + 2 \times 0.4$
    \end{itemize}
  \item Gather PF candidates that fall outside this ellipse and \\
    are not matched to identified leptons
  \item Cluster these PF candidates with anti-k$_T$, $R=0.4$
  \item Count number of jets with $p_T > 5 \GeV$
  \end{itemize}
  \begin{center}
    \textcolor{red}{No uncertainties applied that I'm aware of}
  \end{center}
\end{frame}

\begin{frame}
  \frametitle{B-jet Regression}
  Need to recover energy lost in $b$ decay. We train in a fully-leptonic tt sample
  with 2 tight leptons.

  \hspace{12pt}

  Train on the jet with the highest CMVA value, using the following

  \begin{columns}
    \begin{column}{0.33\linewidth}
      \begin{itemize}
      \item NPV
      \item \# central jets
      \item \# total jets
      \item $p_T$
      \item $\eta$
      \item $\phi$
      \item mass
      \end{itemize}
    \end{column}
    \begin{column}{0.33\linewidth}
      \begin{itemize}
      \item CSV
      \item CMVA
      \item QGL
      \item SV \# tracks
      \item SV $p_T$
      \item SV mass
      \item SV distance
      \item SV uncertainty
      \end{itemize}
    \end{column}
    \begin{column}{0.33\linewidth}
      \begin{itemize}
      \item Max track $p_T$
      \item \# leptons
      \item lead lepton $p_T$
      \item LL $p_{T, \perp}$
      \item LL $\Delta R$
      \item EM fraction
      \item NH fraction
      \item CH fraction
      \end{itemize}
    \end{column}
  \end{columns}
\end{frame}

\begin{frame}
  \frametitle{Tensorflow Implementation}
  \begin{itemize}
  \item VHbb group will use new training in Keras (Tensorflow)
  \item Includes energy fractions in rings around the jet
  \item We included additional information, including from PUPPI, to try to improve mass peak
  \end{itemize}
  \centering
  \includegraphics[width=0.8\linewidth]{tf.pdf}
\end{frame}

\begin{frame}
  \frametitle{Pre-fit Plots}
  \fourfigs{Signal}
          {180515_class/inclusive_signal_maier_event_class.pdf}
          {\ttbar}
          {180514__pre/inclusive_tt_cmva_jet2_cmva.pdf}
          {Z + Heavy Flavor}
          {180514__pre/inclusive_heavyz_cmva_jet2_cmva.pdf}
          {Z + Light Flavor}
          {180514__pre/inclusive_lightz_cmva_jet2_cmva.pdf}
\end{frame}

\begin{frame}
  \frametitle{Uncertainties}
  \begin{itemize}
  \item All uncertainties for the cMVA iterative fit scale factors,
    decorrelated in $p_T$ and $\eta$ bins for jets
  \item JES uncertainty (splitting by source in progress)
  \item Allowing major backgrounds, W/Z+jet (split by number of b gen jets) and tt, to float; to be improved by combination
  \item MC statistical uncertainty
  \item QCD renormalization and factorization uncertainties for each process
    (varies but on the order of a few percent)
  \item EWK and QCD Higgs uncertainties
  \item 2.5\% Luminosity uncertainty
  \item Pileup uncertainty currently applied as flat 2.5\%
  \end{itemize}
\end{frame}

\begin{frame}
  \frametitle{Post-fit Plots}
  \fourfigs{Signal}
          {180515_class_post/inclusive_signal_maier_event_class.pdf}
          {\ttbar}
          {180514_v1/inclusive_tt_cmva_jet2_cmva.pdf}
          {Z + Heavy Flavor}
          {180514_v1/inclusive_heavyz_cmva_jet2_cmva.pdf}
          {Z + Light Flavor}
          {180514_v1/inclusive_lightz_cmva_jet2_cmva.pdf}

\end{frame}

\begin{frame}
  \frametitle{Poorly-Behaved W + jets}
  \fourfigs{Z + HF Pre}
           {180514__pre/inclusive_heavyz_cmva_jet2_cmva.pdf}
           {Z + LF Pre}
           {180514__pre/inclusive_lightz_cmva_jet2_cmva.pdf}
           {Z + HF Post}
           {180514_v1/inclusive_heavyz_cmva_jet2_cmva.pdf}
           {Z + LF Post}
           {180514_v1/inclusive_lightz_cmva_jet2_cmva.pdf}
\end{frame}

\begin{frame}
  \frametitle{Result}
  \begin{itemize}
  \item The MC shape is very flexible
    \begin{itemize}
    \item This is possibly due to statstical uncertainties on the MC
    \end{itemize}
  \item W + jets regions are not controlled here at all
    \begin{itemize}
    \item Our W + jets region will be from combination
    \end{itemize}
  \item Expected sensistivity comes out to 1.30
    \begin{itemize}
    \item Anticipate a gain when PandaAnalysis switches from \SI{25}{GeV} jets
      (3\% difference)
    \end{itemize}
  \end{itemize}
\end{frame}

\begin{frame}
  \frametitle{Sample Stitching}
  Potential solution to reduce effect from MC stats
  \begin{itemize}
  \item Use \texttt{VBJets} samples with non-zero number of prompt b 
  \item Use \texttt{BGenFilter} samples with non-zero status 2 b hadrons
  \item Cut on those quantities in nominal samples, \textcolor{blue}{or mix samples}
  \end{itemize}
\end{frame}

\begin{frame}
  \frametitle{2017 Data: NPV re-weighting}

  Re-weigh to match NPV on an inclusive signal, Z+HF, and Z+LF selection

  \vspace{6pt}

  \twofigs{Before re-weighting}
          {180515_npv/npv.pdf}
          {After re-weighting}
          {180515_lumi/npv.pdf}

\end{frame}

\begin{frame}
  \frametitle{2017 Data: Comparison Plots}

  \twofigs{All Triggers}
          {180515_fin/pfmet.pdf}
          {HLT PFMETNoMu120 IDTight}
          {180516_trig/pfmet.pdf}

  \vspace{12pt}

  \begin{itemize}
  \item MET turn-on is slower in the newer data
  \item Overlapping trigger performs worse in 2017 for our ROI
  \item Events will be more boosted
  \end{itemize}

\end{frame}

\begin{frame}
  \frametitle{Boosted Strategy}
  \begin{itemize}
  \item For 2016 data, prioritizing resolved events gave better yields
    \begin{itemize}
    \item $W\ell\nu$ prioritizes boosted events
    \end{itemize}
  \item Otherwise, selection driven by:
    \begin{itemize}
    \item soft drop mass
    \item fat jet $p_T$
    \item double b-tag score
    \end{itemize}
  \item Using CA15 jets, as in Mono-Higgs
  \item Fit to soft drop mass for control regions
  \item Train another classifier BDT for signal region
  \end{itemize}
\end{frame}

\begin{frame}
  \frametitle{W + jets ``Contamination'' (old plots)}
  \fourfigs{signal}
           {180328_v3/boosted_signal_fatjet1_mSD_corr}
           {Z+HF}
           {180328_v1/boosted_heavyz_fatjet1_mSD_corr}
           {Z+LF}
           {180327_v4/boosted_lightz_fatjet1_mSD_corr}
           {tt}
           {180327_v4/boosted_tt_fatjet1_mSD_corr}
\end{frame}

\begin{frame}
  \frametitle{Conclusion}
  \begin{itemize}
  \item With 2016 strategy and minor changes
    \begin{itemize}
    \item Matching their lone datacard
    \item Reason to believe we can still improve
    \end{itemize}
  \item Boosted region as is gives about 10\% improvement over Z($\nu\nu$)
  \item N-tuple bugs prevented a combination last week, but working on
  \item 2017 analysis also delayed, but to be added soon
  \end{itemize}
\end{frame}

%\beginbackup
%
%\begin{frame}
%  \frametitle{Backup Slides}
%\end{frame}
%
%\begin{frame}
   \frametitle{\small 190611/plot\_time\_60000\_wide}
   \centering
   \includegraphics[width=0.6\linewidth]{190611/plot_time_60000_wide.pdf}
\end{frame}

\begin{frame}
   \frametitle{\small 190611/plot\_time\_wide}
   \centering
   \includegraphics[width=0.6\linewidth]{190611/plot_time_wide.pdf}
\end{frame}

\begin{frame}
   \frametitle{\small 190611/plot\_time\_120000\_compare}
   \centering
   \includegraphics[width=0.6\linewidth]{190611/plot_time_120000_compare.pdf}
\end{frame}

\begin{frame}
   \frametitle{\small 190611/plot\_time\_60000\_compare}
   \centering
   \includegraphics[width=0.6\linewidth]{190611/plot_time_60000_compare.pdf}
\end{frame}

\begin{frame}
   \frametitle{\small 190611/plot\_time\_80000\_compare}
   \centering
   \includegraphics[width=0.6\linewidth]{190611/plot_time_80000_compare.pdf}
\end{frame}

\begin{frame}
   \frametitle{\small 190611/plot\_time\_120000\_narrow}
   \centering
   \includegraphics[width=0.6\linewidth]{190611/plot_time_120000_narrow.pdf}
\end{frame}

\begin{frame}
   \frametitle{\small 190611/plot\_time\_40000\_narrow}
   \centering
   \includegraphics[width=0.6\linewidth]{190611/plot_time_40000_narrow.pdf}
\end{frame}

\begin{frame}
   \frametitle{\small 190611/plot\_time\_160000\_narrow}
   \centering
   \includegraphics[width=0.6\linewidth]{190611/plot_time_160000_narrow.pdf}
\end{frame}

\begin{frame}
   \frametitle{\small 190611/plot\_time\_60000\_narrow}
   \centering
   \includegraphics[width=0.6\linewidth]{190611/plot_time_60000_narrow.pdf}
\end{frame}

\begin{frame}
   \frametitle{\small 190611/plot\_time\_80000\_narrow}
   \centering
   \includegraphics[width=0.6\linewidth]{190611/plot_time_80000_narrow.pdf}
\end{frame}

\begin{frame}
   \frametitle{\small 190611/plot\_time\_narrow}
   \centering
   \includegraphics[width=0.6\linewidth]{190611/plot_time_narrow.pdf}
\end{frame}


%
%\backupend

\end{document}
