\documentclass{beamer}

\author[D. Abercrombie]{
  \emph{Daniel Abercrombie}, Brandon Allen, Zeynep Demiragli,
  Guillelmo Gomez-Ceballos, Dylan George Hsu, \\
  Philip Harris, Yutaro Iiyama, Benedikt Maier, \\
  Siddharth Narayanan, Christoph Paus
}

\title{\bf \sffamily VH(bb) Update}
\date{\today}

\usecolortheme{dove}

\usepackage{listings}

\usepackage{fourier}

\usepackage[absolute,overlay]{textpos}
\usefonttheme{serif}
\usepackage{appendixnumberbeamer}
\usepackage{isotope}
\usepackage{hyperref}
\usepackage[english]{babel}
\usepackage{amsmath}
\setbeamerfont{frametitle}{size=\Large,series=\bf\sffamily}
\setbeamertemplate{frametitle}[default][center]
\usepackage{siunitx}
\usepackage{tabularx}
\usepackage{makecell}

\setbeamertemplate{navigation symbols}{}
\usepackage{graphicx}
\usepackage{color}
\setbeamertemplate{footline}[text line]{\parbox{1.083\linewidth}{\footnotesize \hfill \insertshortauthor \hfill \insertpagenumber /\inserttotalframenumber}}
\setbeamertemplate{headline}[text line]{\parbox{1.083\linewidth}{\footnotesize \hspace{-0.083\linewidth} \textcolor{blue}{\sffamily \insertsection \hfill \insertsubsection}}}

\IfFileExists{/Users/dabercro/GradSchool/Presentations/MIT-logo.pdf}
             {\logo{\includegraphics[height=0.5cm]{/Users/dabercro/GradSchool/Presentations/MIT-logo.pdf}}}
             {\logo{\includegraphics[height=0.5cm]{/home/dabercro/MIT-logo.pdf}}}

\usepackage{changepage}

\newcommand{\beginbackup}{
  \newcounter{framenumbervorappendix}
  \setcounter{framenumbervorappendix}{\value{framenumber}}
}
\newcommand{\backupend}{
  \addtocounter{framenumbervorappendix}{-\value{framenumber}}
  \addtocounter{framenumber}{\value{framenumbervorappendix}}
}

\graphicspath{{figs/}}

\newcommand{\link}[2]{\href{#2}{\textcolor{blue}{\underline{#1}}}}
\newcommand{\clink}[2]{\link{#1}{http://t3serv001.mit.edu/~dabercro/redir/?k=#2}}}

\newcommand{\twofigs}[4]{
  \begin{columns}
    \begin{column}{0.5\linewidth}
      \centering
      \textcolor{blue}{#1} \\
      \includegraphics[width=\linewidth]{#2}
    \end{column}
    \begin{column}{0.5\linewidth}
      \centering
      \textcolor{blue}{#3} \\
      \includegraphics[width=\linewidth]{#4}
    \end{column}
  \end{columns}
}

\newcommand{\fourfigs}[8]{
  \begin{columns}
    \begin{column}{0.3\linewidth}
      \centering
      \textcolor{blue}{#1} \\
      \includegraphics[width=\linewidth]{#2} \\
      \textcolor{blue}{#3} \\
      \includegraphics[width=\linewidth]{#4}
    \end{column}
    \begin{column}{0.3\linewidth}
      \centering
      \textcolor{blue}{#5} \\
      \includegraphics[width=\linewidth]{#6} \\
      \textcolor{blue}{#7} \\
      \includegraphics[width=\linewidth]{#8}
    \end{column}
  \end{columns}
}

\newcommand{\wfourfigs}[8]{
  \begin{columns}
    \begin{column}{0.45\linewidth}
      \centering
      \textcolor{blue}{#1} \\
      \includegraphics[width=\linewidth]{#2} \\
      \textcolor{blue}{#3} \\
      \includegraphics[width=\linewidth]{#4}
    \end{column}
    \begin{column}{0.45\linewidth}
      \centering
      \textcolor{blue}{#5} \\
      \includegraphics[width=\linewidth]{#6} \\
      \textcolor{blue}{#7} \\
      \includegraphics[width=\linewidth]{#8}
    \end{column}
  \end{columns}
}

\newcommand{\ttbar}{\ensuremath{t\bar{t}}}

\begin{document}

\begin{frame}[nonumbering]
  \titlepage

\end{frame}

\begin{frame}
  \frametitle{Introduction}
  \begin{itemize}
  \item Updates on the $W\ell\nu$ channel presented by Dylan at a previous meeting
  \item Status of the $Z\nu\nu$ (resolved) repeating existing analysis
  \item Initial studies of boosted $HbbZ\nu\nu$ sensitivity improvements
  \end{itemize}
\end{frame}

\section{$W\ell\nu$ Updates (Resolved)}

\begin{frame}
  \frametitle{$W\ell\nu$ Status}
  \link{Introduction slides by Dylan}{https://indico.cern.ch/event/690727/contributions/2890780/attachments/1603908/2545128/2018-02-21_statusReportVH.pdf}

  \begin{columns}
    \begin{column}{0.4\linewidth}
      \begin{itemize}
      \item Repeat the WH analysis strategy of HIG-16-044
      \item Obtain expected limits in agreement with the paper
      \end{itemize}
    \end{column}
    \begin{column}{0.6\linewidth}
      \includegraphics[width=\linewidth]{dylan/WHSR_WToENu_MVAVar.pdf}
    \end{column}
  \end{columns}

\end{frame}

\begin{frame}
  \frametitle{WH Resolved Control Regions}

  \wfourfigs{}{dylan/WH2TopCR_WToENu_MVAVar.pdf}
            {}{dylan/WHHeavyFlavorLowMassCR_WToMuNu_MVAVar.pdf}
            {}{dylan/WHLightFlavorCR_WToMuNu_MVAVar.pdf}
            {}{dylan/WHHeavyFlavorHighMassCR_WToENu_MVAVar.pdf}

  \centering
  \textcolor{blue}{Prefit shapes, before extracting the W+bb SF in W+HF region}
\end{frame}

\section{$W\ell\nu$ Updates (Boosted)}

\begin{frame}
  \frametitle{WH Boosted Strategy}
  \begin{columns}
    \begin{column}{0.4\linewidth}
      \hspace*{-0.2\linewidth}
      \includegraphics[width=1.9\linewidth]{dylan/whBoostedStrategyDiagram.pdf}
    \end{column}
    \begin{column}{0.6\linewidth}
      \begin{itemize}
      \item {\bf Take events with:} \\
        PUPPI AK8 fatjet with $p_T > \SI{250}{GeV}$,
        soft drop mass $(m_{SD}) > \SI{40}{GeV}$,
        $(W p_t) > \SI{250}{GeV}$, and
        $\Delta \phi(W, \mathrm{fatjet}) > \SI{2.5}{rad}$
      \item {\bf Split into signal and control regions based on:}
        \begin{itemize}
        \item If there are b-tagged jets outside the fat jet
        \item Fat jet double b-tag score $> 0.8$
        \item $m_{SD}$
        \end{itemize}
      \item Measure the double-b efficiency scale factors for W+LF and $t\overline{t}$, \\
        linking regions in the diagram
      \end{itemize}
    \end{column}
  \end{columns}
\end{frame}

\begin{frame}
  \frametitle{Double-B SFs with W+HF}
  \begin{columns}
    \begin{column}{0.5\linewidth}
      \begin{itemize}
      \item This is a new feature
      \item Table quotes the MC efficiencies for
        double-B$> 0.8$ in the case of 0 or 1+ isolated b-tagged jets
      \item Plots show attempt to measure W+HF fatjets \\
        (0 isolated b-tagged jets, double-b$> 0.8$, $40 < m_{SD} < \SI{80}{GeV}$)
      \item Currently applying 25\% uncertainty due to low purity
      \item Plan to use double-b scale factors from the Higgs-bb Tagging group
      \end{itemize}
    \end{column}
    \begin{column}{0.7\linewidth}
      {\footnotesize
        \lstinputlisting{bsf.txt}
      }
      \includegraphics[width=\linewidth]{dylan/WHHeavyFlavorFJCR_WToMuNu_minSubjetCSV.pdf}
    \end{column}
  \end{columns}
\end{frame}

\begin{frame}
  \frametitle{Boosted W+LF and TTbar Samples}

  \begin{columns}
    \begin{column}{0.45\linewidth}
      \begin{itemize}
      \item Notice W+LF overestimates by 15\% due to $m_{SD}$ differences in data and MC
      \item Flat SF applied to W+jet of 85\% with 15\% uncertainty added to account for this
      \item Needs additional study
      \end{itemize}
      \centering
      \textcolor{blue}{W+LF, double-b$< 0.8$, 0 $iso_b$} \\
      \includegraphics[width=\linewidth]{dylan/WHLightFlavorFJCR_WToENu_MVAVar.pdf} \\
    \end{column}
    \begin{column}{0.45\linewidth}
      \centering
      \textcolor{blue}{TT Failing, \\ double-b$< 0.8$, 1+ $iso_b$} \\
      \includegraphics[width=\linewidth]{dylan/WHTT1bFJCR_WToMuNu_MVAVar.pdf} \\
      \textcolor{blue}{TT Passing, \\ double-b$> 0.8$, 1+ $iso_b$} \\
      \includegraphics[width=\linewidth]{dylan/WHTT2bFJCR_WToMuNu_MVAVar.pdf}
    \end{column}
  \end{columns}

\end{frame}

\begin{frame}
  \frametitle{Boosted WH BDT Classifier}
  \begin{columns}
    \begin{column}{0.5\linewidth}
      \begin{itemize}
      \item Train a BDT event classifier for the boosted category using 13 variables
      \item Most useful variable is the soft drop mass
      \item MC statistics are a limiting factor
      \end{itemize}
      \centering
      \includegraphics[width=0.8\linewidth]{dylan/CorrelationMatrixS_optimizeBoostedAda.png}
    \end{column}
    \begin{column}{0.5\linewidth}
      \centering
      \textcolor{blue}{Over-training check} \\
      \includegraphics[width=0.8\linewidth]{dylan/over.png} \\
      \textcolor{blue}{$\tau_2/\tau_1$ in W+LF CR} \\
      \includegraphics[width=0.8\linewidth]{dylan/WHLightFlavorFJCR_WToMuNu_fj1Tau21.pdf}
    \end{column}
  \end{columns}
\end{frame}

\begin{frame}
  \frametitle{Latest Results for Boosted WH}
  \begin{columns}
    \begin{column}{0.4\linewidth}
      \hspace*{-0.1\linewidth}
      \includegraphics[width=1.2\linewidth]{dylan/WHFJSR_WToMuNu_MVAVar.pdf}

      {With this splitting scheme, boosted category is statistics limited}

    \end{column}
    \begin{column}{0.7\linewidth}
      \begin{itemize}
      \item Perform a simultaneous fit using 2016 data in the control regions
        \begin{itemize}
        \item Boosted WH BDT in the SR
        \item $m_SD$ in the boosted CRs
        \item Resolved regions from the orthodox analysis
        \end{itemize}
      \item Conservative systematic uncertainties
        \begin{itemize}
        \item Measured the double-b SFs
          \begin{itemize}
          \item {\bf W+LF:} $1.50 \pm 0.12$
          \item {\bf ttbar:} $1.12 \pm 0.02$
          \end{itemize}
        \item Log-normal uncertainty of 25\% on the $B_2$ SF for W+b, W+bb
        \item SF with 15\% uncertainty for $m_{SD}$ cut on W+jets
        \item BDT shape uncertainty due to JES
        \end{itemize}
      \item {\bf Expected limit improves by 5\%}
      \end{itemize}
    \end{column}
  \end{columns}
\end{frame}

\begin{frame}
  \frametitle{Next Steps for WH}
  \begin{itemize}
  \item Try to measure the W+HF $B_W$ SF by looking at muon-tagged QCD events
  \item Understand better the W+jets $m_{SD}$ cut scale factor
  \item Improve the $m_{SD}$ resolution for improved BDT performance
    \begin{itemize}
    \item Can we reclaim energy from b-decays through regression?
    \item Can we study the JES uncertainty to reduce it?
    \end{itemize}
  \end{itemize}
\end{frame}

\section{$Z\nu\nu$}

\begin{frame}
  \frametitle{$Z\nu\nu$ Introduction}
  List of samples and triggers are in the backup slides

  \begin{itemize}
  \item We have not reached the sensitivity of the 2016 analysis
  \item Need a benchmark analysis to start the boosted investigation
  \item From this framework, we can explore potential improvements
    from focusing on boosted events
  \end{itemize}

\end{frame}

\section{$Z\nu\nu$ (Resolved) Replication}

\begin{frame}
  \frametitle{$Z\nu\nu$ Selection Summary}
  \begin{itemize}
  \item MET $> \SI{170}{GeV}$
  \item Sort central jets $|\eta| < 2.4$ with $p_T > \SI{25}{GeV}$ by cMVA, \\
    and take top two to reconstruct a potential $H \rightarrow bb$ decay
  \item Lower cMVA must pass loose working point (cMVA $> -0.5884$)
  \item Other cMVA depends on the region
    \begin{itemize}
    \item Pass tight working point for Z+HF and signal
    \item Pass medium working point for tt
    \item Fail medium working point for Z+LF
    \end{itemize}
  \item One must have $p_T > \SI{60}{GeV}$ both $p_T > \SI{35}{GeV}$
  \item di-jet $p_T > \SI{70}{GeV}$, di-jet $m < \SI{500}{GeV}$, $\Delta \phi$(MET, di-jet)$ > 2.0$
  \item Exactly one tight lepton in the tt region \\ (one or more loose leptons)
  \item Zero preselected leptons in other regions, \\
    as defined in AN-2015-168
  \item $N_{jet} \le 3$ (signal, Z+LF), $N_{jet} \le 2$ (Z+HF), $N_{jet} \ge 4$ (tt) ($|\eta| < 2.5$)
  \item At least on b with $\Delta \phi($MET,$b) < \pi$ (tt), \\
    or no jets with $\Delta \phi($MET, $j) < 0.5$ (all other regions)
  \end{itemize}
\end{frame}

\begin{frame}
  \frametitle{Data Yields}
  Slightly modified: Add that first cMVA jet must meet harder $p_T > \SI{60}{GeV}$ cut
  and signal cuts out some background with BDT cut

  \hspace{24pt}

  \begin{tabular}{c c c c}
    Region & Previous Slide & Slightly Modified & 2016 Analysis \\
    \hline
    Signal & 6656           & 4830              & 3865          \\
    Z+HF   & 3789           & 2878              & 2487          \\
    Z+LF   & 14321          & 10365             & 8781          \\
    tt     & 11102          & 7849              & 7931
  \end{tabular}

\end{frame}

\begin{frame}
  \frametitle{B-jet regression: Training variables}

  Need to recover energy lost in $b$ decay. We train in a fully-leptonic tt sample
  with 2 tight leptons.

  \hspace{12pt}

  Train on the jet with the highest CMVA value, using the following

  \begin{columns}
    \begin{column}{0.33\linewidth}
      \begin{itemize}
      \item NPV
      \item \# central jets
      \item \# total jets
      \item $p_T$
      \item $\eta$
      \item $\phi$
      \item mass
      \end{itemize}
    \end{column}
    \begin{column}{0.33\linewidth}
      \begin{itemize}
      \item CSV
      \item CMVA
      \item QGL
      \item SV \# tracks
      \item SV $p_T$
      \item SV mass
      \item SV distance
      \item SV uncertainty
      \end{itemize}
    \end{column}
    \begin{column}{0.33\linewidth}
      \begin{itemize}
      \item Max track $p_T$
      \item \# leptons
      \item lead lepton $p_T$
      \item LL $p_{T, \perp}$
      \item LL $\Delta R$
      \item EM fraction
      \item NH fraction
      \item CH fraction
      \end{itemize}
    \end{column}
  \end{columns}

\end{frame}

\begin{frame}
  \frametitle{Regression Performance on Signal Sample}

  Higgs mass peak is improved

  \twofigs{Masses with Gen Mass}
          {180209_v1/regression_withgen.pdf}
          {Without Gen (Zoomed)}
          {180209_v1/regression.pdf}

  \begin{itemize}
  \item \textcolor{red}{Old Regression} is the regression Dylan previously showed
    (missing a few variables)
  \item \textcolor{green}{New Regression} includes all the variables in the analysis note
  \item Gives good improvement to expect limits,
    but we expect regression from previous efforts will be much better once available
  \end{itemize}

\end{frame}

\begin{frame}
  \frametitle{Signal classification}
  We train a BDT for signal identification in the signal region, minus the $N_{jet}$ cut,
  using the following variables

  \begin{columns}
    \begin{column}{0.5\linewidth}
      \begin{itemize}
      \item MET
      \item regressed di-jet mass
      \item regressed di-jet pt
      \item regressed di-jet pt over MET
      \item cMVA of top three cMVA jets
      \item $p_T$ of top three cMVA jets
      \end{itemize}
    \end{column}
    \begin{column}{0.5\linewidth}
      \begin{itemize}
      \item min($\Delta \phi$(MET, jet))
      \item di-jet $\Delta R$
      \item di-jet $\Delta \eta$
      \item di-jet $\Delta \phi$
      \item $\Delta \phi$(MET, di-jet)
      \item Soft activity ($> \SI{5}{GeV}$)
      \item Number of jets
      \end{itemize}
    \end{column}
  \end{columns}
\end{frame}

\begin{frame}
  \frametitle{CR plots (Pre-Fit)}
  Control with cMVA of the lower cMVA jet, and plot BDT for signal

  \fourfigs{Signal region}
           {180327_v1/inclusive_signal_event_class_reg_3.pdf}
           {Z+HF CR}
           {180327_inc/inclusive_heavyz_cmva_jet2_cmva.pdf}
           {Z+LF CR}
           {180327_inc/inclusive_lightz_cmva_jet2_cmva.pdf}
           {tt CR}
           {180327_inc/inclusive_tt_cmva_jet2_cmva.pdf}
\end{frame}

\begin{frame}
  \frametitle{List of systematics}
  \begin{itemize}
  \item 30\% uncertainty on di-boson and single top cross sections
  \item QCD renormalization and factorization uncertainties for each separate process \\
    (varies but on the order of a few percent)
  \item 2.5\% Luminosity uncertainty
  \item Pileup uncertainty currently applied as flat 5\%
  \item All of the uncertainties available for the cMVA iterative fit scale factors
  \item Large shape uncertainty (to event classifier) from jet $p_T$ uncertainty,
    which moves a number of the di-jet kinematic and counting variables around \\
    (I'd estimate as high as ~10\% for various bins)
  \item Allowing major backgrounds, W/Z+jet (split by number of b-jets) and tt, to float, which should be improved by study
  \end{itemize}
\end{frame}

\begin{frame}
  \frametitle{CR plots (Post-Fit)}
  \fourfigs{Signal region}
           {180327_yesterday_post/inclusive_signal_event_class_reg_3.pdf}
           {Z+HF CR}
           {180327_post/inclusive_heavyz_cmva_jet2_cmva.pdf}
           {Z+LF CR}
           {180327_post/inclusive_lightz_cmva_jet2_cmva.pdf}
           {tt CR}
           {180327_post/inclusive_tt_cmva_jet2_cmva.pdf}

  We currently have an expect limit of 1.30
\end{frame}

\section{$Z\nu\nu$ Boosted}

\begin{frame}
  \frametitle{Boosted Selection}
  \begin{itemize}
  \item We found that doing a ``boosted veto'' on the resolved events took too many
    events out of the analysis, and reduced sensitivity
  \item Benedikt and I tried two different approaches to using boosted variables
    \begin{itemize}
    \item Add boosted jet kinematic and substructure variables to the event classifier BDT
    \item Instead, create a boosted selection that uses a ``resolved veto'' selection
    \end{itemize}
  \item These two improvements seem to stack when put together
  \end{itemize}
\end{frame}

\begin{frame}
  \frametitle{Additional BDT Variables}
  The following variables are filled with dummy values when no fat (PUPPI AK8) jet exists
  \begin{columns}
    \begin{column}{0.5\linewidth}
      \begin{itemize}
      \item $\Delta \phi$(fat jet, MET)
      \item fat jet $p_T$
      \item fat jet $\eta$
      \item fat jet $m_{SD}$
      \item $\frac{ECF(2, 3, 10)}{ECF(1, 2, 10)^2}$
      \item $\Delta R$(fat jet, di-jet)
      \end{itemize}
    \end{column}
    \begin{column}{0.5\linewidth}
      This gives a BDT with improvments of $\approx 5\%$ on the expected limit
    \end{column}
  \end{columns}
\end{frame}

\begin{frame}
  \frametitle{Resolved Veto}
  \begin{itemize}
  \item Require that fewer than 2 AK4 jets pass the cMVA loose working point
  \item Fat jet $p_T > \SI{250}{GeV}$
  \item Fat jet $m_{SD} > \SI{40}{GeV}$
  \item Use $B_2$ cuts instead of leading jet cMVA cuts to separate HF and LF regions
    \begin{itemize}
    \item HF and signal: $B_2 > 0.8$
    \item tt: $B_2 > 0$
    \item LF: $-0.5 < B_2 < 0.8$
    \end{itemize}
  \item Scale factors for fat jet selection still need to be implemented,
    so instead, large $\approx 50\%$, decorrelated (except for signal + HF)
    systematics are applied
  \end{itemize}
\end{frame}

\begin{frame}
  \frametitle{Pre-Fit Plots}
  For now, just constraining based on the $m_SD$ in all regions

  NOTE: I KNOW THESE PLOTS NEED BETTER AXES SCALING
  \fourfigs{signal}
           {180326_v3/boosted_signal_fatjet1_mSD_corr}
           {Z+HF}
           {180326_v3/boosted_heavyz_fatjet1_mSD_corr}
           {Z+LF}
           {180326_v3/boosted_lightz_fatjet1_mSD_corr}
           {tt}
           {180326_v3/boosted_tt_fatjet1_mSD_corr}
\end{frame}

\begin{frame}
  \frametitle{Post-Fit Plots}
  \fourfigs{signal}
           {180327_yesterday_post/boosted_signal_fatjet1_mSD_corr}
           {Z+HF}
           {180327_yesterday_post/boosted_heavyz_fatjet1_mSD_corr}
           {Z+LF}
           {180327_yesterday_post/boosted_lightz_fatjet1_mSD_corr}
           {tt}
           {180327_yesterday_post/boosted_tt_fatjet1_mSD_corr}

  Expect limit improved by 5\%.
  With fat jet variables in BDT, we get all the way down to 1.20
\end{frame}

\section{}

\begin{frame}
  \frametitle{Conclusion}

\end{frame}

\beginbackup

\section{Backup slides}

\begin{frame}
  \frametitle{Backup Slides}
\end{frame}

\begin{frame}
  \frametitle{Data samples and triggers}
  Using MET 2016, 03Feb2017 samples with \SI{35.9}{\per fb}:
  \texttt{/MET/Run2016*-03Feb2017\_*/MINIAOD}

  \vspace{12pt}

  Triggers
  \begin{itemize}
  \item \texttt{HLT\_PFMET170\_NoiseCleaned}
  \item \texttt{HLT\_PFMET170\_HBHECleaned}
  \item \texttt{HLT\_PFMET170\_JetIdCleaned}
  \item \texttt{HLT\_PFMET170\_NotCleaned}
  \item \texttt{HLT\_PFMET170\_HBHE\_BeamHaloCleaned}
  \item \texttt{HLT\_PFMETNoMu120\_NoiseCleaned\_PFMHTNoMu120\_IDTight}
  \item \texttt{HLT\_PFMETNoMu110\_NoiseCleaned\_PFMHTNoMu110\_IDTight}
  \item \texttt{HLT\_PFMETNoMu90\_NoiseCleaned\_PFMHTNoMu90\_IDTight}
  \item \texttt{HLT\_PFMETNoMu90\_PFMHTNoMu90\_IDTight}
  \item \texttt{HLT\_PFMETNoMu100\_PFMHTNoMu100\_IDTight}
  \item \texttt{HLT\_PFMETNoMu110\_PFMHTNoMu110\_IDTight}
  \item \texttt{HLT\_PFMETNoMu120\_PFMHTNoMu120\_IDTight}
  \end{itemize}

\end{frame}

\begin{frame}
  \frametitle{MC Samples}
  All MC samples from the campaign: \texttt{RunIISummer16MiniAODv2-PUMoriond17\_ \\
    80X\_mcRun2\_asymptotic\_2016\_TrancheIV}

  \vspace{12pt}

  {\footnotesize
    \begin{itemize}
    \item \texttt{QCD\_HT*\_TuneCUETP8M1\_13TeV-madgraphMLM-pythia8}
    \item \texttt{TTJets\_SingleLeptFromT\{,bar\}\_TuneCUETP8M2T4\_13TeV- \\
      amcatnloFXFX-pythia8}
    \item \texttt{TTTo2L2Nu\_TuneCUETP8M2\_ttHtranche3\_13TeV-powheg-pythia8}
    \item \texttt{ST\_t-channel\_\{anti,\}top\_4f\_inclusiveDecays\_13TeV- \\
      powhegV2-madspin-pythia8\_TuneCUETP8M1}
    \item \texttt{ST\_tW\_\{anti,\}top\_5f\_inclusiveDecays\_13TeV- \\
      powheg-pythia8\_TuneCUETP8M1}
    \item \texttt{WJetsToLNu\_HT-*\_TuneCUETP8M1\_13TeV-madgraphMLM-pythia8}
    \item \texttt{ZJetsToNuNu\_HT-*\_13TeV-madgraph}
    \item \texttt{\{WW, WZ, ZZ\}\_TuneCUETP8M1\_13TeV-pythia8}
    \end{itemize}
  }
\end{frame}

\begin{frame}
  \frametitle{Signal Samples}

  {
    \begin{itemize}
    \item \texttt{GluGluHToBB\_M125\_13TeV\_powheg\_pythia8}
    \item \texttt{WminusH\_HToBB\_WToLNu\_M125\_13TeV\_powheg\_pythia8}
    \item \texttt{WplusH\_HToBB\_WToLNu\_M125\_13TeV\_powheg\_pythia8}
    \item \texttt{ZH\_HToBB\_ZToLL\_M125\_13TeV\_powheg\_pythia8}
    \item \texttt{ZH\_HToBB\_ZToNuNu\_M125\_13TeV\_powheg\_pythia8}
    \item \texttt{ggZH\_HToBB\_ZToLL\_M125\_13TeV\_powheg\_pythia8}
    \item \texttt{ggZH\_HToBB\_ZToNuNu\_M125\_13TeV\_powheg\_pythia8}
    \end{itemize}
  }
\end{frame}

%\begin{frame}
   \frametitle{\small 190611/plot\_time\_60000\_wide}
   \centering
   \includegraphics[width=0.6\linewidth]{190611/plot_time_60000_wide.pdf}
\end{frame}

\begin{frame}
   \frametitle{\small 190611/plot\_time\_wide}
   \centering
   \includegraphics[width=0.6\linewidth]{190611/plot_time_wide.pdf}
\end{frame}

\begin{frame}
   \frametitle{\small 190611/plot\_time\_120000\_compare}
   \centering
   \includegraphics[width=0.6\linewidth]{190611/plot_time_120000_compare.pdf}
\end{frame}

\begin{frame}
   \frametitle{\small 190611/plot\_time\_60000\_compare}
   \centering
   \includegraphics[width=0.6\linewidth]{190611/plot_time_60000_compare.pdf}
\end{frame}

\begin{frame}
   \frametitle{\small 190611/plot\_time\_80000\_compare}
   \centering
   \includegraphics[width=0.6\linewidth]{190611/plot_time_80000_compare.pdf}
\end{frame}

\begin{frame}
   \frametitle{\small 190611/plot\_time\_120000\_narrow}
   \centering
   \includegraphics[width=0.6\linewidth]{190611/plot_time_120000_narrow.pdf}
\end{frame}

\begin{frame}
   \frametitle{\small 190611/plot\_time\_40000\_narrow}
   \centering
   \includegraphics[width=0.6\linewidth]{190611/plot_time_40000_narrow.pdf}
\end{frame}

\begin{frame}
   \frametitle{\small 190611/plot\_time\_160000\_narrow}
   \centering
   \includegraphics[width=0.6\linewidth]{190611/plot_time_160000_narrow.pdf}
\end{frame}

\begin{frame}
   \frametitle{\small 190611/plot\_time\_60000\_narrow}
   \centering
   \includegraphics[width=0.6\linewidth]{190611/plot_time_60000_narrow.pdf}
\end{frame}

\begin{frame}
   \frametitle{\small 190611/plot\_time\_80000\_narrow}
   \centering
   \includegraphics[width=0.6\linewidth]{190611/plot_time_80000_narrow.pdf}
\end{frame}

\begin{frame}
   \frametitle{\small 190611/plot\_time\_narrow}
   \centering
   \includegraphics[width=0.6\linewidth]{190611/plot_time_narrow.pdf}
\end{frame}



\backupend

\end{document}
