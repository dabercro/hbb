\documentclass{beamer}

\author[D. Abercrombie]{
  {\bf DESY}: A. de Wit, R. Mankel, S. Kaveh, A. Nigamova \\
  \vfill
  {\bf ETH}: C. Grab, A. Calandri, P. Berger, G. Perrin, K. Gedia, \\ C. Reissel, B. Kaech, L. Perrozzi \\
  \vfill
  {\bf RBI}: V. Brigljevic, B. Chitroda, D. Ferencek, \\ S. Mishra, T. Susa \\
  \vfill
  {\bf MIT}: \emph{D. Abercrombie}, G. G\'omez-Ceballos, D. Kovalskyi, \\ B. Maier, C. Paus \\
  \vfill
  {\bf U. Florida}: J. Konigsberg \\
  \vfill
  {\bf RWTH}: A. Schmidt \\
  \vfill
  {\bf INFN Pisa}: A. Rizz
}

\title{\bf \sffamily B-Jet Regression and Di-jet Kinematics}
\date{24 June, 2020}

\usecolortheme{dove}

\usepackage[absolute,overlay]{textpos}
\usefonttheme{serif}
\usepackage{appendixnumberbeamer}
\usepackage{isotope}
\usepackage{hyperref}
\usepackage[english]{babel}
\usepackage{amsmath}
\setbeamerfont{frametitle}{size=\Large,series=\bf\sffamily}
\setbeamertemplate{frametitle}[default][center]
\usepackage{siunitx}
\usepackage{tabularx}
\usepackage{makecell}
\usepackage{comment}

\usepackage{feynmp-auto}

\setbeamertemplate{navigation symbols}{}
\usepackage{graphicx}
\usepackage{color}
\setbeamertemplate{footline}[text line]{\parbox{1.083\linewidth}{\footnotesize \hfill \insertshortauthor \hfill \insertpagenumber /\inserttotalframenumber}}
\setbeamertemplate{headline}[text line]{\parbox{1.083\linewidth}{\footnotesize \hspace{-0.083\linewidth} \textcolor{blue}{\sffamily \insertsection \hfill \insertsubsection}}}

\IfFileExists{/Users/dabercro/GradSchool/Presentations/MIT-logo.pdf}
             {\logo{\includegraphics[height=0.5cm]{/Users/dabercro/GradSchool/Presentations/MIT-logo.pdf}}}
             {\logo{\includegraphics[height=0.5cm]{/home/dabercro/MIT-logo.pdf}}}

\usepackage{changepage}

\newcommand{\beginbackup}{
  \newcounter{framenumbervorappendix}
  \setcounter{framenumbervorappendix}{\value{framenumber}}
}
\newcommand{\backupend}{
  \addtocounter{framenumbervorappendix}{-\value{framenumber}}
  \addtocounter{framenumber}{\value{framenumbervorappendix}}
}

\graphicspath{{figs/}}

\newcommand{\link}[2]{\href{#2}{\textcolor{blue}{\underline{#1}}}}
\newcommand{\clink}[2]{\link{#1}{http://t3serv001.mit.edu/~dabercro/redir/?k=#2}}}

\newcommand{\twofigs}[4]{
  \begin{columns}
    \begin{column}{0.5\linewidth}
      \centering
      \textcolor{blue}{#1} \\
      \includegraphics[width=\linewidth]{#2}
    \end{column}
    \begin{column}{0.5\linewidth}
      \centering
      \textcolor{blue}{#3} \\
      \includegraphics[width=\linewidth]{#4}
    \end{column}
  \end{columns}
}

\newcommand{\threefigs}[6]{
  \begin{columns}
    \begin{column}{0.33\linewidth}
      \centering
      \textcolor{blue}{#1}
      \includegraphics[width=\linewidth]{#2}
    \end{column}
    \begin{column}{0.34\linewidth}
      \centering
      \textcolor{blue}{#3}
      \includegraphics[width=\linewidth]{#4}
    \end{column}
    \begin{column}{0.33\linewidth}
      \centering
      \textcolor{blue}{#5}
      \includegraphics[width=\linewidth]{#6}
    \end{column}
  \end{columns}
}

\newcommand{\fourfigs}[8]{
  \begin{columns}
    \begin{column}{0.3\linewidth}
      \centering
      \textcolor{blue}{#1} \\
      \includegraphics[width=\linewidth]{#2} \\
      \textcolor{blue}{#3} \\
      \includegraphics[width=\linewidth]{#4}
    \end{column}
    \begin{column}{0.3\linewidth}
      \centering
      \textcolor{blue}{#5} \\
      \includegraphics[width=\linewidth]{#6} \\
      \textcolor{blue}{#7} \\
      \includegraphics[width=\linewidth]{#8}
    \end{column}
  \end{columns}
}

\newcommand{\ttbar}{\ensuremath{t\bar{t}}}
\newcommand{\bbbar}{\ensuremath{b\bar{b}}}

\begin{document}

\begin{frame}
  \begin{center}
  \usebeamerfont{title}\inserttitle \\
  \end{center}
  \vfill
  \usebeamerfont{author}\insertauthor \\
  \vfill
  \begin{center}
  \usebeamerfont{date}\insertdate
  \end{center}
\end{frame}

\begin{frame}
  \frametitle{Introduction}

  Two studies are presented for use in the VHbb analysis

  \vfill

  Smearing of b-jets:
  \begin{itemize}
  \item Smearing needs to be re-derived after applying
    energy regression designed to recover energy
    lost in the weak decays of b-jets.
  \item Extensions for DY in NanoAODv7 are available
  \item An explanation and demonstration of smearing is given
  \item Implemented suggestions from previous talk\footnote{\url{https://indico.cern.ch/event/893633/contributions/3774118/attachments/1997977/3333898/dabercro_hbb_200304.pdf}} in JetMET
  \end{itemize}

  Di-jet Kinematics in 2018:
  \begin{itemize}
  \item Plots for $Z\rightarrow \ell\ell$ 2018 response are shown
  \item The di-jet mass performance in signal samples is shown
    for various $V_{p_T}$ and number of additional jets
  \end{itemize}

\end{frame}

\begin{frame}
  \frametitle{Jet Resolution Approach}

  \begin{itemize}
  \item Good detector resolution of leptons allows us to \\
    take advantage of the following process:
  \end{itemize}

  \hfill

  \begin{center}
    \begin{fmffile}{DY_bjet}
      \begin{fmfgraph*}(200, 120)
        \fmfleft{i1,i0}
        \fmfright{o3,o2,o1,o0}
        \fmf{fermion, label=$q$}{i0,v0,v1,i1}
        \fmf{photon, label=$Z$}{v0,v0_1}
        \fmf{fermion, label=$\ell$}{o0,v0_1,o1}
        \fmf{gluon}{v1,v1_1}
        \fmf{fermion, label=$b$}{o2,v1_1,o3}
      \end{fmfgraph*}
    \end{fmffile}
  \end{center}

  \hfill

  \begin{itemize}
  \item The jet resolution can be measured by assuming it is \\
    balanced against the $Z$ in the transverse plane
  \end{itemize}

\end{frame}

\begin{frame}
  \frametitle{Selection -- Leptons}

  \begin{itemize}
  \item Two leptons satisfying:
    \begin{itemize}
    \item $p_T > \SI{20}{GeV}$
    \item $q_1 + q_2 = 0$
    \item Same flavor
    \item Z Selection:
      \begin{itemize}
      \item Muons: \\
        \texttt{pfRelIso04\_all} $< 0.25$ and \\
        $\Delta xy < 0.05$ and $\Delta z < 0.2$
      \item Electrons: \\
        \texttt{mvaFall17V2Iso\_WP90} and \texttt{pfRelIso03\_all} $< 0.15$
      \end{itemize}
    \end{itemize}
  \item Any third lepton (either muon or electron) causes event to be thrown out
  \item $\SI{71}{GeV} < m_{\ell\ell} < \SI{111}{GeV}$
  \item $p_{T,\ell\ell} > \SI{100}{GeV}$
  \end{itemize}

\end{frame}


\begin{frame}
  \frametitle{Selection -- Jets}

  We want one soft jet and one hard jet for the balancing

  \begin{itemize}
  \item Exactly two jets satisfying:
    \begin{itemize}
    \item Does not overlap with any leptons satisfying Z Selection criteria
    \item $p_T > \SI{20}{GeV}$ after regression
    \item $|\eta| < 2.4$
    \item \texttt{jetId} $> 0$ and \texttt{puId} $> 0$
    \end{itemize}
  \item Leading jet satisfies:
    \begin{itemize}
    \item $\Delta\phi(j, \ell\ell) > 2.8$
    \item DeepCSV passing the tight working point
    \end{itemize}
  \item Trailing jet statisfies: $\alpha = \frac{p_{T, j}}{p_{T, \ell\ell}} < 0.3$
  \end{itemize}

\end{frame}


\begin{frame}
  \frametitle{Binned Fit Strategy}

  \begin{itemize}
  \item We extrapolate down to $\alpha = 0$
  \item Use four bins of $\alpha$, each with roughly equal data yields:
  \end{itemize}
  \begin{columns}
    \begin{column}{0.5\linewidth}
      \includegraphics[width=\linewidth]{200622_alpha_lines/smearplot_alpha.pdf}
    \end{column}
    \begin{column}{0.5\linewidth}
      \[
      \alpha
      \begin{cases}
        0 < \alpha < 0.155 \\
        0.155 \le \alpha < 0.185 \\
        0.185 \le \alpha < 0.23 \\
        0.23 \le \alpha < 0.3
      \end{cases}
      \]
    \end{column}
  \end{columns}

\end{frame}


\begin{frame}
  \frametitle{Binned Fit Strategy}

  \begin{columns}
    \begin{column}{0.3\linewidth}
      \centering
      \includegraphics[width=\linewidth]{200619_2018/smearplot_1_jet1_adjusted_response.pdf} \\
      \includegraphics[width=\linewidth]{200619_2018/smearplot_3_jet1_adjusted_response.pdf}
    \end{column}
    \begin{column}{0.3\linewidth}
      \centering
      \includegraphics[width=\linewidth]{200619_2018/smearplot_2_jet1_adjusted_response.pdf} \\
      \includegraphics[width=\linewidth]{200619_2018/smearplot_4_jet1_adjusted_response.pdf}
    \end{column}
  \end{columns}

  \vfill
  MC error includes PSWeights and Scale/Renormalization

\end{frame}


\begin{frame}
  \frametitle{Binned Fit Formula}

  For smearing, we fit $\sigma/\mu$

  \[
  f(\alpha) = (m \times \alpha) \oplus b \times (1 + c_k \times \alpha)
  \]

  \begin{itemize}
  \item $c_k$ is fixed by a linear fit to the MC's intrinsic jet resolution ($p_{T, reco}/p_{T, gen}$) over $\alpha$ as $c_k = m_0/q_0$
  \item Smearing is done by scaling difference between $p_{T,reco}$ and $p_{T,gen}$ by $b_{data}/b_{MC}$
  \end{itemize}

  \vfill

  Scaling is done by a simple linear fit to $\mu$

\end{frame}


\begin{frame}
  \frametitle{Fits for All Years}

  Fits to $\mu$ for scaling and $\sigma/\mu$ for smearing
  \vfill

  \begin{columns}
    \begin{column}{0.33\linewidth}
      \centering
      \textcolor{blue}{2016} \\
      \includegraphics[width=\linewidth]{figs/200623_smear_200619_2016_divmean/mean_jet1_adjusted_response_smear_0.pdf} \\
      \includegraphics[width=\linewidth]{figs/200619_smear_200619_2016_divmean/resolution_jet1_adjusted_response_smear_0.pdf}
    \end{column}
    \begin{column}{0.34\linewidth}
      \centering
      \textcolor{blue}{2017} \\
      \includegraphics[width=\linewidth]{figs/200623_smear_200619_2017_divmean/mean_jet1_adjusted_response_smear_0.pdf} \\
      \includegraphics[width=\linewidth]{figs/200619_smear_200619_2017_divmean/resolution_jet1_adjusted_response_smear_0.pdf}
    \end{column}
    \begin{column}{0.33\linewidth}
      \centering
      \textcolor{blue}{2018} \\
      \includegraphics[width=\linewidth]{figs/200623_smear_200619_2018_divmean/mean_jet1_adjusted_response_smear_0.pdf} \\
      \includegraphics[width=\linewidth]{figs/200619_smear_200619_2018_divmean/resolution_jet1_adjusted_response_smear_0.pdf}
    \end{column}
  \end{columns}

  \vfill
  2016 has the best pre-smearing agreement

\end{frame}


\begin{frame}
  \frametitle{Fit Results}

  \vfill
  \def\arraystretch{1.5}
  \begin{center}
    \begin{tabular}{|l|c|c|}
      \hline
      Year & MC Scaling & Smearing \\
      \hline
      2016 & $1.013 \pm 0.014$ & $2.9 \pm 4.7 \%$ \\
      2017 & $1.017 \pm 0.021$ & $5.8 \pm 6.6 \%$ \\
      2018 & $0.985 \pm 0.019$ & $8.0 \pm 7.3 \%$ \\
      \hline
    \end{tabular}
  \end{center}


\end{frame}


\begin{frame}
  \frametitle{Jet smearing strategy}

  Smearing and scaling MC is done with the following algorithm:
  \begin{itemize}
  \item If a matched genjet is found:
    \begin{itemize}
    \item Find the difference between the jet $p_T$ (before scaling) and the genjet $p_T$
    \item Shift the jet $p_T$ so that the difference increases by the percentage listed
    \item Scale the resulting smeared $p_T$ and return
    \end{itemize}
  \item Else return with just the scaling
  \end{itemize}

  Uncertainty bands are determined from both scaling and smearing uncertainties
  \begin{itemize}
  \item Additional smearing shifts the jet $p_T$ away from the genjet
  \item Scaling without smearing is used as a bound for the reduced smearing variation
  \item Not scaled again
  \end{itemize}

\end{frame}


\begin{frame}
  \frametitle{Nominal Smearing and Scaling}

  \begin{columns}
    \begin{column}{0.33\linewidth}
      \centering
      \textcolor{blue}{2016} \\
      \includegraphics[width=\linewidth]{figs/200623_smear_200619_2016_divmean/mean_jet1_adjusted_response_smeared_scaled_nominal_smear_0.pdf} \\
      \includegraphics[width=\linewidth]{figs/200619_smear_200619_2016_divmean/resolution_jet1_adjusted_response_smeared_scaled_nominal_smear_0.pdf}
    \end{column}
    \begin{column}{0.34\linewidth}
      \centering
      \textcolor{blue}{2017} \\
      \includegraphics[width=\linewidth]{figs/200623_smear_200619_2017_divmean/mean_jet1_adjusted_response_smeared_scaled_nominal_smear_0.pdf} \\
      \includegraphics[width=\linewidth]{figs/200619_smear_200619_2017_divmean/resolution_jet1_adjusted_response_smeared_scaled_nominal_smear_0.pdf}
    \end{column}
    \begin{column}{0.33\linewidth}
      \centering
      \textcolor{blue}{2018} \\
      \includegraphics[width=\linewidth]{figs/200623_smear_200619_2018_divmean/mean_jet1_adjusted_response_smeared_scaled_nominal_smear_0.pdf} \\
      \includegraphics[width=\linewidth]{figs/200619_smear_200619_2018_divmean/resolution_jet1_adjusted_response_smeared_scaled_nominal_smear_0.pdf}
    \end{column}
  \end{columns}

  \vfill
  Both response mean and resolution agree better between MC and Data
  after scaling and smearing application

\end{frame}

\begin{frame}
  \frametitle{Additional Smearing}

  \begin{columns}
    \begin{column}{0.33\linewidth}
      \centering
      \textcolor{blue}{2016}
      \includegraphics[width=\linewidth]{figs/200619_smear_200619_2016_divmean/resolution_jet1_adjusted_response_smeared_scaled_up_smear_0.pdf}
    \end{column}
    \begin{column}{0.34\linewidth}
      \centering
      \textcolor{blue}{2017}
      \includegraphics[width=\linewidth]{figs/200619_smear_200619_2017_divmean/resolution_jet1_adjusted_response_smeared_scaled_up_smear_0.pdf}
    \end{column}
    \begin{column}{0.33\linewidth}
      \centering
      \textcolor{blue}{2018}
      \includegraphics[width=\linewidth]{figs/200619_smear_200619_2018_divmean/resolution_jet1_adjusted_response_smeared_scaled_up_smear_0.pdf}
    \end{column}
  \end{columns}

  \vfill
  Evidence of oversmearing gives this good closure

\end{frame}


\begin{frame}
  \frametitle{Smearing Conclusions}

  \begin{itemize}
  \item Updated smearing numbers are ready for NanoAODv7
  \item Implemented suggestions from JetMET
  \item Scaling added to help MC and Data agreement
  \item Uncertainties are smaller than the discovery analysis
  \end{itemize}

\end{frame}


\begin{frame}
  \frametitle{Di-lepton di-jet plots for VH Analysis}

  \begin{itemize}
  \item Selection of leptons is the same as before
  \item This time, there is no kinematic limits on the trailing jet
  \item $V_{p_T} > \SI{150}{GeV}$
  \item Kinematic fit is applied to improve \\
    the di-jet $p_T$ response and mass resolution
  \item Kinematic fit results are useful as classifier inputs, \\
    even for the boosted selection
  \end{itemize}

\end{frame}


\begin{frame}
  \frametitle{$p_T$ Balance}

  In the Z+HF control region, kinematic fit brings improvement to the $p_T$ balance
  between the di-jet and di-lepton systems,
  in addition to the regression
  \vfill

  \begin{columns}
    \begin{column}{0.33\linewidth}
      \centering
      \textcolor{blue}{No Regression}
      \includegraphics[width=\linewidth]{figs/200622_Zll2018_runplot-v5/Zhf_medhigh_Zll__ptBalance_noReg_.pdf}
    \end{column}
    \begin{column}{0.34\linewidth}
      \centering
      \textcolor{blue}{With Regression}
      \includegraphics[width=\linewidth]{figs/200622_Zll2018_runplot-v5/Zhf_medhigh_Zll__ptBalance_.pdf}
    \end{column}
    \begin{column}{0.33\linewidth}
      \centering
      \textcolor{blue}{With Kinematic Fit}
      \includegraphics[width=\linewidth]{figs/200622_Zll2018_runplot-v5/Zhf_medhigh_Zll__ptBalance_kinfit_.pdf}
    \end{column}
  \end{columns}

  Note: these are pre-fit plots

\end{frame}

\begin{frame}
  \frametitle{Di-jet Mass Performance -- No Additional Jet}

  \begin{columns}
    \begin{column}{0.5\linewidth}
      \centering
      \textcolor{blue}{$150 < V_{p_T} < \SI{250}{GeV}$} \\
      \includegraphics[width=0.8\linewidth]{200622_fit/fits_SR_med_Hmass_0j__.pdf} \\
      \textcolor{blue}{$\SI{250}{GeV} < V_{p_T}$} \\
      \includegraphics[width=0.8\linewidth]{figs/200624_fit_v2/fits_SR_high_Hmass_0j__.pdf}
    \end{column}
    \begin{column}{0.5\linewidth}
      \begin{center}
      \textcolor{blue}{$\SI{150}{GeV} < V_{p_T}$} \\
      \includegraphics[width=0.8\linewidth]{200622_fit/fits_SR_medhigh_Hmass_0j__.pdf} \\

      Values of $\sigma/\mu$
      {\scriptsize
      \begin{tabular}{|r|c|c|c|}
        \hline
        & Low $p_T$ & High $p_T$ & incl. \\
        \hline
        No R. & 0.137 & 0.116 & 0.132 \\
        Reg.  & 0.120 & 0.107 & 0.116 \\
        Fit   & 0.106 & 0.087 & 0.101 \\
        \hline
      \end{tabular}
      }
      \end{center}
      $\Leftarrow$ Fat Jet $\sigma/\mu = 0.116$
    \end{column}
  \end{columns}

\end{frame}

\begin{frame}
  \frametitle{Di-jet Mass Performance -- Additional Jets}


  \begin{columns}
    \begin{column}{0.5\linewidth}
      \centering
      \textcolor{blue}{$150 < V_{p_T} < \SI{250}{GeV}$} \\
      \includegraphics[width=0.8\linewidth]{200622_fit/fits_SR_med_Hmass_ge1j__.pdf} \\
      \textcolor{blue}{$\SI{250}{GeV} < V_{p_T}$} \\
      \includegraphics[width=0.8\linewidth]{200622_fit/fits_SR_high_Hmass_ge1j__.pdf}
    \end{column}
    \begin{column}{0.5\linewidth}
      \centering
      \textcolor{blue}{$\SI{150}{GeV} < V_{p_T}$} \\
      \includegraphics[width=0.8\linewidth]{200622_fit/fits_SR_medhigh_Hmass_ge1j__.pdf} \\

      Values of $\sigma/\mu$
      {\scriptsize
      \begin{tabular}{|r|c|c|c|}
        \hline
        & Low $p_T$ & High $p_T$ & incl. \\
        \hline
        No R. & 0.190 & 0.159 & 0.181 \\
        Reg.  & 0.153 & 0.138 & 0.149 \\
        Fit   & 0.162 & 0.145 & 0.157 \\
        \hline
      \end{tabular}
      }
    \end{column}
  \end{columns}

\end{frame}

\begin{frame}
  \frametitle{Conclusions}

  \begin{itemize}
  \item Smearing and uncertainties are ready for all years
  \item Systematic uncertainties are reduced from previous iterations of the analysis
  \item Kinematic fit brings large improvement to the $p_T$ balance
  \item When there are no additional jets in the event,
    this improvment translates well to improve the di-jet resolution
  \item With additional jets, the regression gives better resolution
  \end{itemize}

\end{frame}

\beginbackup

\begin{frame}
  \centering
    {\Huge \bf\sffamily Backup Slides}
\end{frame}


\begin{frame}
  \frametitle{Di-jet Mass Performance}

  \begin{columns}
    \begin{column}{0.5\linewidth}
      \centering
      \textcolor{blue}{$150 < V_{p_T} < \SI{250}{GeV}$} \\
      \includegraphics[width=0.8\linewidth]{200622_fit/fits_SR_med_Hmass__.pdf} \\
      \textcolor{blue}{$\SI{250}{GeV} < V_{p_T}$} \\
      \includegraphics[width=0.8\linewidth]{200622_fit/fits_SR_high_Hmass__.pdf}
    \end{column}
    \begin{column}{0.5\linewidth}
      \centering
      \textcolor{blue}{$\SI{150}{GeV} < V_{p_T}$} \\
      \includegraphics[width=0.8\linewidth]{200622_fit/fits_SR_medhigh_Hmass__.pdf} \\

      Values of $\sigma/\mu$
      {\scriptsize
      \begin{tabular}{|r|c|c|c|}
        \hline
        & Low $p_T$ & High $p_T$ & incl. \\
        \hline
        No R. & 0.157 & 0.133 & 0.150 \\
        Reg.  & 0.134 & 0.121 & 0.130 \\
        Fit   & 0.129 & 0.112 & 0.124 \\
        \hline
      \end{tabular}
      }
    \end{column}
  \end{columns}

\end{frame}

\begin{frame}
  \frametitle{Di-jet Mass Performance -- One Additional Jet}

  \begin{columns}
    \begin{column}{0.5\linewidth}
      \centering
      \textcolor{blue}{$150 < V_{p_T} < \SI{250}{GeV}$} \\
      \includegraphics[width=0.8\linewidth]{200624_fit/fits_SR_med_Hmass_1j__.pdf} \\
      \textcolor{blue}{$\SI{250}{GeV} < V_{p_T}$} \\
      \includegraphics[width=0.8\linewidth]{200624_fit/fits_SR_high_Hmass_1j__.pdf}
    \end{column}
    \begin{column}{0.5\linewidth}
      \centering
      \textcolor{blue}{$\SI{150}{GeV} < V_{p_T}$} \\
      \includegraphics[width=0.8\linewidth]{200624_fit/fits_SR_medhigh_Hmass_1j__.pdf} \\

      Values of $\sigma/\mu$
      {\scriptsize
      \begin{tabular}{|r|c|c|c|}
        \hline
        & Low $p_T$ & High $p_T$ & incl. \\
        \hline
        No R. & 0.179 & 0.143 & 0.169 \\
        Reg.  & 0.146 & 0.128 & 0.141 \\
        Fit   & 0.152 & 0.131 & 0.146 \\
        \hline
      \end{tabular}
      }
    \end{column}
  \end{columns}

\end{frame}


\begin{frame}
  \frametitle{Di-jet Mass Performance -- Two or More Additional Jets}

  \begin{columns}
    \begin{column}{0.5\linewidth}
      \centering
      \textcolor{blue}{$150 < V_{p_T} < \SI{250}{GeV}$} \\
      \includegraphics[width=0.8\linewidth]{200624_fit/fits_SR_med_Hmass_ge2j__.pdf} \\
      \textcolor{blue}{$\SI{250}{GeV} < V_{p_T}$} \\
      \includegraphics[width=0.8\linewidth]{200624_fit/fits_SR_high_Hmass_ge2j__.pdf}
    \end{column}
    \begin{column}{0.5\linewidth}
      \centering
      \textcolor{blue}{$\SI{150}{GeV} < V_{p_T}$} \\
      \includegraphics[width=0.8\linewidth]{200624_fit/fits_SR_medhigh_Hmass_ge2j__.pdf} \\

      Values of $\sigma/\mu$
      {\scriptsize
      \begin{tabular}{|r|c|c|c|}
        \hline
        & Low $p_T$ & High $p_T$ & incl. \\
        \hline
        No R. & 0.216 & 0.185 & 0.204 \\
        Reg.  & 0.170 & 0.156 & 0.165 \\
        Fit   & 0.187 & 0.169 & 0.181 \\
        \hline
      \end{tabular}
      }
    \end{column}
  \end{columns}

\end{frame}



\backupend

\end{document}
