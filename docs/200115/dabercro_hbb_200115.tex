\documentclass{beamer}

\author[D. Abercrombie]{
  Daniel Abercrombie, \\
  Guillelmo G\'omez-Ceballos, \\
  Dymtro Kovalskyi, \\
  Benedikt Maier, \\
  Christoph Paus
}

\title{\bf \sffamily Moving to Factorized PDFs \\ for Smearing Extraction}
\date{\today}

\usecolortheme{dove}

\usepackage[absolute,overlay]{textpos}
\usefonttheme{serif}
\usepackage{appendixnumberbeamer}
\usepackage{isotope}
\usepackage{hyperref}
\usepackage[english]{babel}
\usepackage{amsmath}
\setbeamerfont{frametitle}{size=\Large,series=\bf\sffamily}
\setbeamertemplate{frametitle}[default][center]
\usepackage{siunitx}
\usepackage{tabularx}
\usepackage{makecell}
\usepackage{comment}

\usepackage{feynmp-auto}

\setbeamertemplate{navigation symbols}{}
\usepackage{graphicx}
\usepackage{color}
\setbeamertemplate{footline}[text line]{\parbox{1.083\linewidth}{\footnotesize \hfill \insertshortauthor \hfill \insertpagenumber /\inserttotalframenumber}}
\setbeamertemplate{headline}[text line]{\parbox{1.083\linewidth}{\footnotesize \hspace{-0.083\linewidth} \textcolor{blue}{\sffamily \insertsection \hfill \insertsubsection}}}

\IfFileExists{/Users/dabercro/GradSchool/Presentations/MIT-logo.pdf}
             {\logo{\includegraphics[height=0.5cm]{/Users/dabercro/GradSchool/Presentations/MIT-logo.pdf}}}
             {\logo{\includegraphics[height=0.5cm]{/home/dabercro/MIT-logo.pdf}}}

\usepackage{changepage}

\newcommand{\beginbackup}{
  \newcounter{framenumbervorappendix}
  \setcounter{framenumbervorappendix}{\value{framenumber}}
}
\newcommand{\backupend}{
  \addtocounter{framenumbervorappendix}{-\value{framenumber}}
  \addtocounter{framenumber}{\value{framenumbervorappendix}}
}

\graphicspath{{figs/}}

\newcommand{\link}[2]{\href{#2}{\textcolor{blue}{\underline{#1}}}}
\newcommand{\clink}[2]{\link{#1}{http://t3serv001.mit.edu/~dabercro/redir/?k=#2}}}

\newcommand{\twofigs}[4]{
  \begin{columns}
    \begin{column}{0.5\linewidth}
      \centering
      \textcolor{blue}{#1} \\
      \includegraphics[width=\linewidth]{#2}
    \end{column}
    \begin{column}{0.5\linewidth}
      \centering
      \textcolor{blue}{#3} \\
      \includegraphics[width=\linewidth]{#4}
    \end{column}
  \end{columns}
}

\newcommand{\fourfigs}[8]{
  \begin{columns}
    \begin{column}{0.3\linewidth}
      \centering
      \textcolor{blue}{#1} \\
      \includegraphics[width=\linewidth]{#2} \\
      \textcolor{blue}{#3} \\
      \includegraphics[width=\linewidth]{#4}
    \end{column}
    \begin{column}{0.3\linewidth}
      \centering
      \textcolor{blue}{#5} \\
      \includegraphics[width=\linewidth]{#6} \\
      \textcolor{blue}{#7} \\
      \includegraphics[width=\linewidth]{#8}
    \end{column}
  \end{columns}
}

\newcommand{\ttbar}{\ensuremath{t\bar{t}}}
\newcommand{\bbbar}{\ensuremath{b\bar{b}}}

\begin{document}

\begin{frame}
  \titlepage
\end{frame}

\begin{frame}
  \frametitle{Introduction}

  \begin{itemize}
  \item Smearing needs to be re-derived after applying \\
    energy regression designed to recover energy \\
    lost in the weak decays of b-jets.
  \item Extracting smearing parameters as a function of $\rho$ \\
    results in large uncertainties
  \item Correlations between $\rho$ bins are not easy to propagate correctly through fit
  \item Alternative to binning would be to do \\
    higher-dimensional fit with a factorized PDF
  \item First step is to perform this fit in 2D space \\
    (jet response vs $\alpha$), and compare to binned fit results
  \end{itemize}

\end{frame}

\begin{frame}
  \frametitle{Jet Resolution Approach}

  \begin{itemize}
  \item Good detector resolution of leptons allows us to \\
    take advantage of the following process:
  \end{itemize}

  \hfill

  \begin{center}
    \begin{fmffile}{DY_bjet}
      \begin{fmfgraph*}(120, 80)
        \fmfleft{i1,i0}
        \fmfright{o2,o1,o0}
        \fmf{fermion, label=$b$}{i0,v0,v1,o2}
        \fmf{photon, label=$Z$}{v0,v0_1}
        \fmf{fermion, label=$\ell$}{o0,v0_1,o1}
        \fmf{gluon}{i1,v1}
      \end{fmfgraph*}
    \end{fmffile}
  \end{center}

  \hfill

  \begin{itemize}
  \item The jet resolution can be measured by assuming it is \\
    balanced against the $Z$ in the transverse plane
  \end{itemize}

\end{frame}


\begin{frame}
  \frametitle{NanoAOD Samples}

  Using NanoAODv5 data samples:

  \begin{itemize}
  \item \texttt{\small /DoubleMuon/Run2018A-Nano1June2019-v1/NANOAOD}
  \item \texttt{\small /DoubleMuon/Run2018B-Nano25Oct2019-v1/NANOAOD}
  \item \texttt{\small /DoubleMuon/Run2018C-Nano25Oct2019-v1/NANOAOD}
  \item \texttt{\small /DoubleMuon/Run2018D-Nano1June2019\_ver2-v1/NANOAOD}
  \item \texttt{\small /EGamma/Run2018A-Nano25Oct2019-v1/NANOAOD}
  \item \texttt{\small /EGamma/Run2018B-Nano25Oct2019-v1/NANOAOD}
  \item \texttt{\small /EGamma/Run2018C-Nano25Oct2019-v1/NANOAOD}
  \item \texttt{\small /EGamma/Run2018D-Nano25Oct2019\_ver2-v1/NANOAOD}
  \end{itemize}

\end{frame}


\begin{frame}
  \frametitle{NanoAOD Samples}

  Using NanoAODv5 MC samples:

  \begin{itemize}
  \item \texttt{\small /DYJetsToLL\_M-50\_*\_TuneCP5\_PSweights\_\\13TeV-madgraphMLM-pythia8/\\RunIIAutumn18NanoAODv5-Nano1June2019\_102X\_\\upgrade2018\_realistic\_v19-v1/NANOAODSIM}
  \item \texttt{\small /TTTo2L2Nu\_TuneCP5\_PSweights\_13TeV-powheg-pythia8/\\RunIIFall17NanoAODv5-PU2017\_12Apr2018\_Nano1June2019\_\\new\_pmx\_102X\_mc2017\_realistic\_v7-v1/NANOAODSIM}
  \end{itemize}

\end{frame}

\begin{frame}
  \frametitle{Jet Resolution Selection}

  \begin{itemize}
  \item Triggers:
    \begin{itemize}
    \item \texttt{HLT\_Mu17\_TrkIsoVVL\_Mu8\_TrkIsoVVL\_DZ\_Mass3p8}
    \item \texttt{HLT\_Mu17\_TrkIsoVVL\_Mu8\_TrkIsoVVL\_DZ\_Mass8}
    \item \texttt{HLT\_Ele115\_CaloIdVT\_GsfTrkIdT}
    \item \texttt{HLT\_Ele27\_WPTight\_Gsf}
    \item \texttt{HLT\_Ele28\_WPTight\_Gsf}
    \item \texttt{HLT\_Ele32\_WPTight\_Gsf}
    \item \texttt{HLT\_Ele35\_WPTight\_Gsf}
    \item \texttt{HLT\_Ele38\_WPTight\_Gsf}
    \item \texttt{HLT\_Ele40\_WPTight\_Gsf}
    \item \texttt{HLT\_Ele32\_WPTight\_Gsf\_L1DoubleEG}
    \end{itemize}
  \end{itemize}

\end{frame}

\begin{frame}
  \frametitle{Jet Resolution Selection}

  \begin{itemize}
  \item Two leptons satisfying:
    \begin{itemize}
    \item $p_T > \SI{20}{GeV}$
    \item $q_1 + q_2 = 0$
    \item Z Selection:
      \begin{itemize}
      \item Muons: \\
        \texttt{pfRelIso04\_all} $< 0.25$ and \\
        $\Delta xy < 0.05$ and $\Delta z < 0.2$
      \item Electrons: \\
        \texttt{mvaFall17V2Iso\_WP90} and \texttt{pfRelIso03\_all} $< 0.15$
      \end{itemize}
    \end{itemize}
  \end{itemize}

\end{frame}


\begin{frame}
  \frametitle{Binned Fit Strategy}

\end{frame}


\begin{frame}
  \frametitle{Binned Fit Uncertainties}

\end{frame}


\begin{frame}
  \frametitle{Binned Fit Results}

\end{frame}


\begin{frame}
  \frametitle{Unbinned Fit Strategy}

\end{frame}


\begin{frame}
  \frametitle{Unbinned Fit Uncertainties}

\end{frame}


\begin{frame}
  \frametitle{Unbinned Fit Results}

\end{frame}


\begin{frame}
  \frametitle{Comparison of Results}

\end{frame}


\begin{frame}
  \frametitle{Next Steps}

\end{frame}


\begin{comment}
\beginbackup

\begin{frame}
  \centering
    {\Huge \bf\sffamily Backup Slides}
\end{frame}

\begin{frame}
   \frametitle{\small 190611/plot\_time\_60000\_wide}
   \centering
   \includegraphics[width=0.6\linewidth]{190611/plot_time_60000_wide.pdf}
\end{frame}

\begin{frame}
   \frametitle{\small 190611/plot\_time\_wide}
   \centering
   \includegraphics[width=0.6\linewidth]{190611/plot_time_wide.pdf}
\end{frame}

\begin{frame}
   \frametitle{\small 190611/plot\_time\_120000\_compare}
   \centering
   \includegraphics[width=0.6\linewidth]{190611/plot_time_120000_compare.pdf}
\end{frame}

\begin{frame}
   \frametitle{\small 190611/plot\_time\_60000\_compare}
   \centering
   \includegraphics[width=0.6\linewidth]{190611/plot_time_60000_compare.pdf}
\end{frame}

\begin{frame}
   \frametitle{\small 190611/plot\_time\_80000\_compare}
   \centering
   \includegraphics[width=0.6\linewidth]{190611/plot_time_80000_compare.pdf}
\end{frame}

\begin{frame}
   \frametitle{\small 190611/plot\_time\_120000\_narrow}
   \centering
   \includegraphics[width=0.6\linewidth]{190611/plot_time_120000_narrow.pdf}
\end{frame}

\begin{frame}
   \frametitle{\small 190611/plot\_time\_40000\_narrow}
   \centering
   \includegraphics[width=0.6\linewidth]{190611/plot_time_40000_narrow.pdf}
\end{frame}

\begin{frame}
   \frametitle{\small 190611/plot\_time\_160000\_narrow}
   \centering
   \includegraphics[width=0.6\linewidth]{190611/plot_time_160000_narrow.pdf}
\end{frame}

\begin{frame}
   \frametitle{\small 190611/plot\_time\_60000\_narrow}
   \centering
   \includegraphics[width=0.6\linewidth]{190611/plot_time_60000_narrow.pdf}
\end{frame}

\begin{frame}
   \frametitle{\small 190611/plot\_time\_80000\_narrow}
   \centering
   \includegraphics[width=0.6\linewidth]{190611/plot_time_80000_narrow.pdf}
\end{frame}

\begin{frame}
   \frametitle{\small 190611/plot\_time\_narrow}
   \centering
   \includegraphics[width=0.6\linewidth]{190611/plot_time_narrow.pdf}
\end{frame}



\backupend
\end{comment}

\end{document}
