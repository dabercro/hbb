\documentclass{beamer}

\author[D. Abercrombie]{
  Daniel Abercrombie, \\
  Guillelmo G\'omez-Ceballos, \\
  Dymtro Kovalskyi, \\
  Benedikt Maier, \\
  Christoph Paus
}

\title{\bf \sffamily Moving to Factorized PDFs \\ for Smearing Extraction}
\date{\today}

\usecolortheme{dove}

\usepackage[absolute,overlay]{textpos}
\usefonttheme{serif}
\usepackage{appendixnumberbeamer}
\usepackage{isotope}
\usepackage{hyperref}
\usepackage[english]{babel}
\usepackage{amsmath}
\setbeamerfont{frametitle}{size=\Large,series=\bf\sffamily}
\setbeamertemplate{frametitle}[default][center]
\usepackage{siunitx}
\usepackage{tabularx}
\usepackage{makecell}
\usepackage{comment}

\usepackage{feynmp-auto}

\setbeamertemplate{navigation symbols}{}
\usepackage{graphicx}
\usepackage{color}
\setbeamertemplate{footline}[text line]{\parbox{1.083\linewidth}{\footnotesize \hfill \insertshortauthor \hfill \insertpagenumber /\inserttotalframenumber}}
\setbeamertemplate{headline}[text line]{\parbox{1.083\linewidth}{\footnotesize \hspace{-0.083\linewidth} \textcolor{blue}{\sffamily \insertsection \hfill \insertsubsection}}}

\IfFileExists{/Users/dabercro/GradSchool/Presentations/MIT-logo.pdf}
             {\logo{\includegraphics[height=0.5cm]{/Users/dabercro/GradSchool/Presentations/MIT-logo.pdf}}}
             {\logo{\includegraphics[height=0.5cm]{/home/dabercro/MIT-logo.pdf}}}

\usepackage{changepage}

\newcommand{\beginbackup}{
  \newcounter{framenumbervorappendix}
  \setcounter{framenumbervorappendix}{\value{framenumber}}
}
\newcommand{\backupend}{
  \addtocounter{framenumbervorappendix}{-\value{framenumber}}
  \addtocounter{framenumber}{\value{framenumbervorappendix}}
}

\graphicspath{{figs/}}

\newcommand{\link}[2]{\href{#2}{\textcolor{blue}{\underline{#1}}}}
\newcommand{\clink}[2]{\link{#1}{http://t3serv001.mit.edu/~dabercro/redir/?k=#2}}}

\newcommand{\twofigs}[4]{
  \begin{columns}
    \begin{column}{0.5\linewidth}
      \centering
      \textcolor{blue}{#1} \\
      \includegraphics[width=\linewidth]{#2}
    \end{column}
    \begin{column}{0.5\linewidth}
      \centering
      \textcolor{blue}{#3} \\
      \includegraphics[width=\linewidth]{#4}
    \end{column}
  \end{columns}
}

\newcommand{\fourfigs}[8]{
  \begin{columns}
    \begin{column}{0.3\linewidth}
      \centering
      \textcolor{blue}{#1} \\
      \includegraphics[width=\linewidth]{#2} \\
      \textcolor{blue}{#3} \\
      \includegraphics[width=\linewidth]{#4}
    \end{column}
    \begin{column}{0.3\linewidth}
      \centering
      \textcolor{blue}{#5} \\
      \includegraphics[width=\linewidth]{#6} \\
      \textcolor{blue}{#7} \\
      \includegraphics[width=\linewidth]{#8}
    \end{column}
  \end{columns}
}

\newcommand{\ttbar}{\ensuremath{t\bar{t}}}
\newcommand{\bbbar}{\ensuremath{b\bar{b}}}

\begin{document}

\begin{frame}
  \titlepage
\end{frame}

\begin{frame}
  \frametitle{Introduction}

  \begin{itemize}
  \item Smearing needs to be re-derived after applying \\
    energy regression designed to recover energy \\
    lost in the weak decays of b-jets.
  \item Extracting smearing parameters as a function of $\rho$ \\
    results in large uncertainties
  \item Correlations between $\rho$ bins are not easy to propagate correctly through fit
  \item Alternative to binning would be to do \\
    higher-dimensional fit with a factorized PDF
  \item First step is to perform this fit in 2D space \\
    (jet response vs $\alpha$), and compare to binned fit results
  \end{itemize}

\end{frame}

\begin{frame}
  \frametitle{Jet Resolution Approach}

  \begin{itemize}
  \item Good detector resolution of leptons allows us to \\
    take advantage of the following process:
  \end{itemize}

  \hfill

  \begin{center}
    \begin{fmffile}{DY_bjet}
      \begin{fmfgraph*}(120, 80)
        \fmfleft{i1,i0}
        \fmfright{o2,o1,o0}
        \fmf{fermion, label=$b$}{i0,v0,v1,o2}
        \fmf{photon, label=$Z$}{v0,v0_1}
        \fmf{fermion, label=$\ell$}{o0,v0_1,o1}
        \fmf{gluon}{i1,v1}
      \end{fmfgraph*}
    \end{fmffile}
  \end{center}

  \hfill

  \begin{itemize}
  \item The jet resolution can be measured by assuming it is \\
    balanced against the $Z$ in the transverse plane
  \end{itemize}

\end{frame}


\begin{frame}
  \frametitle{Jet Resolution Selection}

\end{frame}


\begin{frame}
  \frametitle{Binned Fit Strategy}

\end{frame}


\begin{frame}
  \frametitle{Binned Fit Uncertainties}

\end{frame}


\begin{frame}
  \frametitle{Binned Fit Results}

\end{frame}


\begin{frame}
  \frametitle{Unbinned Fit Strategy}

\end{frame}


\begin{frame}
  \frametitle{Unbinned Fit Uncertainties}

\end{frame}


\begin{frame}
  \frametitle{Unbinned Fit Results}

\end{frame}


\begin{frame}
  \frametitle{Comparison of Results}

\end{frame}


\begin{frame}
  \frametitle{Next Steps}

\end{frame}


\begin{comment}
\beginbackup

\begin{frame}
  \centering
    {\Huge \bf\sffamily Backup Slides}
\end{frame}

\begin{frame}
   \frametitle{\small 190813\_testreg\_013/Jet\_puppi\_charged\_ptfrac}
   \centering
   \includegraphics[width=0.6\linewidth]{190813_testreg_013/Jet_puppi_charged_ptfrac.pdf}
\end{frame}

\begin{frame}
   \frametitle{\small 190813\_testreg\_013/Jet\_puppi\_neutral\_ptfrac}
   \centering
   \includegraphics[width=0.6\linewidth]{190813_testreg_013/Jet_puppi_neutral_ptfrac.pdf}
\end{frame}

\begin{frame}
   \frametitle{\small 190813\_testreg\_013/Jet\_puppi\_charged\_pu\_ptfrac}
   \centering
   \includegraphics[width=0.6\linewidth]{190813_testreg_013/Jet_puppi_charged_pu_ptfrac.pdf}
\end{frame}

\begin{frame}
   \frametitle{\small 190813\_testreg\_013/Jet\_puppi\_neutral\_pu\_ptfrac}
   \centering
   \includegraphics[width=0.6\linewidth]{190813_testreg_013/Jet_puppi_neutral_pu_ptfrac.pdf}
\end{frame}

\begin{frame}
   \frametitle{\small 190813\_bukin/signal\_hbb\_m\_190725\_lstm\_pf}
   \centering
   \includegraphics[width=0.6\linewidth]{190813_bukin/signal_hbb_m_190725_lstm_pf.pdf}
\end{frame}

\begin{frame}
   \frametitle{\small 190813\_bukin/signal\_hbb\_m}
   \centering
   \includegraphics[width=0.6\linewidth]{190813_bukin/signal_hbb_m.pdf}
\end{frame}

\begin{frame}
   \frametitle{\small 190813\_bukin/signal\_hbb\_m\_190723\_origin}
   \centering
   \includegraphics[width=0.6\linewidth]{190813_bukin/signal_hbb_m_190723_origin.pdf}
\end{frame}

\begin{frame}
   \frametitle{\small 190813\_bukin/signal\_hbb\_m\_190724\_puppi\_direction}
   \centering
   \includegraphics[width=0.6\linewidth]{190813_bukin/signal_hbb_m_190724_puppi_direction.pdf}
\end{frame}

\begin{frame}
   \frametitle{\small 190813\_ratio/signal\_hbb\_m\_190724\_puppi\_direction}
   \centering
   \includegraphics[width=0.6\linewidth]{190813_ratio/signal_hbb_m_190724_puppi_direction.pdf}
\end{frame}

\begin{frame}
   \frametitle{\small 190813\_bukin/signal\_hbb\_m\_190724\_origin\_direction}
   \centering
   \includegraphics[width=0.6\linewidth]{190813_bukin/signal_hbb_m_190724_origin_direction.pdf}
\end{frame}

\begin{frame}
   \frametitle{\small 190813\_ratio/signal\_hbb\_m\_190724\_origin\_direction}
   \centering
   \includegraphics[width=0.6\linewidth]{190813_ratio/signal_hbb_m_190724_origin_direction.pdf}
\end{frame}

\begin{frame}
   \frametitle{\small 190813\_testreg\_013/Jet\_pf\_0\_transformed\_px}
   \centering
   \includegraphics[width=0.6\linewidth]{190813_testreg_013/Jet_pf_0_transformed_px.pdf}
\end{frame}

\begin{frame}
   \frametitle{\small 190813\_testreg\_013/Jet\_pf\_1\_transformed\_px}
   \centering
   \includegraphics[width=0.6\linewidth]{190813_testreg_013/Jet_pf_1_transformed_px.pdf}
\end{frame}

\begin{frame}
   \frametitle{\small 190813\_testreg\_013/Jet\_pf\_2\_transformed\_px}
   \centering
   \includegraphics[width=0.6\linewidth]{190813_testreg_013/Jet_pf_2_transformed_px.pdf}
\end{frame}

\begin{frame}
   \frametitle{\small 190813\_testreg\_013/Jet\_pf\_3\_transformed\_px}
   \centering
   \includegraphics[width=0.6\linewidth]{190813_testreg_013/Jet_pf_3_transformed_px.pdf}
\end{frame}

\begin{frame}
   \frametitle{\small 190813\_testreg\_013/Jet\_pf\_0\_transformed\_py}
   \centering
   \includegraphics[width=0.6\linewidth]{190813_testreg_013/Jet_pf_0_transformed_py.pdf}
\end{frame}

\begin{frame}
   \frametitle{\small 190813\_testreg\_013/Jet\_pf\_1\_transformed\_py}
   \centering
   \includegraphics[width=0.6\linewidth]{190813_testreg_013/Jet_pf_1_transformed_py.pdf}
\end{frame}

\begin{frame}
   \frametitle{\small 190813\_testreg\_013/Jet\_pf\_2\_transformed\_py}
   \centering
   \includegraphics[width=0.6\linewidth]{190813_testreg_013/Jet_pf_2_transformed_py.pdf}
\end{frame}

\begin{frame}
   \frametitle{\small 190813\_testreg\_013/Jet\_pf\_3\_transformed\_py}
   \centering
   \includegraphics[width=0.6\linewidth]{190813_testreg_013/Jet_pf_3_transformed_py.pdf}
\end{frame}

\begin{frame}
   \frametitle{\small 190813\_testreg\_013/Jet\_pf\_0\_transformed\_pz}
   \centering
   \includegraphics[width=0.6\linewidth]{190813_testreg_013/Jet_pf_0_transformed_pz.pdf}
\end{frame}

\begin{frame}
   \frametitle{\small 190813\_testreg\_013/Jet\_pf\_1\_transformed\_pz}
   \centering
   \includegraphics[width=0.6\linewidth]{190813_testreg_013/Jet_pf_1_transformed_pz.pdf}
\end{frame}

\begin{frame}
   \frametitle{\small 190813\_testreg\_013/Jet\_pf\_2\_transformed\_pz}
   \centering
   \includegraphics[width=0.6\linewidth]{190813_testreg_013/Jet_pf_2_transformed_pz.pdf}
\end{frame}

\begin{frame}
   \frametitle{\small 190813\_testreg\_013/Jet\_pf\_3\_transformed\_pz}
   \centering
   \includegraphics[width=0.6\linewidth]{190813_testreg_013/Jet_pf_3_transformed_pz.pdf}
\end{frame}



\backupend
\end{comment}

\end{document}
