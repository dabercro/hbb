\documentclass{beamer}

\author[D. Abercrombie]{
  Daniel Abercrombie \\
  for the VHbb Group
}

\title{\bf \sffamily B-jet Energy Smearing for the VHbb Analysis}
\date{4 March 2020}

\usecolortheme{dove}

\usepackage[absolute,overlay]{textpos}
\usefonttheme{serif}
\usepackage{appendixnumberbeamer}
\usepackage{isotope}
\usepackage{hyperref}
\usepackage[english]{babel}
\usepackage{amsmath}
\setbeamerfont{frametitle}{size=\Large,series=\bf\sffamily}
\setbeamertemplate{frametitle}[default][center]
\usepackage{siunitx}
\usepackage{tabularx}
\usepackage{makecell}
\usepackage{comment}

\usepackage{feynmp-auto}

\setbeamertemplate{navigation symbols}{}
\usepackage{graphicx}
\usepackage{color}
\setbeamertemplate{footline}[text line]{\parbox{1.083\linewidth}{\footnotesize \hfill \insertshortauthor \hfill \insertpagenumber /\inserttotalframenumber}}
\setbeamertemplate{headline}[text line]{\parbox{1.083\linewidth}{\footnotesize \hspace{-0.083\linewidth} \textcolor{blue}{\sffamily \insertsection \hfill \insertsubsection}}}

\IfFileExists{/Users/dabercro/GradSchool/Presentations/MIT-logo.pdf}
             {\logo{\includegraphics[height=0.5cm]{/Users/dabercro/GradSchool/Presentations/MIT-logo.pdf}}}
             {\logo{\includegraphics[height=0.5cm]{/home/dabercro/MIT-logo.pdf}}}

\usepackage{changepage}

\newcommand{\beginbackup}{
  \newcounter{framenumbervorappendix}
  \setcounter{framenumbervorappendix}{\value{framenumber}}
}
\newcommand{\backupend}{
  \addtocounter{framenumbervorappendix}{-\value{framenumber}}
  \addtocounter{framenumber}{\value{framenumbervorappendix}}
}

\graphicspath{{figs/}}

\newcommand{\link}[2]{\href{#2}{\textcolor{blue}{\underline{#1}}}}
\newcommand{\clink}[2]{\link{#1}{http://t3serv001.mit.edu/~dabercro/redir/?k=#2}}}

\newcommand{\twofigs}[4]{
  \begin{columns}
    \begin{column}{0.5\linewidth}
      \centering
      \textcolor{blue}{#1} \\
      \includegraphics[width=\linewidth]{#2}
    \end{column}
    \begin{column}{0.5\linewidth}
      \centering
      \textcolor{blue}{#3} \\
      \includegraphics[width=\linewidth]{#4}
    \end{column}
  \end{columns}
}

\newcommand{\fourfigs}[8]{
  \begin{columns}
    \begin{column}{0.3\linewidth}
      \centering
      \textcolor{blue}{#1} \\
      \includegraphics[width=\linewidth]{#2} \\
      \textcolor{blue}{#3} \\
      \includegraphics[width=\linewidth]{#4}
    \end{column}
    \begin{column}{0.3\linewidth}
      \centering
      \textcolor{blue}{#5} \\
      \includegraphics[width=\linewidth]{#6} \\
      \textcolor{blue}{#7} \\
      \includegraphics[width=\linewidth]{#8}
    \end{column}
  \end{columns}
}

\newcommand{\ttbar}{\ensuremath{t\bar{t}}}
\newcommand{\bbbar}{\ensuremath{b\bar{b}}}

\begin{document}


\begin{frame}
  \titlepage
\end{frame}


\begin{frame}
  \frametitle{Introduction}

  \begin{itemize}
  \item Smearing needs to be re-derived after applying
    energy regression designed to recover energy
    lost in the weak decays of b-jets.
  \item Smearing method for discovery analysis does not close for all phase space
  \item Extracting smearing parameters as a function of $\rho$
    improves closure, but results in large fit uncertainties
  \item Alternative to binning would be to do
    an unbinned fit with a factorized PDF
  \item First step is to perform this fit in 2D space
    (jet response vs $\alpha$), and compare to binned fit results
  \end{itemize}

\end{frame}


\begin{frame}
  \frametitle{Jet Resolution Approach}

  \begin{itemize}
  \item Good detector resolution of leptons allows us to \\
    take advantage of the following process:
  \end{itemize}

  \hfill

  \begin{center}
    \begin{fmffile}{DY_bjet}
      \begin{fmfgraph*}(200, 120)
        \fmfleft{i1,i0}
        \fmfright{o3,o2,o1,o0}
        \fmf{fermion, label=$q$}{i0,v0,v1,i1}
        \fmf{photon, label=$Z$}{v0,v0_1}
        \fmf{fermion, label=$\ell$}{o0,v0_1,o1}
        \fmf{gluon}{v1,v1_1}
        \fmf{fermion, label=$b$}{o2,v1_1,o3}
      \end{fmfgraph*}
    \end{fmffile}
  \end{center}

  \hfill

  \begin{itemize}
  \item The jet resolution can be measured by assuming it is \\
    balanced against the $Z$ in the transverse plane
  \end{itemize}

\end{frame}


\begin{frame}
  \frametitle{Samples}

  Using NanoAODv6 setup with updated B-jet regression weights:

  \begin{itemize}
  \item \texttt{\small /DoubleMuon/Run2018A-17Sep2018-v2/MINIAOD}
  \item \texttt{\small /DoubleMuon/Run2018B-17Sep2018-v1/MINIAOD}
  \item \texttt{\small /DoubleMuon/Run2018C-17Sep2018-v1/MINIAOD}
  \item \texttt{\small /DoubleMuon/Run2018D-PromptReco-v2/MINIAOD}
  \item \texttt{\small /EGamma/Run2018A-17Sep2018-v2/MINIAOD}
  \item \texttt{\small /EGamma/Run2018B-17Sep2018-v1/MINIAOD}
  \item \texttt{\small /EGamma/Run2018C-17Sep2018-v1/MINIAOD}
  \item \texttt{\small /EGamma/Run2018D-22Jan2019-v2/MINIAOD}
  \end{itemize}

\end{frame}


\begin{frame}
  \frametitle{NanoAOD Samples}

  Using NanoAODv6 setup with updated B-jet regression weights:

  \begin{itemize}
  \item \texttt{\small /DYJetsToLL\_M-50\_HT-*\_TuneCP5\_PSweights\_\\13TeV-madgraphMLM-pythia8/\\RunIIAutumn18MiniAOD-102X\_upgrade2018\_realistic\_v15-v?\footnote{400to600 is v3, all others are v2}/\\MINIAODSIM}
  \item \texttt{\small /TTTo2L2Nu\_TuneCP5\_13TeV-powheg-pythia8/\\RunIIAutumn18MiniAOD-102X\_upgrade2018\_realistic\_v15-v1/\\MINIAODSIM}
  \end{itemize}

\end{frame}


\begin{frame}
  \frametitle{Selection -- Triggers}

  In data only, there are no trigger efficiencies applied to MC

  \begin{itemize}
  \item Triggers:
    \begin{itemize}
    \item \texttt{HLT\_Mu17\_TrkIsoVVL\_Mu8\_TrkIsoVVL\_DZ\_Mass3p8}
    \item \texttt{HLT\_Mu17\_TrkIsoVVL\_Mu8\_TrkIsoVVL\_DZ\_Mass8}
    \item \texttt{HLT\_Ele115\_CaloIdVT\_GsfTrkIdT}
    \item \texttt{HLT\_Ele27\_WPTight\_Gsf}
    \item \texttt{HLT\_Ele28\_WPTight\_Gsf}
    \item \texttt{HLT\_Ele32\_WPTight\_Gsf}
    \item \texttt{HLT\_Ele35\_WPTight\_Gsf}
    \item \texttt{HLT\_Ele38\_WPTight\_Gsf}
    \item \texttt{HLT\_Ele40\_WPTight\_Gsf}
    \item \texttt{HLT\_Ele32\_WPTight\_Gsf\_L1DoubleEG}
    \end{itemize}
  \end{itemize}

\end{frame}


\begin{frame}
  \frametitle{Selection -- Leptons}

  \begin{itemize}
  \item Two leptons satisfying:
    \begin{itemize}
    \item $p_T > \SI{20}{GeV}$
    \item $q_1 + q_2 = 0$
    \item Same flavor
    \item Z Selection:
      \begin{itemize}
      \item Muons: \\
        \texttt{pfRelIso04\_all} $< 0.25$ and \\
        $\Delta xy < 0.05$ and $\Delta z < 0.2$
      \item Electrons: \\
        \texttt{mvaFall17V2Iso\_WP90} and \texttt{pfRelIso03\_all} $< 0.15$
      \end{itemize}
    \end{itemize}
  \item Any third lepton (either muon or electron) causes event to be thrown out
  \item $\SI{71}{GeV} < m_{\ell\ell} < \SI{111}{GeV}$
  \item $p_{T,\ell\ell} > \SI{100}{GeV}$
  \end{itemize}

\end{frame}


\begin{frame}
  \frametitle{Selection -- Jets}

  Selecting events with one jet is hard due to jet acceptance,
  and would result in few events

  \vfill

  \begin{itemize}
  \item Exactly two jets satisfying:
    \begin{itemize}
    \item Does not overlap with any leptons satisfying Z Selection criteria
    \item $p_T > \SI{20}{GeV}$ after regression
    \item \texttt{jetId} $> 0$ and \texttt{puId} $> 0$
    \end{itemize}
  \item Leading jet satisfies:
    \begin{itemize}
    \item $|\eta| < 2.4$
    \item $\Delta\phi(j, \ell\ell) > 2.8$
    \item \texttt{btagDeepB} $> 0.8$
    \end{itemize}
  \item Trailing jet statisfies: $\alpha = \frac{p_{T, j}}{p_{T, \ell\ell}} < 0.3$
  \end{itemize}

\end{frame}


\begin{frame}
  \frametitle{MC Weights}

  \begin{itemize}
  \item Reweight by \texttt{genWeight * btagWeight\_DeepCSVB}
  \item Cross sections:
    \vfill
    \begin{tabular}{|l|S|}
      \hline
      Sample & \mathrm{Cross Section [pb]} \\
      \hline
      \texttt{DYJetsToLL\_M-50\_HT-70to100\_*} & 143.0 \\
      \texttt{DYJetsToLL\_M-50\_HT-100to200\_*} & 147.4 \\
      \texttt{DYJetsToLL\_M-50\_HT-200to400\_*} & 40.99 \\
      \texttt{DYJetsToLL\_M-50\_HT-400to600\_*} & 5.678 \\
      \texttt{DYJetsToLL\_M-50\_HT-600to800\_*} & 1.367 \\
      \texttt{DYJetsToLL\_M-50\_HT-800to1200\_*} & 0.6304 \\
      \texttt{DYJetsToLL\_M-50\_HT-1200to2500\_*} & 0.1514 \\
      \texttt{DYJetsToLL\_M-50\_HT-2500toInf\_*} & 0.003565 \\
      \texttt{TTTo2L2Nu\_* (powheg)} & 88.288 \\
      \hline
    \end{tabular}
    \vfill
    DY are MadGraph samples
  \end{itemize}

\end{frame}


\begin{frame}
  \frametitle{Binned Fit Strategy}

  \begin{itemize}
  \item We use events with a second jet, and extrapolate down to where $\alpha = 0$
  \item This should give us the behavior of the one jet diagram,
    without needing to deal with few events or jet acceptance.
  \item We use four bins of $\alpha$, each with roughly equal data yields:
  \end{itemize}
  \begin{columns}
    \begin{column}{0.5\linewidth}
      \includegraphics[width=\linewidth]{200303_alpha_lines_v2/smearplot_alpha.pdf}
    \end{column}
    \begin{column}{0.5\linewidth}
      \[
      \alpha
      \begin{cases}
        0 < \alpha < 0.155 \\
        0.155 \le \alpha < 0.185 \\
        0.185 \le \alpha < 0.23 \\
        0.23 \le \alpha < 0.3
      \end{cases}
      \]
    \end{column}
  \end{columns}

\end{frame}


\begin{frame}
  \frametitle{Binned Fit Strategy}

  \begin{columns}
    \begin{column}{0.3\linewidth}
      \centering
      \includegraphics[width=\linewidth]{200303_nbjets_noenv/smearplot_1_jet1_response.pdf} \\
      \includegraphics[width=\linewidth]{200303_nbjets_noenv/smearplot_3_jet1_response.pdf}
    \end{column}
    \begin{column}{0.3\linewidth}
      \centering
      \includegraphics[width=\linewidth]{200303_nbjets_noenv/smearplot_2_jet1_response.pdf} \\
      \includegraphics[width=\linewidth]{200303_nbjets_noenv/smearplot_4_jet1_response.pdf}
    \end{column}
  \end{columns}

  \vfill
  We can see the MC resolution is better than data in all bins

\end{frame}


\begin{frame}
  \frametitle{Binned Fit Uncertainties}

  \begin{itemize}
  \item Standard deviation of each histogram has some uncertainty \\
    that can be propagated through the fit
  \item We add the the envelopes from all parton shower and \\
    most\footnote{Not using scale weights where renormalization and refactorization are varied in opposite directions} scale (renormalization/refactorization) weights stored in NanoAOD
  \end{itemize}

  \begin{columns}
    \begin{column}{0.5\linewidth}
      \centering
      \textcolor{blue}{Statistical Uncertainties} \\
      \includegraphics[width=0.8\linewidth]{200303_nbjets_noenv/smearplot_2_jet1_response.pdf}
    \end{column}
    \begin{column}{0.5\linewidth}
      \centering
      \textcolor{blue}{Additional Uncertainties} \\
      \includegraphics[width=0.8\linewidth]{200303_nbjets/smearplot_2_jet1_response.pdf}
    \end{column}
  \end{columns}

\end{frame}


\begin{frame}
  \frametitle{Binned Fit Formula}

  \[
  f(\alpha) = (m \times \alpha) \oplus b \times (1 + c_k \times \alpha)
  \]

  \begin{itemize}
  \item $c_k$ is fixed by a linear fit to the MC's intrinsic jet resolution ($p_{T, reco}/p_{T, gen}$) over $\alpha$ as $c_k = m_0/q_0$
  \item Smearing is done by scaling difference between $p_{T,reco}$ and $p_{T,gen}$ by $b_{data}/b_{MC}$
  \end{itemize}

  \centering
  \includegraphics[width=0.6\linewidth]{200303_smear_200303_nbjets/resolution_jet1_response_smear_0.pdf}

\end{frame}


\begin{frame}
  \frametitle{Binned Fit Results}

  Uncertainties calculated with:

  \[
  \delta = \sqrt{\left(\frac{\delta_{data}}{c_{MC}}\right)^2 +
    \left(\frac{c_{data}\delta_{MC}}{c_{MC}^2}\right)^2}
  \]

  \vfill

  \begin{center}
    Generator and reco difference to be scaled by: $3.2 \pm 6.6 \%$
  \end{center}

  \vfill

  We set the floor for the smearing to 0\%,
  so the down variation of smearing is just no smearing

\end{frame}


\begin{frame}
  \frametitle{Unbinned Fit Strategy}

  \begin{itemize}
  \item Same events, variables, cuts, and weights as before
  \item Fit 2D p.d.f. over jet response (labelled $x$) and $\alpha$
  \[
  f(x, \alpha) =
  G(x, \mu(\alpha), \sigma(\alpha)|\alpha)
  \times
  F(\alpha, \theta)
  \]
  \item $G$ is a Gaussian distribution
  \item $\mu$ as a linear function
  \item $F$ is the sum of a Landau and Gaussian distribution with fixed parameters
  \item $\sigma$ is of the same form as the binned fit,
    with $c_k$ being derived from a fit to
    intrinsic resolution with a linear $\sigma(\alpha)$:
  \[
  \sigma(\alpha) = (m \times \alpha) \oplus c \times (1 + c_k \times \alpha)
  \]
  \end{itemize}

\end{frame}


\begin{frame}
  \frametitle{Unbinned Fit -- Response}

  \begin{columns}
    \begin{column}{0.5\linewidth}
      \centering
      MC \\
      \includegraphics[width=\linewidth]{200303_cb_roofit/sum_xsec_weight_jet1_response_mc_slice0.pdf} \\
      Intrinsic \\
      \includegraphics[width=\linewidth]{200303_cb_roofit/sum_xsec_weight_jet1_response_intrinsic_gen_slice0.pdf}
    \end{column}
    \begin{column}{0.5\linewidth}
      \centering
      Data \\
      \includegraphics[width=\linewidth]{200303_cb_roofit/sum_xsec_weight_jet1_response_data_slice0.pdf} \\
      Fits look good
    \end{column}
  \end{columns}

\end{frame}


\begin{frame}
  \frametitle{Unbinned Fit -- $\alpha$}

  \begin{columns}
    \begin{column}{0.5\linewidth}
      \centering
      MC \\
      \includegraphics[width=\linewidth]{200303_cb_roofit/sum_xsec_weight_alpha_mc_slice0.pdf} \\
      Intrinsic \\
      \includegraphics[width=\linewidth]{200303_cb_roofit/sum_xsec_weight_alpha_gen_slice0.pdf}
    \end{column}
    \begin{column}{0.5\linewidth}
      \centering
      Data \\
      \includegraphics[width=\linewidth]{200303_cb_roofit/sum_xsec_weight_alpha_data_slice0.pdf} \\
      Fits look good in other direction too
    \end{column}
  \end{columns}


\end{frame}


\begin{frame}
  \frametitle{Unbinned Fit Uncertainties}

  \begin{itemize}
  \item Cannot make histogram envelope for uncertainties, \\
    so made separate fits for each weight
  \item Half of the difference between maximum and minimum smearing values for each
    Parton Shower and Scale variations are added in quadrature to the uncertainty
  \end{itemize}

  \vfill

  \centering
  Generator and reco difference to be scaled by: $1.4 \pm 5.3 (\mathrm{stat}) \pm 1.7 (\mathrm{scale}) \pm 1.3 (\mathrm{PS}) \%$

\end{frame}


\begin{frame}
  \frametitle{Comparison of Results}

  \begin{itemize}
  \item Additional smearing does cover closure for $\alpha = 0$
  \end{itemize}

  \vfill

  \fourfigs{Nominal Binned}
           {200303_smear_200303_cb/resolution_jet1_response_single_scale_nominal_smear_0.pdf}
           {Up Var. Binned}
           {200303_smear_200303_cb/resolution_jet1_response_single_scale_up_smear_0.pdf}
           {Nominal Unbinned}
           {200303_smear_200303_cb/resolution_jet1_response_unbinned_scale_nominal_smear_0.pdf}
           {Up Var. Unbinned}
           {200303_smear_200303_cb/resolution_jet1_response_unbinned_scale_up_smear_0.pdf}

\end{frame}


\begin{frame}
  \frametitle{Dependence on $\rho$}

  \begin{itemize}
  \item Smearing is needed more at low $\rho$ values    
  \end{itemize}

  \vfill

  \twofigs{}
          {200303_rho_200303_cb/resolution_jet1_response_rho_0.pdf}
          {}
          {200303_rho_200303_cb/resolution_jet1_response_rho_1.pdf}

\end{frame}


\begin{frame}
  \frametitle{$\rho$ Bin Selection}

  \begin{itemize}
  \item Too many $\rho$ bins causes large statisticaly uncertainties
  \item Too few still gives large fit uncertainties
  \end{itemize}

  \vfill

  \twofigs{}
          {200224_rho_200224_3rho_nb/scale_fit.pdf}
          {}
          {200303_rho_200303_cb/scale_fit.pdf}

\end{frame}


\begin{frame}
  \frametitle{Results Using Two Bins}

  \begin{itemize}
  \item Nominal agreement is good
  \item Up variation can likely be reduced
  \end{itemize}

  \twofigs{Nominal Smearing}
          {200303_smear_200303_cb/resolution_jet1_response_scale_nominal_smear_0.pdf}
          {Additional Smearing}
          {200303_smear_200303_cb/resolution_jet1_response_scale_up_smear_0.pdf}

\end{frame}


\begin{frame}
  \frametitle{Current Work: Unbinned Fit with $\rho$}

  \begin{itemize}
  \item Same events, variables, cuts, and weights as before
  \item Fit 2D p.d.f. over jet response (labelled $x$) and $\alpha$
  \[
  f(x, \alpha, \rho) =
  G(x, \mu(\alpha), \sigma(\alpha, \rho)|\alpha, \rho)
  \times
  F(\alpha, \theta)
  \times
  H(\rho, \phi)
  \]
  \item $G$ is a Gaussian distribution
  \item $\mu$ as a linear function
  \item $F$ is the sum of a Landau and Gaussian distribution with fixed parameters
  \item $\sigma$ now has $c(\rho)$ which is a first order polynominal in $\rho$
  \[
  \sigma(\alpha, \rho) = (m \times \alpha) \oplus c(\rho) \times (1 + c_k \times \alpha)
  \]
  \end{itemize}

\end{frame}


\begin{frame}
  \frametitle{Need Shape for $\rho$}

  \begin{itemize}
  \item Working on shape for $\rho$
  \item Suspect the fit uncertainties will be dramatically reduced
    once we have a good p.d.f.
  \end{itemize}

  \twofigs{Data}
          {200303_roofit/sum_xsec_weight_rhoAll_data.pdf}
          {MC}
          {200303_roofit/sum_xsec_weight_rhoAll_mc.pdf}

\end{frame}


\begin{frame}
  \frametitle{Preliminary Results}

  \twofigs{Nominal Smearing}
          {200304_smear_200304_3d/resolution_jet1_response_unbinned_3d_nominal_smear_0.pdf}
          {Additional Smearing}
          {200304_smear_200304_3d/resolution_jet1_response_unbinned_3d_up_smear_0.pdf}

  \begin{itemize}
  \item Even without an optimal fit, already have uncertainties similar to binned fit
  \item Closure with nominal smearing is very good
  \end{itemize}

\end{frame}


\begin{frame}
  \frametitle{Conclusions}

  \begin{itemize}
  \item We have measured the smearing needed for the updated b-jet regression
  \item The method has also been implemented in an unbinned fit
  \item These two measurements agree well
  \item The fit has been extended into the $\rho$ dimension to account for other causes of smearing
  \item Results are promising, but have large fit uncertainties
  \item Currently implementing fit with $\rho$ using an unbinned fit
  \end{itemize}

\end{frame}


\begin{comment}
\beginbackup

\begin{frame}
  \centering
    {\Huge \bf\sffamily Backup Slides}
\end{frame}

\begin{frame}
   \frametitle{\small 190611/plot\_time\_60000\_wide}
   \centering
   \includegraphics[width=0.6\linewidth]{190611/plot_time_60000_wide.pdf}
\end{frame}

\begin{frame}
   \frametitle{\small 190611/plot\_time\_wide}
   \centering
   \includegraphics[width=0.6\linewidth]{190611/plot_time_wide.pdf}
\end{frame}

\begin{frame}
   \frametitle{\small 190611/plot\_time\_120000\_compare}
   \centering
   \includegraphics[width=0.6\linewidth]{190611/plot_time_120000_compare.pdf}
\end{frame}

\begin{frame}
   \frametitle{\small 190611/plot\_time\_60000\_compare}
   \centering
   \includegraphics[width=0.6\linewidth]{190611/plot_time_60000_compare.pdf}
\end{frame}

\begin{frame}
   \frametitle{\small 190611/plot\_time\_80000\_compare}
   \centering
   \includegraphics[width=0.6\linewidth]{190611/plot_time_80000_compare.pdf}
\end{frame}

\begin{frame}
   \frametitle{\small 190611/plot\_time\_120000\_narrow}
   \centering
   \includegraphics[width=0.6\linewidth]{190611/plot_time_120000_narrow.pdf}
\end{frame}

\begin{frame}
   \frametitle{\small 190611/plot\_time\_40000\_narrow}
   \centering
   \includegraphics[width=0.6\linewidth]{190611/plot_time_40000_narrow.pdf}
\end{frame}

\begin{frame}
   \frametitle{\small 190611/plot\_time\_160000\_narrow}
   \centering
   \includegraphics[width=0.6\linewidth]{190611/plot_time_160000_narrow.pdf}
\end{frame}

\begin{frame}
   \frametitle{\small 190611/plot\_time\_60000\_narrow}
   \centering
   \includegraphics[width=0.6\linewidth]{190611/plot_time_60000_narrow.pdf}
\end{frame}

\begin{frame}
   \frametitle{\small 190611/plot\_time\_80000\_narrow}
   \centering
   \includegraphics[width=0.6\linewidth]{190611/plot_time_80000_narrow.pdf}
\end{frame}

\begin{frame}
   \frametitle{\small 190611/plot\_time\_narrow}
   \centering
   \includegraphics[width=0.6\linewidth]{190611/plot_time_narrow.pdf}
\end{frame}



\backupend
\end{comment}

\end{document}
