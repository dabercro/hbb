\documentclass{beamer}

\author[D. Abercrombie]{
  Daniel Abercrombie \\
  for the VHbb Group
}

\title{\bf \sffamily B-jet Energy Smearing for the VHbb Analysis}
\date{\today}

\usecolortheme{dove}

\usepackage[absolute,overlay]{textpos}
\usefonttheme{serif}
\usepackage{appendixnumberbeamer}
\usepackage{isotope}
\usepackage{hyperref}
\usepackage[english]{babel}
\usepackage{amsmath}
\setbeamerfont{frametitle}{size=\Large,series=\bf\sffamily}
\setbeamertemplate{frametitle}[default][center]
\usepackage{siunitx}
\usepackage{tabularx}
\usepackage{makecell}
\usepackage{comment}

\usepackage{feynmp-auto}

\setbeamertemplate{navigation symbols}{}
\usepackage{graphicx}
\usepackage{color}
\setbeamertemplate{footline}[text line]{\parbox{1.083\linewidth}{\footnotesize \hfill \insertshortauthor \hfill \insertpagenumber /\inserttotalframenumber}}
\setbeamertemplate{headline}[text line]{\parbox{1.083\linewidth}{\footnotesize \hspace{-0.083\linewidth} \textcolor{blue}{\sffamily \insertsection \hfill \insertsubsection}}}

\IfFileExists{/Users/dabercro/GradSchool/Presentations/MIT-logo.pdf}
             {\logo{\includegraphics[height=0.5cm]{/Users/dabercro/GradSchool/Presentations/MIT-logo.pdf}}}
             {\logo{\includegraphics[height=0.5cm]{/home/dabercro/MIT-logo.pdf}}}

\usepackage{changepage}

\newcommand{\beginbackup}{
  \newcounter{framenumbervorappendix}
  \setcounter{framenumbervorappendix}{\value{framenumber}}
}
\newcommand{\backupend}{
  \addtocounter{framenumbervorappendix}{-\value{framenumber}}
  \addtocounter{framenumber}{\value{framenumbervorappendix}}
}

\graphicspath{{figs/}}

\newcommand{\link}[2]{\href{#2}{\textcolor{blue}{\underline{#1}}}}
\newcommand{\clink}[2]{\link{#1}{http://t3serv001.mit.edu/~dabercro/redir/?k=#2}}}

\newcommand{\twofigs}[4]{
  \begin{columns}
    \begin{column}{0.5\linewidth}
      \centering
      \textcolor{blue}{#1} \\
      \includegraphics[width=\linewidth]{#2}
    \end{column}
    \begin{column}{0.5\linewidth}
      \centering
      \textcolor{blue}{#3} \\
      \includegraphics[width=\linewidth]{#4}
    \end{column}
  \end{columns}
}

\newcommand{\fourfigs}[8]{
  \begin{columns}
    \begin{column}{0.3\linewidth}
      \centering
      \textcolor{blue}{#1} \\
      \includegraphics[width=\linewidth]{#2} \\
      \textcolor{blue}{#3} \\
      \includegraphics[width=\linewidth]{#4}
    \end{column}
    \begin{column}{0.3\linewidth}
      \centering
      \textcolor{blue}{#5} \\
      \includegraphics[width=\linewidth]{#6} \\
      \textcolor{blue}{#7} \\
      \includegraphics[width=\linewidth]{#8}
    \end{column}
  \end{columns}
}

\newcommand{\ttbar}{\ensuremath{t\bar{t}}}
\newcommand{\bbbar}{\ensuremath{b\bar{b}}}

\begin{document}


\begin{frame}
  \titlepage
\end{frame}


\begin{frame}
  \frametitle{Introduction}

  \begin{itemize}
  \item Smearing needs to be re-derived after applying
    energy regression designed to recover energy
    lost in the weak decays of b-jets.
  \item Smearing method for discovery analysis does not close for all phase space
  \item Extracting smearing parameters as a function of $\rho$
    improves closure, but results in large fit uncertainties
  \item Alternative to binning would be to do
    an unbinned fit with a factorized PDF
  \item First step is to perform this fit in 2D space
    (jet response vs $\alpha$), and compare to binned fit results
  \end{itemize}

\end{frame}


\begin{frame}
  \frametitle{Jet Resolution Approach}

  \begin{itemize}
  \item Good detector resolution of leptons allows us to \\
    take advantage of the following process:
  \end{itemize}

  \hfill

  \begin{center}
    \begin{fmffile}{DY_bjet}
      \begin{fmfgraph*}(200, 120)
        \fmfleft{i1,i0}
        \fmfright{o3,o2,o1,o0}
        \fmf{fermion, label=$q$}{i0,v0,v1,i1}
        \fmf{photon, label=$Z$}{v0,v0_1}
        \fmf{fermion, label=$\ell$}{o0,v0_1,o1}
        \fmf{gluon}{v1,v1_1}
        \fmf{fermion, label=$b$}{o2,v1_1,o3}
      \end{fmfgraph*}
    \end{fmffile}
  \end{center}

  \hfill

  \begin{itemize}
  \item The jet resolution can be measured by assuming it is \\
    balanced against the $Z$ in the transverse plane
  \end{itemize}

\end{frame}


\begin{frame}
  \frametitle{Samples}

  Using NanoAODv6 setup with updated B-jet regression weights:

  \begin{itemize}
  \item \texttt{\small /DoubleMuon/Run2018A-17Sep2018-v2/MINIAOD}
  \item \texttt{\small /DoubleMuon/Run2018B-17Sep2018-v1/MINIAOD}
  \item \texttt{\small /DoubleMuon/Run2018C-17Sep2018-v1/MINIAOD}
  \item \texttt{\small /DoubleMuon/Run2018D-PromptReco-v2/MINIAOD}
  \item \texttt{\small /EGamma/Run2018A-17Sep2018-v2/MINIAOD}
  \item \texttt{\small /EGamma/Run2018B-17Sep2018-v1/MINIAOD}
  \item \texttt{\small /EGamma/Run2018C-17Sep2018-v1/MINIAOD}
  \item \texttt{\small /EGamma/Run2018D-22Jan2019-v2/MINIAOD}
  \end{itemize}

\end{frame}


\begin{frame}
  \frametitle{NanoAOD Samples}

  Using NanoAODv6 setup with updated B-jet regression weights:

  \begin{itemize}
  \item \texttt{\small /DYJetsToLL\_M-50\_HT-*\_TuneCP5\_PSweights\_\\13TeV-madgraphMLM-pythia8/\\RunIIAutumn18MiniAOD-102X\_upgrade2018\_realistic\_v15-v?\footnote{400to600 is v3, all others are v2}/\\MINIAODSIM}
  \item \texttt{\small /TTTo2L2Nu\_TuneCP5\_13TeV-powheg-pythia8/\\RunIIAutumn18MiniAOD-102X\_upgrade2018\_realistic\_v15-v1/\\MINIAODSIM}
  \end{itemize}

\end{frame}


\begin{frame}
  \frametitle{Selection -- Triggers}

  In data only, there are no trigger efficiencies applied to MC

  \begin{itemize}
  \item Triggers:
    \begin{itemize}
    \item \texttt{HLT\_Mu17\_TrkIsoVVL\_Mu8\_TrkIsoVVL\_DZ\_Mass3p8}
    \item \texttt{HLT\_Mu17\_TrkIsoVVL\_Mu8\_TrkIsoVVL\_DZ\_Mass8}
    \item \texttt{HLT\_Ele115\_CaloIdVT\_GsfTrkIdT}
    \item \texttt{HLT\_Ele27\_WPTight\_Gsf}
    \item \texttt{HLT\_Ele28\_WPTight\_Gsf}
    \item \texttt{HLT\_Ele32\_WPTight\_Gsf}
    \item \texttt{HLT\_Ele35\_WPTight\_Gsf}
    \item \texttt{HLT\_Ele38\_WPTight\_Gsf}
    \item \texttt{HLT\_Ele40\_WPTight\_Gsf}
    \item \texttt{HLT\_Ele32\_WPTight\_Gsf\_L1DoubleEG}
    \end{itemize}
  \end{itemize}

\end{frame}


\begin{frame}
  \frametitle{Selection -- Leptons}

  \begin{itemize}
  \item Two leptons satisfying:
    \begin{itemize}
    \item $p_T > \SI{20}{GeV}$
    \item $q_1 + q_2 = 0$
    \item Same flavor
    \item Z Selection:
      \begin{itemize}
      \item Muons: \\
        \texttt{pfRelIso04\_all} $< 0.25$ and \\
        $\Delta xy < 0.05$ and $\Delta z < 0.2$
      \item Electrons: \\
        \texttt{mvaFall17V2Iso\_WP90} and \texttt{pfRelIso03\_all} $< 0.15$
      \end{itemize}
    \end{itemize}
  \item Any third lepton (either muon or electron) causes event to be thrown out
  \item $\SI{71}{GeV} < m_{\ell\ell} < \SI{111}{GeV}$
  \item $p_{T,\ell\ell} > \SI{100}{GeV}$
  \end{itemize}

\end{frame}


\begin{frame}
  \frametitle{Selection -- Jets}

  Selecting events with one jet is hard due to jet acceptance,
  and would result in few events

  \vfill

  \begin{itemize}
  \item Exactly two jets satisfying:
    \begin{itemize}
    \item Does not overlap with any leptons satisfying Z Selection criteria
    \item $p_T > \SI{20}{GeV}$ after regression
    \item \texttt{jetId} $> 0$ and \texttt{puId} $> 0$
    \end{itemize}
  \item Leading jet satisfies:
    \begin{itemize}
    \item $|\eta| < 2.4$
    \item $\Delta\phi(j, \ell\ell) > 2.8$
    \item \texttt{btagDeepB} $> 0.9$
    \end{itemize}
  \item Trailing jet statisfies: $\alpha = \frac{p_{T, j}}{p_{T, \ell\ell}} < 0.3$
  \end{itemize}

\end{frame}


\begin{frame}
  \frametitle{MC Weights}

  \begin{itemize}
  \item Reweight by \texttt{genWeight * btagWeight\_DeepCSVB}
  \item Cross sections:
    \vfill
    \begin{tabular}{|l|S|}
      \hline
      Sample & \mathrm{Cross Section [pb]} \\
      \hline
      \texttt{DYJetsToLL\_M-50\_HT-70to100\_*} & 143.0 \\
      \texttt{DYJetsToLL\_M-50\_HT-100to200\_*} & 147.4 \\
      \texttt{DYJetsToLL\_M-50\_HT-200to400\_*} & 40.99 \\
      \texttt{DYJetsToLL\_M-50\_HT-400to600\_*} & 5.678 \\
      \texttt{DYJetsToLL\_M-50\_HT-600to800\_*} & 1.367 \\
      \texttt{DYJetsToLL\_M-50\_HT-800to1200\_*} & 0.6304 \\
      \texttt{DYJetsToLL\_M-50\_HT-1200to2500\_*} & 0.1514 \\
      \texttt{DYJetsToLL\_M-50\_HT-2500toInf\_*} & 0.003565 \\
      \texttt{TTTo2L2Nu\_* (powheg)} & 88.288 \\
      \hline
    \end{tabular}
    \vfill
    DY are MadGraph samples
  \end{itemize}

\end{frame}


\begin{frame}
  \frametitle{Binned Fit Strategy}

  \begin{itemize}
  \item We use events with a second jet, and extrapolate down to where $\alpha = 0$
  \item This should give us the behavior of the one jet diagram,
    without needing to deal with few events or jet acceptance.
  \item Create four bins of $\alpha$:
  \end{itemize}
  \begin{columns}
    \begin{column}{0.5\linewidth}
      \includegraphics[width=\linewidth]{200303_alpha_lines_v2/smearplot_alpha.pdf}
    \end{column}
    \begin{column}{0.5\linewidth}
      \[
      \alpha
      \begin{cases}
        0 < \alpha < 0.155 \\
        0.155 \le \alpha < 0.185 \\
        0.185 \le \alpha < 0.23 \\
        0.23 \le \alpha < 0.3
      \end{cases}
      \]
    \end{column}
  \end{columns}

\end{frame}


\begin{frame}
  \frametitle{Binned Fit Strategy}

  \begin{columns}
    \begin{column}{0.3\linewidth}
      \centering
      \includegraphics[width=\linewidth]{200303_nbjets_noenv/smearplot_1_jet1_response.pdf} \\
      \includegraphics[width=\linewidth]{200303_nbjets_noenv/smearplot_3_jet1_response.pdf}
    \end{column}
    \begin{column}{0.3\linewidth}
      \centering
      \includegraphics[width=\linewidth]{200303_nbjets_noenv/smearplot_2_jet1_response.pdf} \\
      \includegraphics[width=\linewidth]{200303_nbjets_noenv/smearplot_4_jet1_response.pdf}
    \end{column}
  \end{columns}

  \vfill
  We can see the MC resolution is better than data in all bins

\end{frame}


\begin{frame}
  \frametitle{Binned Fit Uncertainties}

  \begin{itemize}
  \item Standard deviation of each histogram has some uncertainty \\
    that can be propagated through the fit
  \item Previous plots just have statistical uncertainties
  \item We add the the envelopes from parton shower and \\
    scale (renormalization/refactorization) weights
  \end{itemize}

  \begin{columns}
    \begin{column}{0.5\linewidth}
      \centering
      \textcolor{blue}{Statistical Uncertainties} \\
      \includegraphics[width=0.8\linewidth]{200303_nbjets_noenv/smearplot_2_jet1_response.pdf}
    \end{column}
    \begin{column}{0.5\linewidth}
      \centering
      \textcolor{blue}{Additional Uncertainties} \\
      \includegraphics[width=0.8\linewidth]{200303_nbjets/smearplot_2_jet1_response.pdf}
    \end{column}
  \end{columns}

\end{frame}


\begin{frame}
  \frametitle{Binned Fit Formula}

  \[
  f(\alpha) = (m \times \alpha) \oplus b \times (1 + c_k \times \alpha)
  \]

  \begin{itemize}
  \item $c_k$ is fixed by a linear fit to the MC's intrinsic jet resolution ($p_{T, reco}/p_{T, gen}$) over $\alpha$ as $c_k = m_0/q_0$
  \item Smearing is done by scaling difference between $p_{T,reco}$ and $p_{T,gen}$ by $b_{data}/b_{MC}$
  \end{itemize}

  \centering
  \includegraphics[width=0.6\linewidth]{200303_smear_200303_nbjets/resolution_jet1_response_smear_0.pdf}

\end{frame}


\begin{frame}
  \frametitle{Binned Fit Results}

  Uncertainties calculated with:

  \[
  \delta = \sqrt{\left(\frac{\delta_{data}}{c_{MC}}\right)^2 +
    \left(\frac{c_{data}\delta_{MC}}{c_{MC}^2}\right)^2}
  \]

  \vfill

  Generator and reco difference to be scaled by: $3.2 \pm 6.6 \%$

\end{frame}


\begin{frame}
  \frametitle{Unbinned Fit Strategy}

  \begin{itemize}
  \item Same events, variables, cuts, and weights as before
  \item Fit 2D p.d.f. over jet response (labelled $x$) and $\alpha$
  \[
  f(x, \alpha) =
  G(x, \mu(\alpha), \sigma(\alpha)|\alpha)
  \times
  F(\alpha, \mu_0, \sigma_0)
  \]
  \item $G$ is a Gaussian distribution
  \item $\mu$ as a linear function
  \item $F$ is a Gaussian with fixed mean $\mu_0$ and width $\sigma_0$
  \item $\sigma$ is of the same form as the binned fit,
    with $c_k$ being derived from a fit to
    intrinsic resolution with a linear $\sigma(\alpha)$:
  \[
  \sigma(\alpha) = (m \times \alpha) \oplus c \times (1 + c_k \times \alpha)
  \]
  \end{itemize}

\end{frame}


\begin{frame}
  \frametitle{Unbinned Fit Uncertainties}

  \begin{itemize}
  \item Cannot make histogram envelope for uncertainties, \\
    so made separate fits for each weight
  \end{itemize}

\end{frame}


\begin{frame}
  \frametitle{Comparison of Results}

  \begin{itemize}
  \item As before, does not agree for all $\alpha$
  \end{itemize}

\end{frame}



\begin{comment}
\beginbackup

\begin{frame}
  \centering
    {\Huge \bf\sffamily Backup Slides}
\end{frame}

\begin{frame}
   \frametitle{\small 190611/plot\_time\_60000\_wide}
   \centering
   \includegraphics[width=0.6\linewidth]{190611/plot_time_60000_wide.pdf}
\end{frame}

\begin{frame}
   \frametitle{\small 190611/plot\_time\_wide}
   \centering
   \includegraphics[width=0.6\linewidth]{190611/plot_time_wide.pdf}
\end{frame}

\begin{frame}
   \frametitle{\small 190611/plot\_time\_120000\_compare}
   \centering
   \includegraphics[width=0.6\linewidth]{190611/plot_time_120000_compare.pdf}
\end{frame}

\begin{frame}
   \frametitle{\small 190611/plot\_time\_60000\_compare}
   \centering
   \includegraphics[width=0.6\linewidth]{190611/plot_time_60000_compare.pdf}
\end{frame}

\begin{frame}
   \frametitle{\small 190611/plot\_time\_80000\_compare}
   \centering
   \includegraphics[width=0.6\linewidth]{190611/plot_time_80000_compare.pdf}
\end{frame}

\begin{frame}
   \frametitle{\small 190611/plot\_time\_120000\_narrow}
   \centering
   \includegraphics[width=0.6\linewidth]{190611/plot_time_120000_narrow.pdf}
\end{frame}

\begin{frame}
   \frametitle{\small 190611/plot\_time\_40000\_narrow}
   \centering
   \includegraphics[width=0.6\linewidth]{190611/plot_time_40000_narrow.pdf}
\end{frame}

\begin{frame}
   \frametitle{\small 190611/plot\_time\_160000\_narrow}
   \centering
   \includegraphics[width=0.6\linewidth]{190611/plot_time_160000_narrow.pdf}
\end{frame}

\begin{frame}
   \frametitle{\small 190611/plot\_time\_60000\_narrow}
   \centering
   \includegraphics[width=0.6\linewidth]{190611/plot_time_60000_narrow.pdf}
\end{frame}

\begin{frame}
   \frametitle{\small 190611/plot\_time\_80000\_narrow}
   \centering
   \includegraphics[width=0.6\linewidth]{190611/plot_time_80000_narrow.pdf}
\end{frame}

\begin{frame}
   \frametitle{\small 190611/plot\_time\_narrow}
   \centering
   \includegraphics[width=0.6\linewidth]{190611/plot_time_narrow.pdf}
\end{frame}



\backupend
\end{comment}

\end{document}
