\documentclass{beamer}

\author[D. Abercrombie]{
  Daniel Abercrombie, \\
  Guillelmo G\'omez-Ceballos, \\
  Dymtro Kovalskyi, \\
  Benedikt Maier, \\
  Christoph Paus
}

\title{\bf \sffamily Moving to Factorized PDFs \\ for Smearing Extraction}
\date{\today}

\usecolortheme{dove}

\usepackage[absolute,overlay]{textpos}
\usefonttheme{serif}
\usepackage{appendixnumberbeamer}
\usepackage{isotope}
\usepackage{hyperref}
\usepackage[english]{babel}
\usepackage{amsmath}
\setbeamerfont{frametitle}{size=\Large,series=\bf\sffamily}
\setbeamertemplate{frametitle}[default][center]
\usepackage{siunitx}
\usepackage{tabularx}
\usepackage{makecell}
\usepackage{comment}

\usepackage{feynmp-auto}

\setbeamertemplate{navigation symbols}{}
\usepackage{graphicx}
\usepackage{color}
\setbeamertemplate{footline}[text line]{\parbox{1.083\linewidth}{\footnotesize \hfill \insertshortauthor \hfill \insertpagenumber /\inserttotalframenumber}}
\setbeamertemplate{headline}[text line]{\parbox{1.083\linewidth}{\footnotesize \hspace{-0.083\linewidth} \textcolor{blue}{\sffamily \insertsection \hfill \insertsubsection}}}

\IfFileExists{/Users/dabercro/GradSchool/Presentations/MIT-logo.pdf}
             {\logo{\includegraphics[height=0.5cm]{/Users/dabercro/GradSchool/Presentations/MIT-logo.pdf}}}
             {\logo{\includegraphics[height=0.5cm]{/home/dabercro/MIT-logo.pdf}}}

\usepackage{changepage}

\newcommand{\beginbackup}{
  \newcounter{framenumbervorappendix}
  \setcounter{framenumbervorappendix}{\value{framenumber}}
}
\newcommand{\backupend}{
  \addtocounter{framenumbervorappendix}{-\value{framenumber}}
  \addtocounter{framenumber}{\value{framenumbervorappendix}}
}

\graphicspath{{figs/}}

\newcommand{\link}[2]{\href{#2}{\textcolor{blue}{\underline{#1}}}}
\newcommand{\clink}[2]{\link{#1}{http://t3serv001.mit.edu/~dabercro/redir/?k=#2}}}

\newcommand{\twofigs}[4]{
  \begin{columns}
    \begin{column}{0.5\linewidth}
      \centering
      \textcolor{blue}{#1} \\
      \includegraphics[width=\linewidth]{#2}
    \end{column}
    \begin{column}{0.5\linewidth}
      \centering
      \textcolor{blue}{#3} \\
      \includegraphics[width=\linewidth]{#4}
    \end{column}
  \end{columns}
}

\newcommand{\fourfigs}[8]{
  \begin{columns}
    \begin{column}{0.3\linewidth}
      \centering
      \textcolor{blue}{#1} \\
      \includegraphics[width=\linewidth]{#2} \\
      \textcolor{blue}{#3} \\
      \includegraphics[width=\linewidth]{#4}
    \end{column}
    \begin{column}{0.3\linewidth}
      \centering
      \textcolor{blue}{#5} \\
      \includegraphics[width=\linewidth]{#6} \\
      \textcolor{blue}{#7} \\
      \includegraphics[width=\linewidth]{#8}
    \end{column}
  \end{columns}
}

\newcommand{\ttbar}{\ensuremath{t\bar{t}}}
\newcommand{\bbbar}{\ensuremath{b\bar{b}}}

\begin{document}

\begin{frame}
  \titlepage
\end{frame}

\begin{frame}
  \frametitle{Introduction}

  \begin{columns}
    \begin{column}{0.6\linewidth}
      \begin{itemize}
      \item Smearing needs to be re-derived after applying
        energy regression designed to recover energy
        lost in the weak decays of b-jets.
      \item Extracting smearing parameters as a function of $\rho$
        results in large fit uncertainties
      \item Alternative to binning would be to do
        an unbinned fit with a factorized PDF
      \item First step is to perform this fit in 2D space
        (jet response vs $\alpha$), and compare to binned fit results
      \end{itemize}
    \end{column}
    \begin{column}{0.4\linewidth}
      \includegraphics[width=\linewidth]{191216_resolution_tosmear/smear_fit.pdf}
    \end{column}
  \end{columns}

  \vfill
  In other words, the binned fit presented here is a step back
  in order to verify the following unbinned fit

\end{frame}

\begin{frame}
  \frametitle{Jet Resolution Approach}

  \begin{itemize}
  \item Good detector resolution of leptons allows us to \\
    take advantage of the following process:
  \end{itemize}

  \hfill

  \begin{center}
    \begin{fmffile}{DY_bjet}
      \begin{fmfgraph*}(200, 120)
        \fmfleft{i1,i0}
        \fmfright{o3,o2,o1,o0}
        \fmf{fermion, label=$q$}{i0,v0,v1,i1}
        \fmf{photon, label=$Z$}{v0,v0_1}
        \fmf{fermion, label=$\ell$}{o0,v0_1,o1}
        \fmf{gluon}{v1,v1_1}
        \fmf{fermion, label=$b$}{o2,v1_1,o3}
      \end{fmfgraph*}
    \end{fmffile}
  \end{center}

  \hfill

  \begin{itemize}
  \item The jet resolution can be measured by assuming it is \\
    balanced against the $Z$ in the transverse plane
  \end{itemize}

\end{frame}


\begin{frame}
  \frametitle{NanoAOD Samples}

  Using NanoAODv5 data samples:

  \begin{itemize}
  \item \texttt{\small /DoubleMuon/Run2018A-Nano1June2019-v1/NANOAOD}
  \item \texttt{\small /DoubleMuon/Run2018B-Nano25Oct2019-v1/NANOAOD}
  \item \texttt{\small /DoubleMuon/Run2018C-Nano25Oct2019-v1/NANOAOD}
  \item \texttt{\small /DoubleMuon/Run2018D-Nano1June2019\_ver2-v1/NANOAOD}
  \item \texttt{\small /EGamma/Run2018A-Nano25Oct2019-v1/NANOAOD}
  \item \texttt{\small /EGamma/Run2018B-Nano25Oct2019-v1/NANOAOD}
  \item \texttt{\small /EGamma/Run2018C-Nano25Oct2019-v1/NANOAOD}
  \item \texttt{\small /EGamma/Run2018D-Nano25Oct2019\_ver2-v1/NANOAOD}
  \end{itemize}

\end{frame}


\begin{frame}
  \frametitle{NanoAOD Samples}

  Using NanoAODv5 MC samples:

  \begin{itemize}
  \item \texttt{\small /DYJetsToLL\_M-50\_HT-*\_TuneCP5\_PSweights\_\\13TeV-madgraphMLM-pythia8/\\RunIIAutumn18NanoAODv5-Nano1June2019\_102X\_\\upgrade2018\_realistic\_v19*/NANOAODSIM}
  \item \texttt{\small /TTTo2L2Nu\_TuneCP5\_13TeV-powheg-pythia8/\\RunIIAutumn18NanoAODv5-Nano1June2019\_102X\_\\upgrade2018\_realistic\_v19-v1/NANOAODSIM}
  \end{itemize}

\end{frame}

\begin{frame}
  \frametitle{Jet Resolution Selection}

  \begin{itemize}
  \item Triggers:
    \begin{itemize}
    \item \texttt{HLT\_Mu17\_TrkIsoVVL\_Mu8\_TrkIsoVVL\_DZ\_Mass3p8}
    \item \texttt{HLT\_Mu17\_TrkIsoVVL\_Mu8\_TrkIsoVVL\_DZ\_Mass8}
    \item \texttt{HLT\_Ele115\_CaloIdVT\_GsfTrkIdT}
    \item \texttt{HLT\_Ele27\_WPTight\_Gsf}
    \item \texttt{HLT\_Ele28\_WPTight\_Gsf}
    \item \texttt{HLT\_Ele32\_WPTight\_Gsf}
    \item \texttt{HLT\_Ele35\_WPTight\_Gsf}
    \item \texttt{HLT\_Ele38\_WPTight\_Gsf}
    \item \texttt{HLT\_Ele40\_WPTight\_Gsf}
    \item \texttt{HLT\_Ele32\_WPTight\_Gsf\_L1DoubleEG}
    \end{itemize}
  \end{itemize}

\end{frame}

\begin{frame}
  \frametitle{Jet Resolution Selection}

  \begin{itemize}
  \item Two leptons satisfying:
    \begin{itemize}
    \item $p_T > \SI{20}{GeV}$
    \item $q_1 + q_2 = 0$
    \item Same flavor
    \item Z Selection:
      \begin{itemize}
      \item Muons: \\
        \texttt{pfRelIso04\_all} $< 0.25$ and \\
        $\Delta xy < 0.05$ and $\Delta z < 0.2$
      \item Electrons: \\
        \texttt{mvaFall17V2Iso\_WP90} and \texttt{pfRelIso03\_all} $< 0.15$
      \end{itemize}
    \end{itemize}
  \item Any third lepton with same flavor causes event to be thrown out
  \item $\SI{71}{GeV} < m_{\ell\ell} < \SI{111}{GeV}$
  \item $p_{T,\ell\ell} > \SI{100}{GeV}$
  \end{itemize}

\end{frame}


\begin{frame}
  \frametitle{Jet Resolution Selection}

  Selecting events with one jet is hard due to jet acceptance,
  and would result in few events

  \vfill

  \begin{itemize}
  \item Exactly two jets satisfying:
    \begin{itemize}
    \item Does not overlap with any leptons satisfying Z Selection criteria
    \item $p_T > \SI{20}{GeV}$ after regression
    \item \texttt{jetId} $> 0$ and \texttt{puId} $> 0$
    \end{itemize}
  \item Leading jet satisfies:
    \begin{itemize}
    \item $|\eta| < 2.4$
    \item $\Delta\phi(j, \ell\ell) > 2.8$
    \item \texttt{btagDeepB} $> 0.9$
    \end{itemize}
  \item Trailing jet statisfies: $\alpha = \frac{p_{T, j}}{p_{T, \ell\ell}} < 0.3$
  \end{itemize}

\end{frame}


\begin{frame}
  \frametitle{Corrections}

  NanoAODv5 does not include the latest JECs, so we reapply those:

  \begin{itemize}
  \item Data: \texttt{Winter19\_Autumn18\{A, B, C, D\}\_V19\_DATA}
  \item MC: \texttt{Winter19\_Autumn18\_V19\_MC}
  \end{itemize}

  Both the \texttt{Jet} collection and \texttt{MET} are adjusted by these

\end{frame}


\begin{frame}
  \frametitle{MC Weights}

  \begin{itemize}
  \item Reweight by \texttt{genWeight * btagWeight\_DeepCSVB}
  \item Cross sections:
    \vfill
    \begin{tabular}{|l|S|}
      \hline
      Sample & \mathrm{Cross Section [pb]} \\
      \hline
      \texttt{DYJetsToLL\_M-50\_HT-70to100\_*} & 143.0 \\
      \texttt{DYJetsToLL\_M-50\_HT-100to200\_*} & 147.4 \\
      \texttt{DYJetsToLL\_M-50\_HT-200to400\_*} & 40.99 \\
      \texttt{DYJetsToLL\_M-50\_HT-400to600\_*} & 5.678 \\
      \texttt{DYJetsToLL\_M-50\_HT-600to800\_*} & 1.367 \\
      \texttt{DYJetsToLL\_M-50\_HT-800to1200\_*} & 0.6304 \\
      \texttt{DYJetsToLL\_M-50\_HT-1200to2500\_*} & 0.1514 \\
      \texttt{DYJetsToLL\_M-50\_HT-2500toInf\_*} & 0.003565 \\
      \texttt{TTTo2L2Nu\_* (powheg)} & 88.288 \\
      \hline
    \end{tabular}
    \vfill
    DY are MadGraph samples
  \end{itemize}

\end{frame}


\begin{frame}
  \frametitle{Binned Fit Strategy}

  \begin{itemize}
  \item We use events with a second jet, and extrapolate down to where $\alpha = 0$
  \item This should give us the behavior of the one jet diagram,
    without needing to deal with few events or jet acceptance.
  \item Create four bins of $\alpha$:
  \end{itemize}
  \begin{columns}
    \begin{column}{0.5\linewidth}
      \includegraphics[width=\linewidth]{200116_singlebin_alpha/smearplot_alpha.pdf}
    \end{column}
    \begin{column}{0.5\linewidth}
      \[
      \alpha
      \begin{cases}
        0 < \alpha < 0.155 \\
        0.155 \le \alpha < 0.185 \\
        0.185 \le \alpha < 0.23 \\
        0.23 \le \alpha < 0.3
      \end{cases}
      \]
    \end{column}
  \end{columns}


\end{frame}


\begin{frame}
  \frametitle{Binned Fit Strategy}

  \begin{columns}
    \begin{column}{0.3\linewidth}
      \centering
      \includegraphics[width=\linewidth]{200115_1D_bins/smearplot_1_jet1_response.pdf} \\
      \includegraphics[width=\linewidth]{200115_1D_bins/smearplot_3_jet1_response.pdf}
    \end{column}
    \begin{column}{0.3\linewidth}
      \centering
      \includegraphics[width=\linewidth]{200115_1D_bins/smearplot_2_jet1_response.pdf} \\
      \includegraphics[width=\linewidth]{200115_1D_bins/smearplot_4_jet1_response.pdf}
    \end{column}
  \end{columns}

  \vfill
  We can see the MC resolution is better than data in all bins

\end{frame}


\begin{frame}
  \frametitle{Binned Fit Uncertainties}

  \begin{itemize}
  \item Standard deviation of each histogram has some uncertainty \\
    that can be propagated through the fit
  \item Previous plots just have statistical uncertainties
  \item We add the the envelopes from parton shower and \\
    scale (renormalization/refactorization) weights
  \end{itemize}

  \begin{columns}
    \begin{column}{0.5\linewidth}
      \centering
      \textcolor{blue}{Statistical Uncertainties} \\
      \includegraphics[width=0.8\linewidth]{200115_1D_bins/smearplot_2_jet1_response.pdf}
    \end{column}
    \begin{column}{0.5\linewidth}
      \centering
      \textcolor{blue}{Additional Uncertainties} \\
      \includegraphics[width=0.8\linewidth]{200116_1D_bins_allenv/smearplot_2_jet1_response.pdf}
    \end{column}
  \end{columns}

\end{frame}


\begin{frame}
  \frametitle{Comparison of Uncertainty Sources}

  \vspace{8pt}

  \twofigs{Scale Uncertainties}
          {200116_1D_bins_scaleenv/smearplot_2_jet1_response.pdf}
          {Parton Shower Uncertainties}
          {200116_1D_bins_psenv/smearplot_2_jet1_response.pdf}

  \begin{itemize}
  \item These plots have no statistical uncertainties
  \item Scale Uncertainties are larger in the center
  \item Parton Shower Uncertainties contribute more at edges
  \end{itemize}

\end{frame}


\begin{frame}
  \frametitle{Binned Fit Formula}

  \[
  f(\alpha) = (m \times \alpha) \oplus c \times (1 + c_k \times \alpha)
  \]

  \begin{itemize}
  \item $c_k$ is fixed by a linear fit to the MC's intrinsic jet resolution ($p_{T, reco}/p_{T, gen}$) over $\alpha$ as $c_k = m_0/q_0$
  \item Scaling is done by $c_{data}/c_{MC}$
  \item Smearing is done by $\sqrt{c_{data}^2 - c_{MC}^2}$
  \end{itemize}

  \centering
  \includegraphics[width=0.6\linewidth]{200116_1D_bins_allenv_resolution/resolution_jet1_response_smear_0.pdf}

\end{frame}


\begin{frame}
  \frametitle{Binned Fit Results}

  Uncertainties calculated with:

  \[
  \delta_{scale} = \sqrt{\left(\frac{\delta_{data}}{c_{MC}}\right)^2 +
    \left(\frac{c_{data}\delta_{MC}}{c_{MC}^2}\right)^2}
  \]
  \[
  \delta_{smear} = \sqrt{\frac{(c_{data}\delta_{data})^2 +
      (c_{MC}\delta_{MC})^2}{c_{data}^2 - c_{MC}^2}}
  \]

  \vfill
  \centering
  \begin{tabular}{l|c|c|}
    & Smearing & Scaling \\
    \hline
    Result & $5.1 \pm 2.7 \%$ & $6.4 \pm 7.2 \%$ \\
    \hline
  \end{tabular}

\end{frame}


\begin{frame}
  \frametitle{Unbinned Fit Strategy}

  \begin{itemize}
  \item Same events, variables, cuts, and weights as before
  \item Fit 2D p.d.f. over jet response (labelled $x$) and $\alpha$
  \[
  f(x, \alpha) =
  G(x, \mu(\alpha), \sigma(\alpha)|\alpha)
  \times
  F(\alpha, \mu_0, \sigma_0)
  \]
  \item $G$ is a Gaussian distribution
  \item $\mu$ as a linear function
  \item $F$ is a Gaussian with fixed mean $\mu_0$ and width $\sigma_0$
  \item $\sigma$ is of the same form as the binned fit,
    with $c_k$ being derived from a fit to
    intrinsic resolution with a linear $\sigma(\alpha)$:
  \[
  \sigma(\alpha) = (m \times \alpha) \oplus c \times (1 + c_k \times \alpha)
  \]
  \end{itemize}


\end{frame}


\begin{frame}
  \frametitle{Unbinned Fit Uncertainties}

  \begin{itemize}
  \item Cannot make histogram envelope for uncertainties, \\
    so made separate fits for each weight
  \item Varied shape of fit over $\alpha$ ($F$ in previous slide) to also be Landau fit
  \end{itemize}

  \twofigs{Gaussian $\alpha$ fit (MC)}
          {200121_roofit/Gaussian_xsec_weight_mc.pdf}
          {Landau $\alpha$ fit (MC)}
          {200121_roofit/Landau_xsec_weight_mc.pdf}

\end{frame}


\begin{frame}
  \frametitle{Unbinned Fit Results}

  \centering

  \begin{tabular}{|l|c|c|}
    \hline
    Uncertainty & Smearing & Scaling \\
    \hline
    Nominal Result & $4.7 \pm 2.7 \%$ & $5.5 \pm 6.5 \%$ \\
    Landau $\alpha$ & $2.9 \pm 4.0 \%$ & $2.4 \pm 6.5 \%$ \\
    \hline
    Max Scale Weight & $4.8 \pm 2.4 \%$ & $6.5 \pm 7.2 \%$ \\
    Min Scale Weight & $2.9 \pm 4.0 \%$ & $2.4 \pm 6.5 \%$ \\
    \hline
    Max Parton Shower & $4.2 \pm 3.2 \%$ & $4.9 \pm 7.6 \%$ \\
    Min Parton Shower & $-3.6 \pm 3.4 \%$ & $-3.2 \pm 5.9 \%$ \\
    \hline
  \end{tabular}

  \begin{itemize}
  \item Negative smearings means that the data would need to be smeared to match MC
  \item Varying the \texttt{PSWeight} gives a large asymmetric uncertainty
  \item Uncertainties applied are just the variables from the fit
  \end{itemize}

\end{frame}


\begin{frame}
  \frametitle{Comparison of Results}

  \fourfigs{Binned Smear}
           {200122_comparison_resolution/resolution_jet1_response_single_smear_nominal_smear_0.pdf}
           {Binned Scale}
           {200122_comparison_resolution/resolution_jet1_response_single_scale_nominal_smear_0.pdf}
           {Unbinned Smear}
           {200122_comparison_resolution/resolution_jet1_response_unbinned_smear_nominal_smear_0.pdf}
           {Unbinned Scale}
           {200122_comparison_resolution/resolution_jet1_response_unbinned_scale_nominal_smear_0.pdf}

  \begin{itemize}
  \item As before, does not agree for all $\alpha$
  \item Smearing has better agreement at $\alpha = 0$ than the scaling
  \end{itemize}

\end{frame}


\begin{frame}
  \frametitle{Di-jet Mass Peaks in Signal Samples}

  \twofigs{MET Peak}
          {200122_peak_met/comparison.pdf}
          {Di-lepton Peak}
          {200122_peak_lep/comparison.pdf}

  \begin{itemize}
  \item As expected, regression brings resolution improvement,
    even after smearing
  \end{itemize}

\end{frame}


\begin{frame}
  \frametitle{Conclusions}

  \begin{itemize}
  \item Smearing much less than previously expected now that $t\bar{t}$ is in the picture
  \item Smearing tends to close better than scaling, but requires random numbers
  \item Unbinned and binned fits give same results with no $\rho$ bins
  \item Systematics in unbinned fits are more complicated:
    \begin{itemize}
    \item Shape of $\alpha$ (and likely $\rho$) not easy to fit, and affects result
    \item Relies on fit of response mean as well
    \item Parton shower uncertainties are particularly large asymmetric
    \end{itemize}
  \end{itemize}

\end{frame}


\begin{comment}
\beginbackup

\begin{frame}
  \centering
    {\Huge \bf\sffamily Backup Slides}
\end{frame}

\begin{frame}
   \frametitle{\small 190611/plot\_time\_60000\_wide}
   \centering
   \includegraphics[width=0.6\linewidth]{190611/plot_time_60000_wide.pdf}
\end{frame}

\begin{frame}
   \frametitle{\small 190611/plot\_time\_wide}
   \centering
   \includegraphics[width=0.6\linewidth]{190611/plot_time_wide.pdf}
\end{frame}

\begin{frame}
   \frametitle{\small 190611/plot\_time\_120000\_compare}
   \centering
   \includegraphics[width=0.6\linewidth]{190611/plot_time_120000_compare.pdf}
\end{frame}

\begin{frame}
   \frametitle{\small 190611/plot\_time\_60000\_compare}
   \centering
   \includegraphics[width=0.6\linewidth]{190611/plot_time_60000_compare.pdf}
\end{frame}

\begin{frame}
   \frametitle{\small 190611/plot\_time\_80000\_compare}
   \centering
   \includegraphics[width=0.6\linewidth]{190611/plot_time_80000_compare.pdf}
\end{frame}

\begin{frame}
   \frametitle{\small 190611/plot\_time\_120000\_narrow}
   \centering
   \includegraphics[width=0.6\linewidth]{190611/plot_time_120000_narrow.pdf}
\end{frame}

\begin{frame}
   \frametitle{\small 190611/plot\_time\_40000\_narrow}
   \centering
   \includegraphics[width=0.6\linewidth]{190611/plot_time_40000_narrow.pdf}
\end{frame}

\begin{frame}
   \frametitle{\small 190611/plot\_time\_160000\_narrow}
   \centering
   \includegraphics[width=0.6\linewidth]{190611/plot_time_160000_narrow.pdf}
\end{frame}

\begin{frame}
   \frametitle{\small 190611/plot\_time\_60000\_narrow}
   \centering
   \includegraphics[width=0.6\linewidth]{190611/plot_time_60000_narrow.pdf}
\end{frame}

\begin{frame}
   \frametitle{\small 190611/plot\_time\_80000\_narrow}
   \centering
   \includegraphics[width=0.6\linewidth]{190611/plot_time_80000_narrow.pdf}
\end{frame}

\begin{frame}
   \frametitle{\small 190611/plot\_time\_narrow}
   \centering
   \includegraphics[width=0.6\linewidth]{190611/plot_time_narrow.pdf}
\end{frame}



\backupend
\end{comment}


\end{document}
