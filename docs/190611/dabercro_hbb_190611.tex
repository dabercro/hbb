\documentclass{beamer}

\author[D. Abercrombie]{
  Daniel Abercrombie
}

\title{\bf \sffamily B-regression Trainings}
\date{\today}

\usecolortheme{dove}

\usepackage[absolute,overlay]{textpos}
\usefonttheme{serif}
\usepackage{appendixnumberbeamer}
\usepackage{isotope}
\usepackage{hyperref}
\usepackage[english]{babel}
\usepackage{amsmath}
\setbeamerfont{frametitle}{size=\Large,series=\bf\sffamily}
\setbeamertemplate{frametitle}[default][center]
\usepackage{siunitx}
\usepackage{tabularx}
\usepackage{makecell}
\usepackage{comment}

\setbeamertemplate{navigation symbols}{}
\usepackage{graphicx}
\usepackage{color}
\setbeamertemplate{footline}[text line]{\parbox{1.083\linewidth}{\footnotesize \hfill \insertshortauthor \hfill \insertpagenumber /\inserttotalframenumber}}
\setbeamertemplate{headline}[text line]{\parbox{1.083\linewidth}{\footnotesize \hspace{-0.083\linewidth} \textcolor{blue}{\sffamily \insertsection \hfill \insertsubsection}}}

\IfFileExists{/Users/dabercro/GradSchool/Presentations/MIT-logo.pdf}
             {\logo{\includegraphics[height=0.5cm]{/Users/dabercro/GradSchool/Presentations/MIT-logo.pdf}}}
             {\logo{\includegraphics[height=0.5cm]{/home/dabercro/MIT-logo.pdf}}}

\usepackage{changepage}

\newcommand{\beginbackup}{
  \newcounter{framenumbervorappendix}
  \setcounter{framenumbervorappendix}{\value{framenumber}}
}
\newcommand{\backupend}{
  \addtocounter{framenumbervorappendix}{-\value{framenumber}}
  \addtocounter{framenumber}{\value{framenumbervorappendix}}
}

\graphicspath{{figs/}}

\newcommand{\link}[2]{\href{#2}{\textcolor{blue}{\underline{#1}}}}
\newcommand{\clink}[2]{\link{#1}{http://t3serv001.mit.edu/~dabercro/redir/?k=#2}}}

\newcommand{\twofigs}[4]{
  \begin{columns}
    \begin{column}{0.5\linewidth}
      \centering
      \textcolor{blue}{#1} \\
      \includegraphics[width=\linewidth]{#2}
    \end{column}
    \begin{column}{0.5\linewidth}
      \centering
      \textcolor{blue}{#3} \\
      \includegraphics[width=\linewidth]{#4}
    \end{column}
  \end{columns}
}

\newcommand{\fourfigs}[8]{
  \begin{columns}
    \begin{column}{0.3\linewidth}
      \centering
      \textcolor{blue}{#1} \\
      \includegraphics[width=\linewidth]{#2} \\
      \textcolor{blue}{#3} \\
      \includegraphics[width=\linewidth]{#4}
    \end{column}
    \begin{column}{0.3\linewidth}
      \centering
      \textcolor{blue}{#5} \\
      \includegraphics[width=\linewidth]{#6} \\
      \textcolor{blue}{#7} \\
      \includegraphics[width=\linewidth]{#8}
    \end{column}
  \end{columns}
}

\newcommand{\ttbar}{\ensuremath{t\bar{t}}}
\newcommand{\bbbar}{\ensuremath{b\bar{b}}}

\begin{document}

\begin{frame}
  \titlepage
\end{frame}

\begin{frame}
  \frametitle{Introduction}

  \begin{itemize}
  \item We are trying to verify that adding PF candidates to the regression helps
  \item We compare the the regression that was done before
  \item Using a separate check from Benedikt's, we see PF candidates help to some extent
  \item We are also seeing what looks like over training
  \item Having multiple GPUs is starting to quickly increase \\
    how many trainings we can try at the same time
  \end{itemize}

\end{frame}

\begin{frame}
  \frametitle{Hint of overtraining}

  \begin{center}
  \includegraphics[width=0.6\linewidth]{/home/dabercro/hbb/analysis/more/plot.pdf}
  \end{center}

  \begin{itemize}
  \item \textcolor{blue}{JECs}
  \item \textcolor{red}{Old training at one checkpoint}
  \item \textcolor{yellow}{Same training at later checkpoint}
  \end{itemize}

\end{frame}

\begin{frame}
  \frametitle{Jets Used in Training}

  \begin{itemize}
  \item Using framework synced with NanoAOD file provided by b-regression team
  \item Hadronic $t\bar{t}$
  \item Only requirement for jet is that it's matched to a gen-jet with a $b$ inside
  \item Overlapping neutrinos are added to jet
  \item Target $p_T$ only
  \end{itemize}

\end{frame}

\begin{frame}
  \frametitle{Variables used in training}

  \begin{itemize}
  \item Jet kinematics ($\eta, m_T, m, p_T, p_{TD}$)
  \item Jet energy fractions (charged, neutral; EM, hadronic)
  \item Energy rings (charged, neutral, EM, muonic; $R < 0.05, 0.1, 0.2, 0.3, 0.4$)
  \item Number of constituents
  \item Pileup ($\rho$)
  \item Vertex kinematics (location, significance, $m, n_\mathrm{trk}, p_T$)
  \item Leading track $p_T$
  \item Leading lepton kinematics ($p_T, \Delta R, p_{T\perp j}$)
  \item One hot encoded lepton PDGID
  \end{itemize}

  We added information for the first 5 PF candidates

  \begin{itemize}
  \item $p_T$ fraction
  \item $\Delta \eta, \Delta \phi$
  \item $\Delta z, \Delta xy$
  \item Puppi weight
  \item PDGID (not one hot encoded...)
  \end{itemize}

\end{frame}

\begin{frame}
  \frametitle{Model loss function}

  \begin{itemize}
  \item Only concerned with central value for now
  \item Use Huber loss function for both cases
  \end{itemize}

  \begin{gather*}
    y = x_\mathrm{pred} - x_\mathrm{true} \\
    f(y) = \begin{cases}
      \frac12 y^2 & \mbox{for } |y| < \delta \\
      \delta(|y| - \frac12 \delta) & \mbox{otherwise}
      \end{cases}
  \end{gather*}

  \begin{itemize}
  \item Tried narrow ($\delta = 0.05$) and wide ($\delta = 0.25$) loss functions
  \end{itemize}
\end{frame}

\begin{frame}
  \frametitle{Training Progress}

  \twofigs{Narrow Loss}
          {190611/plot_narrow.pdf}
          {Wide Loss}
          {190611/plot_wide.pdf}

  \begin{itemize}
  \item \textcolor{blue}{Old variables}
  \item \textcolor{red}{PF candidates added}
  \item \textcolor{green}{Previous overtraining}
  \end{itemize}

  \begin{itemize}
  \item Saved models every 20,000 steps
  \item Except for \textcolor{green}{Previous overtraining}
  \end{itemize}

\end{frame}

\begin{frame}
  \frametitle{Model differences (old)}

  \begin{center}
    \textcolor{green}{Old training} \\
    \includegraphics[width=\linewidth]{old_narrow_board.png} \\
  \end{center}

  \begin{itemize}
  \item Differences exist, difficult to say if significant
  \item Most nodes of the model are not used, \\
    possibly sub-optimal architecture
  \end{itemize}

\end{frame}

\begin{frame}
  \frametitle{Model differences (new)}

  \begin{center}
    \textcolor{red}{New training} \\
    \includegraphics[width=\linewidth]{pf_narrow_board.png} \\
  \end{center}

  \begin{itemize}
  \item Differences exist, difficult to say if significant
  \item Most nodes of the model are not used, \\
    possibly sub-optimal architecture
  \end{itemize}

\end{frame}

\begin{frame}
  \frametitle{Violin plots}

  Here's just an explanation of the resolution figure of merit

  \twofigs{$\eta$}
          {190611_violin_190607_pf_narrowloss_60000/bjetreg_2018_violin_eta_Hbb_dreg_2018_wide.pdf}
          {$p_T$}
          {190611_violin_190607_pf_narrowloss_60000/bjetreg_2018_violin_Hbb_dreg_2018_wide.pdf}

\end{frame}

\begin{frame}
  \frametitle{Trainings Over Time}
    
  \begin{itemize}
  \item Overtraining doesn't seem to be showing up anymore...
  \end{itemize}

  \twofigs{Narrow}
          {190611/plot_time_pf_narrow.pdf}
          {Wide}
          {190611/plot_time_pf_wide.pdf}

\end{frame}

\begin{frame}
  \frametitle{PF Candidate Gains}

  \fourfigs{40,000 steps}
           {190611/plot_time_40000_wide.pdf}
           {80,000 steps}
           {190611/plot_time_80000_wide.pdf}
           {120,000 steps}
           {190611/plot_time_120000_wide.pdf}
           {160,000 steps}
           {190611/plot_time_160000_wide.pdf}

\end{frame}

\begin{frame}
  \frametitle{Loss Function Difference}

  \twofigs{40,000 Steps}
          {190611/plot_time_40000_compare.pdf}
          {160,000 Steps}
          {190611/plot_time_160000_compare.pdf}

  \begin{itemize}
  \item Might be a hint of the overtraining we saw before
  \end{itemize}

\end{frame}

\begin{frame}
  \frametitle{Conclusions}

  \begin{itemize}
  \item Having trouble confirming improvement from PF candidates
  \item Either seeing overfitting, or bad loss function selection
  \item Most important metric will be the improvement of the di-jet mass peak,
    but not shown here
  \item Benedikt is trying synced framework with Keras model
  \end{itemize}

\end{frame}

\beginbackup

\begin{frame}
  \frametitle{Backup Slides}
\end{frame}

\begin{frame}
   \frametitle{\small 190611/plot\_time\_60000\_wide}
   \centering
   \includegraphics[width=0.6\linewidth]{190611/plot_time_60000_wide.pdf}
\end{frame}

\begin{frame}
   \frametitle{\small 190611/plot\_time\_wide}
   \centering
   \includegraphics[width=0.6\linewidth]{190611/plot_time_wide.pdf}
\end{frame}

\begin{frame}
   \frametitle{\small 190611/plot\_time\_120000\_compare}
   \centering
   \includegraphics[width=0.6\linewidth]{190611/plot_time_120000_compare.pdf}
\end{frame}

\begin{frame}
   \frametitle{\small 190611/plot\_time\_60000\_compare}
   \centering
   \includegraphics[width=0.6\linewidth]{190611/plot_time_60000_compare.pdf}
\end{frame}

\begin{frame}
   \frametitle{\small 190611/plot\_time\_80000\_compare}
   \centering
   \includegraphics[width=0.6\linewidth]{190611/plot_time_80000_compare.pdf}
\end{frame}

\begin{frame}
   \frametitle{\small 190611/plot\_time\_120000\_narrow}
   \centering
   \includegraphics[width=0.6\linewidth]{190611/plot_time_120000_narrow.pdf}
\end{frame}

\begin{frame}
   \frametitle{\small 190611/plot\_time\_40000\_narrow}
   \centering
   \includegraphics[width=0.6\linewidth]{190611/plot_time_40000_narrow.pdf}
\end{frame}

\begin{frame}
   \frametitle{\small 190611/plot\_time\_160000\_narrow}
   \centering
   \includegraphics[width=0.6\linewidth]{190611/plot_time_160000_narrow.pdf}
\end{frame}

\begin{frame}
   \frametitle{\small 190611/plot\_time\_60000\_narrow}
   \centering
   \includegraphics[width=0.6\linewidth]{190611/plot_time_60000_narrow.pdf}
\end{frame}

\begin{frame}
   \frametitle{\small 190611/plot\_time\_80000\_narrow}
   \centering
   \includegraphics[width=0.6\linewidth]{190611/plot_time_80000_narrow.pdf}
\end{frame}

\begin{frame}
   \frametitle{\small 190611/plot\_time\_narrow}
   \centering
   \includegraphics[width=0.6\linewidth]{190611/plot_time_narrow.pdf}
\end{frame}



\backupend

\end{document}
